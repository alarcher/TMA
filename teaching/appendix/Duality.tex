
\chapter{Duality}

\section{In finite dimension}

\begin{dfntn}[Dual space]
Let $E$ be a finite dimensional real vector space, its dual $E^\star$ is the space of linear forms on $E$, denoted by $\xLin(E;\xR)$.
\end{dfntn}

\begin{dfntn}[Dual basis]
Let $E$ be a finite dimensional real vector space, $\dim(E) = N$ and $\Basis = (e_1,\cdots,e_N)$ a basis of $E$.
Let us denote, for any $i,j \in \intdisc{1,N}$, by:
\begin{equation*}
\begin{array}{lllll}
e^\star_i\;:\: & E   & \fromto & \xR \\
  \hfill       & e_j & \mapsto & \delta_{ij}
\end{array}
\end{equation*}
the $i$-th coordinate.
The dual family of $\Basis$, $\Basis^\star = (e^\star_1,\cdots,e^\star_N)$ is a basis of $E^\star$.
\end{dfntn}

Thus we can write any element $u \in E$ as:
\begin{equation*}
u = \sum_{i=1}^N  e_i^\star(u) e_i
\end{equation*}
Proving that $\Basis^\star$ is a basis of $E^\star$ requires that $\Fam{e_i}$ generates $E^\star$ and that its elements are linearly independent.
The corollary of the first condition is that $\dim(\Basis^\star) = N$.
