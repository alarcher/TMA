
\chapter{Tensor formul\ae}

This short chapter is a reminder of tensor notations and identities that are useful for usual partial differential equations.

\section{Operators}

A tensor of order $p$ denotes an element of $\xR^{d_1\times\dots\times d_p}$, with $d_i$ dimension on the $i$-th axis: a zero-order tensor is a scalar, a first-order tensor is a vector, and a second-order tensor is a matrix.
The following operators are defined for second order tensors.


\subsection{Tensor product}
(order $p + q$)
\begin{equation}
(\tP \Tens \tQ)_{ijkl} = \tP_{ij} \tQ_{kl} \: \ve_i \Tens \ve_j \Tens \ve_k \Tens \ve_l
\end{equation}

\subsection{Dot product (simple contraction)}
(order $p + q -2$)
\begin{equation}
(\tP \cdot \tQ) = \Trace{p,p+1}{\tP \Tens \tQ}
\end{equation}
with $\Trace{p, p+1}{\cdot}$ the Trace operator with respect to indices $p$ and $p+1$.
In index notation, it consists of a summation on indices $p$ and $p+1$ (contraction),
\begin{equation}
(\tP \cdot \tQ) = \sum_j \tP_{ij}\tQ_{jk}
\end{equation}
Since summation occurs on a pair of indices, the order of the resulting tensor is reduced by two.

\subsection{Double-dot product (double contraction)}
(order $p + q -4$)
\begin{equation}
(\tP : \tQ) = \Trace{p-1,p+1}{ \Trace{p,p+2}{\tP \Tens \tQ}}
\end{equation}
with two contractions in this case, corresponding in index notation to summations over two  pairs of indices,
\begin{equation}
(\tP : \tQ) = \sum_i\sum_j \tP_{ij}\tQ_{ij}
\end{equation}
so that the order of the resulting tensor is reduced by four.
If $\tP = \tQ$ this corresponds to the Frobenius norm.

\subsection{Gradient}

The gradient of a scalar field is the first order tensor
\begin{equation}
\Grad f = \left[\pd{i} f\right]_i
\end{equation}
while the gradient of a vector field is the second order tensor
\begin{equation}
\Grad \vv = \left[\pd{j} \vv_i\right]_{ij}
\end{equation}
such that the derivative is applied to the last index.

\subsection{Divergence}
\begin{equation}
\Div(\tT) = \nabla\tT : \tG = \Trace{p,p+1}{\nabla\tT}
\end{equation}
with $\tG$ the metric tensor, which entries are the scalar product of the chosen basis vectors.
If the canonical basis is chosen, it is simply the identity matrix, therefore the divergence is simply the sum of diagonal entries of the gradient.

\subsection{Curl (Rotational)}
\begin{equation}
\nabla\Vect\tT = -\nabla\tT : \tH
\end{equation}
with $\tH$ is the orientation tensor, which entries are the mixed product of the chosen basis vectors. The curl operator is also denoted as $\nabla\times$.

\section{Identities}

\subsection{First order tensors}

\begin{itemize}
\item Gradient of a vector field:
\begin{equation}
\nabla(f\vv) = f \nabla\vv + \vv\Tens\nabla f
\end{equation}
which corresponds to the expansion of the derivative of a product, as the index notation shows
\begin{equation}
\pd{j}(f\vv_i) = f \pd{j}\vv_i + \vv_i\pd{j} f
\end{equation}

\item Divergence of a vector field:
\begin{equation}
\Div(f\vv) = f \Div(\vv) + \vv\cdot\nabla f
\end{equation}
which corresponds to the expansion of the derivative of a product, as the index notation shows
\begin{equation}
\pd{i}(f\vv_i) = f \pd{i}\vv_i + \vv_i\pd{i} f
\end{equation}

\item Identity of the advection operator:
\begin{equation}
(\vv\cdot\nabla)\vv = \frac{1}{2} \nabla(\vv\cdot\vv) + (\nabla\Vect\vv)\Vect\vv
\end{equation}

\item Identity for the Laplace operator:
\begin{equation}
\Delta\vv = \nabla(\Div\vv) - \nabla\Vect(\nabla\Vect \vv)
\end{equation}

\item Divergence of a vector product:
\begin{equation}
\Div(\uu\Vect\vv) = \vv\cdot(\nabla\Vect\uu) - \uu\cdot(\nabla\Vect\vv)
\end{equation}

\item Curl of of vector product:
\begin{equation}
\nabla\Vect(\uu\Vect\vv) = (\Div\vv)\uu - (\Div\uu)\vv + (\vv\cdot\nabla)\uu - (\uu\cdot\nabla)\vv\
\end{equation}
\end{itemize}


\subsection{Second order tensors}

\begin{itemize}
\item Dyadic/scalar mixed product:
\begin{equation}
(\uu\Tens\vv)\cdot\ww = (\vv\cdot\ww)\uu
\end{equation}

\item Gradient of a tensor field:
\begin{equation}
\nabla(\tT\cdot\vv) = \tT\cdot\nabla\vv + \vv\cdot\nabla\tT\T
\end{equation}

\item Divergence of a tensor field:
\begin{equation}
\Div(\vv\cdot\tT) = \vv\cdot\Div(\tT) + \tT : \nabla\vv
\end{equation}
which corresponds to the expansion of the derivative of a product, as the index notation shows
\begin{equation}
\pd{j}(\vv_i\tT_{ij}) = \vv_i\pd{j}\tT_{ij} + \tT_{ij}\pd{j}\vv_i
\end{equation}

\item Divergence of a dyadic product:
\begin{equation}
\Div(\uu\Tens\vv) = (\Div \vv)\uu + (\vv\cdot\nabla)\uu
\end{equation}
\end{itemize}
