
\chapter{Finite Element spaces}\label{sec:fem}

In the previous lectures we have studied the properties of coercive problems in an abstract setting and described Ritz and Galerkin methods for the approximation of the solution to a PDE, respectively in the case of symmetric and non-symmetric bilinear forms.

\medskip
The abstract setting reads:
\begin{equation*}
\left\lvert
\begin{array}{l}
\mbox{Find $\uh\in\xVh\subset \xH$ such that:}\\[2ex]
a(\uh,\vh) = L(\vh)\quad,\;\forany  \vh \in \xVh
\end{array}
\right.
\end{equation*}
such that:
\begin{itemize}
\item $\xVh$ is a finite dimensional approximation space characterized by a discretization parameter $h$,
\item $a(\xDot,\xDot)$ is a continuous bilinear form on $\xVh\times\xVh$, coercive \wrt $\norm{\xDot}_{\xV}$,
\item $L(\xDot)$ is a continuous linear form.
\end{itemize}

Under these assumptions existence and uniqueness of a solution to the approximate problem holds owing to the Lax--Milgram Theorem and $\uh$ is called \textit{discrete solution}.
Provided this abstract framework which allows us to seek approximate solutions to PDEs, we need now to define the discrete space $\xVh$ and construct a basis $(\varphi_1,\cdots,\varphi_{\NxVh})$ of $\xVh$, $\NxVh = \dim(\xVh)$, on which the discrete solution is decomposed as
\begin{equation*}
\uh = \sum_{j=1}^{\NxVh} u_j\; \varphi_j
\end{equation*}
with $\lbrace u_j \rbrace$ a family of $\NxVh$ real numbers called \textit{global degrees of freedom} and $\lbrace \varphi_j \rbrace$ a family of $\NxVh$ elements of $\xVh$ called \textit{global shape functions}.

\medskip
Previously no assumption was made on the finite dimensional space $\xV_n$ aside from that $\xV_n \subset \xV$.
The change of notation to $\xV_h$ is to reflect that the discrete space $\xV_h$ will be caracterized more carefully as an \textit{approximation space} by constructing the shape functions and by defining the degrees of freedom.

\medskip
To construct the finite element space $\xVh$, three ingredients are introduced:
\begin{enumerate}
\item An admissible mesh $\meshT$ generated by a tesselation of domain $\dom$.
\item A reference finite element $\RefFE$ to construct a basis of $\xVh$ and define the meaning of $u_j$.
\item A mapping that generates a finite element $\FE$ for any cell in the mesh from the reference element $\RefFE$.
\end{enumerate}

\medskip
As a preliminary step the approximation of the Poisson problem in one dimension by linear Lagrange finite elements is described to give an overview of the methodology without hitting the technnical difficulties.
Concepts and notations for the discretization of the physical domain are then introduced.
Provided that all the requirements are identified, a framework for all the finite element methods is introduced by stating the definition of a finite element.
Finally, the generation of the finite element space from a reference finite element will be described.
Some examples of finite element spaces are listed at the end of the chapter.

\section{A preliminary example in one dimension of space}

For the sake of completeness, steps performed to derive a Galerkin method for the Poisson Problem \ref{pb:poisson} on $\dom = (0, 1)$ are sketched below to recapitulate the methodology.

\medskip
A solution is sought in the distributional sense by testing the equation against smooth functions,
\begin{equation*}
- \int_\dom u''(x) v(x) \md x = \int_\dom f(x) v(x) \md x\:,\qquad\forany v \in \xCinfc(\dom)
\end{equation*}
then reporting derivatives on the test functions using the integration by part
\begin{equation*}
- \int_\dom u''(x) v(x) \md x = - \underbrace{[u'(x)v(x)]_0^1}_{= 0} + \int_\dom u'(x) v'(x) \md x
\end{equation*}
the weak formulation consists of finding $u\in\xV$ such that
\begin{equation*}
\int_\dom u'(x) v'(x) \md x = \int_\dom f(x) v(x) \md x\:,\qquad\forany v \in \xV
\end{equation*}
given $f \in \xLtwo(\dom)$.
The choice of solution space and test space is guided by the equation and the data: in this case $\xV = \xHonec(\dom)$ since $v$ and $v'$ should be controlled in $\xLtwo(\dom)$, and homogeneous Dirichlet boundary conditions are imposed.

\medskip
The approximate problem by a Galerkin method consists of seeking a discrete solution $\uh$ in a finite dimensional space $\xVh \in \xV$, such that
\begin{equation*}
\int_\dom \uh'(x) v'(x) \md x = \int_\dom f(x) v(x) \md x\:,\qquad\forany v \in \xVh
\end{equation*}
and given a basis $\Fam{\varphi_j}$ of $\xVh$ any function $w\in\xVh$ can be written as
\begin{equation*}
w_h = \sum_{j=1}^{\NxVh} \underbrace{w_j}_{\in\,\xR}\;\varphi_j
\end{equation*}
with $\NxVh = \dim{\xVh}$, and moreover
\begin{equation*}
w'_h = \sum_{j=1}^{\NxVh} \underbrace{w_j}_{\in\,\xR}\;\varphi'_j
\end{equation*}
by simple application of the derivation on the linear combination.
Bounds of the sum will be omitted to simplify the notation,
\begin{equation*}
\sum_{j=1}^{\NxVh} \sim \sum_{j}
\end{equation*}
 when there is no possible confusion.

\medskip
Inserting the Galerkin decomposition in the weak formulation and using the commutativity of the derivation with the linear combinations,
\begin{equation*}
\int_\dom \Bigl(\sum_{j} u_j\;\varphi'_j(x)\Bigr) \Bigl(\sum_{i} v_i\;\varphi'_i(x)\Bigr) \md x = \int_\dom f(x) \Bigl(\sum_{i} v_i\;\varphi_i(x)\Bigr) \md x
\end{equation*}
and the integration can also commute with the linear combinations,
\begin{equation*}
\sum_{j} \sum_{i} u_j v_i \int_\dom \varphi'_j(x)\varphi'_i(x) \md x = \sum_{i} v_i\int_\dom f(x) \varphi_i(x) \md x
\end{equation*}
so that up to some cosmetic reordering, for any $v \in \xV$
\begin{equation*}
\sum_{i}  \sum_{j} v_i \int_\dom \varphi'_i(x)\varphi'_j(x) \md x u_j = \sum_{i} v_i\int_\dom f(x) \varphi_i(x) \md x
\end{equation*}
the relation to the linear system of algebraic equations becomes evident,
\begin{equation*}
\vecv^\trans \matA\;\vecu = \vecv^\trans \vecb
\end{equation*}
for any $\vecv = [v_i]\in \xR^\NxVh$, with
\begin{equation*}
\matA = \left[\int_\dom \varphi'_j(x)\varphi'_i(x) \md x\right]_{ij},\quad\vecb = \left[\int_\dom f(x) \varphi_i(x) \md x\right]_{i}
\end{equation*}
and the discrete solution is represented by the solution vector $\vecu = [u_j]\in \xR^\NxVh$.
Computing contributions $\matA_{ij}$ and $\vecb_i$ is possible as soon as the basis $\Fam{\varphi}_j$ is constructed explicitly.

\medskip
\begin{rmrk}
The choice of indices $i$ and $j$ in the previous expressions follows the usual convention for row and column indices. The matrix $\matA$ represents a linear application from the solution space $\xH$ to the trial space $\xV$: therefore the solution space is the column space, while the trial space is the row space.
In the case of Galerkin approximations where $\xH = \xV$ -- and even more when the bilinear form is symmetric -- it is tempting to choose the indices arbitrarily, but for the sake of consistency following the convention is recommended.
\end{rmrk}

\medskip
A discretization of the computational domain $\bar\dom = [0,1]$ is constructed by partitioning the interval into disjoints subintervals $K_i = [x_{i}, x_{i+1}]$, $1\leq i \leq N_K$.
\begin{equation*}
\bar\dom = \displaystyle\Union_{i = 1}^{N_K} K_i\;,\qquad \Interior{K_i}\cap \Interior{K_j} = \Empty
\end{equation*}
The family of cells $\Fam{K_i}$ defines a \textit{mesh} noted $\meshT$.

\medskip
In this example the approximation space is constructed with piecewise linear Lagrange polynomials.
The chosen discrete space is
\begin{equation}\label{sp:lagrangeP1}
\xVh = \lbrace v \in \xC^0(\bar\dom)\cap\xHonec(\dom) : \forany K \in \meshT, v\rvert_{K} \in \xPone(K) \rbrace
\end{equation}
which consists of functions continuous over $\dom$ that are linear on each cell $K$, and $\xVh \subset \xV$ so that the approximation is $\xHonec$--conformal.

\begin{figure}[h!]
\centering
\begin{tikzpicture}
  \begin{axis}[
    xmin=0.,
    xmax=1.,
    ymin=0.,
    ymax=1.,
    axis x line*=bottom,
    axis y line*=center,
    /pgfplots/xtick={0., 0.2, ..., 1},
    /pgfplots/ytick={0., 0.2, ..., 1},
    width=\textwidth,
    height=\axisdefaultheight
    ]
    \addplot[darkblue, thick, domain=0:1, samples=100]{0.80*(sin(x*180.) - 0.4*sin(2.*x*180.))};
    \addplot[black, domain=0:1, samples=6]{0.80*(sin(x*180.) - 0.4*sin(2.*x*180.))};
    \addplot[ybar, bar width=0pt, black, dotted, thin, domain=0:1, samples=6]{0.80*(sin(x*180.) - 0.4*sin(2.*x*180.))};
  \end{axis}
\end{tikzpicture}
\caption{The function $v(x) = 0.8 \sin(2\pi x) - 0.32 \sin(4\pi x)$ and its linear interpolate $\Iop{\xVh} v$.}
\end{figure}

\medskip
The construction of the basis $\lbrace\varphi_j\rbrace$ of $\xVh$ should produce a linear system that can be solved easily so the matrix should be as sparse as possible.
Given the expression of the contributions $\matA_{ij}$ the requirement is that functions $\varphi_j$ overlap as little as possible with each other: the support of $\varphi_j$ should be reduced so that contributions are non-zero only for neighbouring $\varphi_j$ and $\varphi_i$ functions, and this is consistent with the locality of differential operators.

\begin{figure}[h!]
\centering
\begin{tikzpicture}
  \begin{axis}[
    xmin=0.,
    xmax=1.,
    ymin=0.,
    ymax=1.1,
    axis x line*=bottom,
    axis y line*=center,
    xtick={0.0,0.3,0.4,0.5,0.6,0.7,1.0},
    xticklabels={$0$, $x_{i-2}$, $x_{i-1}$, $x_{i}$, $x_{i+1}$, $x_{i+2}$, $1$},
    /pgfplots/ytick={0., 1.},
    width=\textwidth,
    height=\axisdefaultheight
    ]
    \addplot[black, dashed, domain=0.3:0.4]{10*(x-0.3)} node[above] {$\varphi_{i-1}$};
    \addplot[black, dashed, domain=0.4:0.5]{10*(0.5-x)};
    \addplot[black, domain=0.4:0.5]{10*(x-0.4)} node[above] {$\varphi_i$};
    \addplot[black, domain=0.5:0.6]{10*(0.6-x)};
    \addplot[black, dashed, domain=0.5:0.6]{10*(x-0.5)} node[above] {$\varphi_{i+1}$};
    \addplot[black, dashed, domain=0.6:0.7]{10*(0.7-x)};
    \addplot[black, dotted, thin, domain=0:1, samples=6]{1};
  \end{axis}
\end{tikzpicture}
\caption{Shape functions for Lagrange $\xPone$ on the unit interval.}
\end{figure}

\medskip
The support of constructed functions $\varphi_i$ overlaps only with $\varphi_{i-1}$ and $\varphi_{i+1}$, and the expression of all the functions can be obtained from one another by an affine transformation.

\section{Admissible mesh}

\begin{dfntn}[Mesh]
Let $\dom$ be polygonal ($d=2$) or polyhedral ($d=3$) subset of $\xR^d$, we define $\meshT$ (a triangulation in the simplicial case) as a finite family $\Fam{K_i}$ of disjoints non-empty subsets of $\dom$ named cells.
Moreover $\mathcal N_h = \Fam{\mathcal N_i}$ denotes the set a vertices of $\meshT$ and $\varepsilon_h = \Fam{\sigma_{KL} = K\cap L}$ denotes the set of edges.
\end{dfntn}

\begin{dfntn}[Mesh size]
\begin{equation*}
\sizeT = \max_{K\in\meshT}(\diam(K))
\end{equation*}
with $\diam(K)$ with diameter of the cell, \ie the maximum distance between two points of $K$.
\end{dfntn}

\begin{dfntn}[Geometrically conforming mesh]
A mesh is said geometrically conforming if two neighbouring cells share either exactly one vertex, exactly one edge in the case $d = 2$, or in the case $d = 3$ exactly one face.
\end{dfntn}

\medskip
The meaning of the previous condition is that there should not be any ``hanging node'' on a facet.
Moreover some theoretical results require that the mesh satisfies some regularity condition: for example, bounded ratio of equivalent ball diameter, Delaunay condition on the angles of a triangle, \dots

\section{Construction of a Finite Element}

\begin{dfntn}[Finite Element -- \cite{EG} page 19, \cite{BS} page 69]
A Finite Element consists of a triple $\FE$, such that
\begin{itemize}
\item $\CellK$ is a compact, connected subset of $\xR^d$ with non-empty interior and with regular boundary (typically Lipshitz continuous),
\item $\SpaceP$ is a finite dimensional vector space of functions $p : K \rightarrow \xR$, which is the space of shape functions; $\dim(\SpaceP) = N$.
\item $\DualBasis$ is a set $\Fam \sigma_j$ of linear forms,
\begin{equation*}
\begin{array}{lllll}
\sigma_j:& \SpaceP & \rightarrow & \xR &\qquad  ,\; \forany j \in \intdisc{1,N}\\[2ex]
\hfill   & p       & \mapsto     & p_j = \sigma_j(p) &
\end{array}
\end{equation*}
which is a basis of $\xLin(\SpaceP,\xR)$, the dual of $\SpaceP$.
\end{itemize}
\end{dfntn}

\medskip
Practically, the definition  requires first to consider the Finite Element on a cell $K$ which can be an interval ($d=1$), a polygon ($d=2$) or a polyhedron ($d=3$) (Example: triangle, quadrangle, tetrahedron, hexahedron), then an approximation space $\SpaceP$ (Example: polynomial space) and the local degrees of freedom $\Sigma$ are chosen (Example: value at $N$ geometrical nodes $\Fam{a_i}$, $\sigma_i(\varphi_j) = \varphi_j(a_i)$).
The local shape functions $\Fam{\varphi_i}$ are then constructed so as to ensure unisolvence.

\medskip
\begin{prpstn}[Determination of the local shape functions]
Let $\Fam{\sigma_i}_{1 \leq i \leq N}$ be the set of \textit{local degrees of freedoms}, the \textit{local shape functions} are defined as $\Fam{\varphi_i}_{1 \leq i \leq N}$ a basis of $\SpaceP$ such that,
\begin{equation*}
\sigma_i(\varphi_j) = \delta_{ij}\quad,\;\forany i, j \in \intdisc{1,N}
\end{equation*}
\end{prpstn}

\begin{dfntn}[Unisolvence] A Finite Element is said unisolvent if for any vector $(\alpha_1, \cdots, \alpha_N) \in \xR^N$ there exists a unique representant $p \in \SpaceP$ such that $\sigma_i(p) = \alpha_i$, $\forany \in \intdisc{1,N}$.
\end{dfntn}

The unisolvence property of a Finite Element is equivalent to construct $\Sigma$ as dual basis of $\SpaceP$, thus we can express any function $p \in \SpaceP$ as
\begin{equation*}
p = \sum_{j=1}^{N} \sigma_j(p)\; \varphi_j
\end{equation*}
the unique decomposition on $\Fam{\varphi_j}$, with $p_j = \sigma_j(p)$ the $j$-th degree of freedom.
In other words, the choice of $\Sigma = \Fam{\sigma_j}$ ensures that the vector of degree of freedoms $(p_1,\cdots, p_N)$ uniquely defines a function of $\SpaceP$.
Defining $\Sigma$ as dual basis of $\SpaceP$ is equivalent to:
\begin{subequations}
\begin{equation}\label{unisolvence:generates}
\dim(\SpaceP) = \card(\Sigma) = N
\end{equation}
\begin{equation}\label{unisolvence:free}
\forany p \in \SpaceP,\;(\sigma_i(p) = 0, 1 \leq i \leq N) \Rightarrow (p = 0)
\end{equation}
\end{subequations}
in which Property \eqref{unisolvence:generates} ensures that $\Sigma$ generates $\xLin(\SpaceP,\xR)$ and Property \eqref{unisolvence:free} that $\Fam{\sigma_i}$ are linearly independent.

\medskip
Usually the unisolvence is part of the definition of a Finite Element since chosing the shape functions such that $\sigma_i(\varphi_j) = \delta_{ij}$ is equivalent.

\begin{dfntn}[Local interpolation operator -- \cite{EG} page 20]
\begin{equation*}
\begin{array}{llll}
\IopK{\xV}: & \xV(K) & \fromto & \SpaceP\\[2ex]
\hfill    & v    & \mapsto & \displaystyle\sum_{j=1}^{N} \sigma_j(v)\; \varphi_j
\end{array}
\end{equation*}
\end{dfntn}

\begin{rmrk}
The notation using the dual basis can be confusing but with the relation $\sigma_i(p) = p(a_i)$ in the nodal Finite Element case it is easier to understand that the set $\Sigma$ of linear forms defines how the interpolated function $\IopK{\xV}u$ ``represents'' its infinite dimensional counterpart $u$ through the definition of the degrees of freedom.
In the introduction, we defined simply $u_i = \sigma_i(u)$ without expliciting it. A natural choice is the pointwise representation $u_i = u(a_i)$ at geometrical nodes $\Fam{a_i}$, which is the case of Lagrange elements, but it is not the only possible choice!
For example, $\sigma_i$ can be:
\begin{itemize}
\item a mean flux trough each facet of the element (Raviart--Thomas)
\begin{equation*}
\sigma_i(v) = \int_\xi v\xDot\n_\xi\ds
\end{equation*}
\item a mean value over each facet of the element (Crouzeix--Raviart)
\begin{equation*}
\sigma_i(v) = \int_\xi v\ds
\end{equation*}
\item a mean value of the tangential component over each facet of the element (Nédelec)
\begin{equation*}
\sigma_i(v) = \int_\xi v\xDot\boldsymbol\tau_\xi\ds
\end{equation*}
\end{itemize}
A specific choice of linear form allows a control on a certain quantity: divergence for the first two examples, and curl for the third.
The approximations will then not only be $\xH^s$-conformal but also include the divergence or the curl in the space.
\end{rmrk}

\section{Transport of the Finite Element}

In practice to avoid the construction of shape functions for any Finite Element $\FE$, $K \in \meshT$, the local shape functions are evaluated for a \textit{reference Finite Element} $\RefFE$ defined on a \textit{reference cell} $\RefK$ and then transported onto any cell $K$ of the mesh.
For example, in the case of simplicial meshes the reference cell in one dimension is the unit interval $[0,1]$, in two dimension the unit triangle with vertices $\Set{(0,0),(0,1),(1,0)}$.
In so doing, we can generate any Finite Element $\FE$ on the mesh from $\RefFE$ provided that we can construct a mapping such that $\FE$ and $\RefFE$ are equivalents.

\begin{dfntn}[Equivalent Finite Elements]
Two Finite Elements $\FE$ and $\RefFE$ are said \textit{equivalent} if there exists a bijection $T_K$ from $\RefK$ onto $\CellK$  such that:
\begin{equation*}
\forany p \in \SpaceP,\; p\mathop\circ T_K \in \RefP
\end{equation*}
and
\begin{equation*}
\Sigma = T_K(\RefSigma)
\end{equation*}
\end{dfntn}

\medskip
By collecting the local shape functions and local degrees of freedom from all the generated $\FE$ on the mesh, we then construct \textit{global shape functions} and \textit{global degrees of freedom} and thus the approximation space $\xVh$.

\medskip
For Lagrange elements the transformation used to transport the Finite Element on the mesh is an \textit{affine mapping}, but this is not suitable in general!

%\section{High-order finite elements}

%\section{Non-conforming finite elements}

\section{Numerical integration}

The contributions are integrated numerically, usually using quadrature rules.

\section{Method}

\begin{lgrthm}[Finite Element Method]\label{alg:fem}

Solving a problem by a Finite Element Method is defined by the following procedure:
\begin{enumerate}
\item Choose a reference finite element $\RefFE$.
\item Construct an admissible mesh $\meshT$ such that any cell $\CellK \in \meshT$ is in bijection with the reference cell $\RefK$.
\item Define a mapping to transport the reference finite element defined on $\RefK$ onto any $\CellK \in \meshT$ to $\FE$.
\item Construct a basis for $\xVh$ by collecting all the finite element basis of finite elements $\lbrace \FE \rbrace_{\CellK \in \meshT}$ sharing the same degree of freedom.

\end{enumerate}
\end{lgrthm}

\begin{rmrk}
The Finite Element approximation is said $\xH$-conformal if $\xVh\subset \xH$ and is said non-conformal is $\xVh\not\subset \xH$. In this latter case the approximate problem can be constructed by building an approximate bilinear form
\begin{equation*}
a_h(\xDot,\xDot) = a(\xDot,\xDot) + s(\xDot,\xDot)
\end{equation*}
as described, for instance, in the case of stabilized methods for advection-dominated problems in Section \ref{ssec:stab_galerkin}.
\end{rmrk}


\section{Exercises}



