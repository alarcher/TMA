
\chapter{Simplicial Lagrange Finite Elements}\label{sec:lagrange}

%-------------------------------------------------------------------------------
\section{Polynomial interpolation in one dimension}

Let $\xP_k([a,b])$ be the space of polynomials $p = \sum_{i=0}^k \alpha_i x^i$ of degre lower or equal to $k$ on the interval $[a,b]$, with $c_i x^i$ the monomial of order $i$, $c_i$ a real number.

\medskip
A natural basis of $\xP_k([a,b])$ consists of the set of monomials $\Set{1,x,x^2,\cdots,x^k}$.
Its elements are linearly independent but in the frame of Finite Elements we can chose another basis which is the Lagrange basis $\Set{\Lgr{k}{i}}_{0 \leq i \leq k}$ of degree $k$ defined on a set of $k+1$ points $\Fam{\xi_i}_{0 \leq i \leq k}$ which are called \textit{Lagrange nodes}.

\begin{dfntn}[Lagrange polynomials -- \cite{EG} page 21, \cite{CDE} page 76]
\label{def:lagrange_poly}
The Lagrange polynomial of degree $k$ associated with node $\xi_m$ reads:
\begin{equation*}
\Lgr{k}{m}(x) = \dfrac{\prod_{\substack{i=0\\i\not=m}}^k (x - \xi_i)}{\prod_{\substack{i=0\\i\not=m}}^k (\xi_m - \xi_i)}
\end{equation*}
\end{dfntn}

\begin{prpstn}[Nodal basis -- \cite{EG}]
\begin{equation*}
\Lgr{k}{i}(\xi_j) = \delta_{ij}\quad,\;0 \leq i,j \leq k
\end{equation*}
\end{prpstn}

\bigskip
The following result gives a pointwise control of the interpolation error:
\begin{thrm}[Pointwise interpolation inequality -- \cite{CDE} page 79]
\label{th:polyinterpol}
Let $u\in\xC^{k+1}([a,b])$ and $\Projh{k} u \in \xP_k([a,b])$ its Lagrange interpolate of order $k$, with Lagrange nodes $\Fam{\xi_i}_{0\leq i \leq k}$, then $\forany x \in [a,b]$:
\begin{equation*}
\Abs{u(x) - \Projh{k} u(x)} \leq \Abs{\dfrac{\prod_{i=0}^k (x - \xi_i)}{(k+1)!}}\max_{[a,b]}\Abs{\partial^{k+1} u}
\end{equation*}
\end{thrm}

%-------------------------------------------------------------------------------
\section{A nodal element}

Let us take $\Fam{\xi_1, \cdots, \xi_N}$ a family of points of $K$ such that $\sigma_i(p)= p(\xi_i)$, $1\leq i \leq N$:
\begin{itemize}
\item $\Fam{\xi}_{1\leq i \leq N}$ is the set of \textit{geometrical nodes},
\item $\Fam{\varphi_i}_{1\leq i \leq N}$ is a \textit{nodal basis} of $\SpaceP$, \ie $\varphi_i(\xi_j) = \delta_{ij}$.
\end{itemize}
We can verify, for any $p \in\SpaceP$ that:
\begin{equation*}
p(\xi_j) = \sum_{i=1}^{N} \sigma_i(p)\; \underbrace{\varphi_i(\xi_i)}_{\delta_{ij}}\quad,\; 1 \leq i,j \leq N
\end{equation*}
which reduces to:
\begin{equation*}
p(\xi_j) = \sigma_i(p)
\end{equation*}

\medskip
\begin{rmrk}[Support of shape functions]
The polynomial basis being defined such that $\varphi_j(\xi_i) = \delta_{ij}$ then any shape function $\varphi_i$ is compactly supported on the union of cells containing the node $\xi_i$.
\end{rmrk}

The Lagrange polynomials \eqref{def:lagrange_poly} are used to build directly the one-dimensional shape functions, while in higher dimensions the expression of the shape functions is reformulated in term of barycentric coordinates:
\begin{eqnarray*}
\lambda_i:& \xR^d &\fromto \xR\\
          & \bfx  &\mapsto \lambda_i(\bfx) = 1 - \dfrac{(\bfx - \mathbf\xi_i)\cdot\n_i}{(\mathbf\xi_{f} - \mathbf\xi_i)\cdot\n_i}
\end{eqnarray*}
with $\n_i$ the unit outward normal to the facet opposite to $\xi_i$, and $\xi_f$ a node belonging to this facet.

%-------------------------------------------------------------------------------
\section{Reference Finite Element}

%-------------------------------------------------------------------------------
\section{Lagrange $\xP_k$ elements}

\begin{figure}[h!]\label{fig:lagrange_P1_1D}
\centering
\begin{tikzpicture}
  \begin{axis}[
    xmin=0.,
    xmax=1.,
    ymin=-0.4,
    ymax=1.2,
    axis x line*=middle,
    axis y line*=middle,
    xtick={0.0,1.0},
    xticklabels={$0$, $1$},
    /pgfplots/ytick={0., 1.},
    width=\axisdefaultwidth,
    height=\axisdefaultheight
    ]
    \addplot[black, domain=0:1]{1.-x} node[above] at (10,140) {$\hat{\varphi}_{0}$};
    \addplot[black, domain=0:1]{x} node[above] at (90,140) {$\hat{\varphi}_{1}$};
    \addplot[black, dotted, thin, domain=0:1, samples=2]{1};
    \addplot[black, dotted, thin] coordinates { (1,0) (1,1)};
  \end{axis}
\end{tikzpicture}
\caption{Shape functions for Lagrange $\xPone$ on the interval $\hat{K} = [0,1]$.}
\end{figure}

\begin{figure}[h!]\label{fig:lagrange_P2_1D}
\centering
\begin{tikzpicture}
  \begin{axis}[
    xmin=0.,
    xmax=1.,
    ymin=-0.4,
    ymax=1.2,
    axis x line*=middle,
    axis y line*=middle,
    xtick={0.0,1.0},
    xticklabels={$0$, $1$},
    /pgfplots/ytick={0., 1.},
    width=\axisdefaultwidth,
    height=\axisdefaultheight
    ]
    \addplot[black, domain=0:1]{2*(x-1)*(x-0.5)} node[above] at (10,140) {$\hat{\varphi}_{0}$};
    \addplot[black, domain=0:1]{4*x*(1-x)} node[above] at (50,140) {$\hat{\varphi}_{1}$};
    \addplot[black, domain=0:1]{2*x*(x-0.5} node[above] at (90,140) {$\hat{\varphi}_{2}$};
    \addplot[black, dotted, thin, domain=0:1, samples=2]{1};
    \addplot[black, dotted, thin] coordinates { (1,0) (1,1)};
    \addplot[darkblue, dotted, thin] coordinates { (0.5,0) (0.5,1)};
  \end{axis}
\end{tikzpicture}
\caption{Shape functions for Lagrange $\xPtwo$ on the interval $\hat{K} = [0,1]$.}
\end{figure}

\begin{figure}[h!]\label{fig:lagrange_P3_1D}
\centering
\begin{tikzpicture}
  \begin{axis}[
    xmin=0.,
    xmax=1.,
    ymin=-0.4,
    ymax=1.2,
    axis x line*=middle,
    axis y line*=middle,
    xtick={0.0,1.0},
    xticklabels={$0$, $1$},
    /pgfplots/ytick={0., 1.},
    width=\axisdefaultwidth,
    height=\axisdefaultheight
    ]
    \addplot[black, domain=0:1, samples=100]{-4.5*(x-1)*(x-2/3)*(x-1/3))} node[above] at (10,140) {$\hat{\varphi}_{0}$};
    \addplot[black, domain=0:1, samples=100]{+13.5*(x-1)*(x-2/3)*x} node[above] at (40,140) {$\hat{\varphi}_{1}$};
    \addplot[black, domain=0:1, samples=100]{-13.5*(x-1)*(x-1/3)*x} node[above] at (60,140) {$\hat{\varphi}_{2}$};
    \addplot[black, domain=0:1, samples=100]{+4.5*(x-2/3)*(x-1/3)*x} node[above] at (90,140) {$\hat{\varphi}_{3}$};
    \addplot[black, dotted, thin, domain=0:1, samples=2]{1};
    \addplot[black, dotted, thin] coordinates { (1,0) (1,1)};
    \addplot[darkblue, dotted, thin] coordinates { (1/3,0) (1/3,1)};
    \addplot[darkblue, dotted, thin] coordinates { (2/3,0) (2/3,1)};
  \end{axis}
\end{tikzpicture}
\caption{Shape functions for Lagrange $\xP_3$ on the interval $\hat{K} = [0,1]$.}
\end{figure}

%-------------------------------------------------------------------------------
\section{Extension to multiple dimensions}

\begin{xmpl}[Lagrange elements of polynomial degree $k=1,2$ on triangle]
The shape functions are given by:
\begin{equation*}
\begin{array}{lrlll}
k = 1, & \varphi_i &=& \lambda_i &,\;0 \leq i \leq d\\[2ex]
k = 2, & \varphi_i &=& \lambda_i (2 \lambda_i - 1) & ,\;0 \leq i \leq d\\
         & \varphi_i &=& 4 \lambda_i \lambda_j       & ,\;0 \leq i \leq d
\end{array}
\end{equation*}
\end{xmpl}

%-------------------------------------------------------------------------------
\section{Examples in one dimension}

The approximate problem of Problem \eqref{pb:weak_poissonHone} by Lagrange $\xP_1$ elements reads:
\begin{subequations}\label{pb:poisson_p1}
\begin{equation}
\left\lvert
\begin{array}{ll}
\mbox{Find $u \in \xVh$, given $f \in \xLtwo(\dom)$, such that:}\\[2ex]
\displaystyle\int_\dom \Grad u\xDot \Grad v\dx = \int_\dom f v  \dx\quad,\;\forany  v\in \xVh
\end{array}
\right .
\end{equation}
with the approximation space $\xVh$ chosen as:
\begin{equation}
\xVh = \Set{ v \in \xCzero(\bar\dom)\cap\xHonec(\dom) : v|_K \in\xP_1(K),\forany K \in\meshT }
\end{equation}
\end{subequations}


\section{Exercises}
