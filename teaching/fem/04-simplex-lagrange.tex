
\chapter{Simplicial Lagrange Finite Elements}\label{sec:lagrange}

%-------------------------------------------------------------------------------
\section{Definitions}

Vector spaces of polynomials are used as approximation spaces to construct Finite Elements.

\begin{dfntn}[Space of polynomials with real coefficients]
Let $K\subset \xR^d$, $k \in \xN$, $\xP_k(K)$ is the vector space of polynomials with real coefficients of degree $k$ on $K$, and the canonical basis is given by the family $\Fam{ x_1^{\alpha_1}\cdots x_i^{\alpha_i}\cdots x_d^{\alpha_d} : |\alpha| = k }$, with $x_i$ the $i$-th coordinate of $\xx\in\xR^d$.
\end{dfntn}

The dimension of the space is given by
\begin{equation*}
\dim(\xP_k(\xR^d)) = \frac{1}{d!} \prod_{i=1}^d (k + i)
\end{equation*}
so that in particular $\dim(\xP_k(\xR^1)) = k + 1$, which means that such polynomials will be uniquely defined by its values at $k+1$ nodes.

\begin{xmpl}
Linear and quadratic polynomials in different dimensions of space are listed below.
\begin{equation*}
\begin{array}{lcl}
\xPone(\xR^1) &=& \Span\Fam{ 1, x } \\
\xPone(\xR^2) &=& \Span\Fam{ 1, x, y } \\
\xPone(\xR^3) &=& \Span\Fam{ 1, x, y, z } \\
\xPtwo(\xR^1) &=& \Span\Fam{ 1, x, x^2 } \\
\xPtwo(\xR^2) &=& \Span\Fam{ 1, x, y, xy, x^2, y^2 } \\
\xPtwo(\xR^3) &=& \Span\Fam{ 1, x, y, z, xy, yz, xz, x^2, y^2, z^2 }
\end{array}
\end{equation*}
\end{xmpl}

Simplicial Lagrange Finite Elements are considered, for which the approximation space $\SpaceP$ will be polynomials on $K$, a $d$-simplex.
Simplices are a generalization of triangles to $d$ dimensions, which consists of intervals ($d=1$), triangles ($d=2$), or tetrahedra ($d=3$).

\begin{dfntn}[Simplex]
Let $\Fam{\vertex_i}$ be a family of $d + 1$ points of $\xR^d$ that do not belong to the same hyperplane, the associated $d$-simplex $K$ is the convex hull of these points.
Points $\Fam{\vertex_i}$ are called \textit{vertices} of the simplex, and pairs $\epsilon_{ij} = (\vertex_i, \vertex_j)$, $i\neq j$, consist of the eddges.
The diameter of the simplex is the maximum Euclidean distance between two vertices,
\[
\diam(K) = \displaystyle\max_{1\leq i,j\leq d+1}\norm{\vertex_i - \vertex_j}
\]
\end{dfntn}

The convex hull is the minimum subset of $\xR^d$ enclosing the points.
The condition that all points are not in the same hyperplane means that the convex hull does not degenerate into a lower-dimensional entity; for instance in two dimensions, a triangle degenerates to a segment when points are aligned, and in three dimensions, a tetrahedron degenerates if all points are in the same plane.
Therefore a degenerate simplex has a zero $d$-dimensional Lebesgue measure.

\medskip
Consider the matrix $M_d$ of $\xR^{(d+1)\times(d+1)}$ consisting of column vectors with coordinates of vertices $\Fam{\vertex_i}$ of $K$, and completed by a unit row.
\begin{equation*}
M_1 =
\begin{bmatrix}
x_0 & x_1 \\
1   & 1   \\
\end{bmatrix}
\qquad
M_2 =
\begin{bmatrix}
x_0 & x_1 & x_2 \\
y_0 & y_1 & y_2 \\
1   & 1   & 1   \\
\end{bmatrix}
\qquad
M_3 =
\begin{bmatrix}
x_0 & x_1 & x_2 & x_3 \\
y_0 & y_1 & y_2 & y_3 \\
z_0 & z_1 & z_2 & z_3 \\
1   & 1   & 1   & 1   \\
\end{bmatrix}
\end{equation*}

The determinant of $M_d$ gives the signed $d$-dimensional measure (volume) of the corresponding simplex $K$.
\begin{equation*}
\det(M_d) = \pm 2 |K|
\end{equation*}
In particular, if all the points are contained in the same hyperplane then any vertex is the linear combinations of others so that the determinant is zero.
Given that the sign of the determinant depends on permutations of the matrix $M_d$, in practice the meaning of the sign is the \textit{orientation} of the simplex which depends on how vertices are numbered.


%-------------------------------------------------------------------------------
\section{Polynomial interpolation in one dimension}

Let $\xP_k([a,b])$ be the space of polynomials $p = \sum_{i=0}^k \alpha_i x^i$ of degre lower or equal to $k$ on the interval $[a,b]$, with $c_i x^i$ the monomial of order $i$, $c_i$ a real number.

\medskip
A natural basis of $\xP_k([a,b])$ consists of the set of monomials $\Set{1,x,x^2,\cdots,x^k}$.
Its elements are linearly independent but in the frame of Finite Elements we can chose another basis which is the Lagrange basis $\Set{\Lgr{k}{i}}_{0 \leq i \leq k}$ of degree $k$ defined on a set of $k+1$ points $\Fam{\xi_i}_{0 \leq i \leq k}$ which are called \textit{Lagrange nodes}.

\begin{dfntn}[Lagrange polynomials -- \cite{EG} page 21, \cite{CDE} page 76]
\label{def:lagrange_poly}
The Lagrange polynomial of degree $k$ associated with node $\xi_m$ reads
\begin{equation*}
\Lgr{k}{m}(x) = \frac{\displaystyle\prod_{\substack{i=0\\i\not=m}}^k (x - \xi_i)}{\displaystyle\prod_{\substack{i=0\\i\not=m}}^k (\xi_m - \xi_i)}
\end{equation*}
and
\begin{equation*}
\sum_{i=0}^{k}\Lgr{k}{i}(x) = 1
\end{equation*}
\end{dfntn}

\begin{prpstn}[Nodal basis]
\begin{equation*}
\Lgr{k}{i}(\xi_j) = \delta_{ij}\quad,\;0 \leq i,j \leq k
\end{equation*}
\end{prpstn}


\bigskip
The following result gives a pointwise control of the interpolation error.
\begin{thrm}[Pointwise interpolation inequality -- \cite{CDE} page 79]
\label{th:polyinterpol}
Let $u\in\xC^{k+1}([a,b])$ and $\Projh{k} u \in \xP_k([a,b])$ its Lagrange interpolate of order $k$, with Lagrange nodes $\Fam{\xi_i}_{0\leq i \leq k}$, then $\forany x \in [a,b]$:
\begin{equation*}
\Abs{u(x) - \Projh{k} u(x)} \leq \frac{\Abs{\displaystyle\prod_{i=0}^k (x - \xi_i)}}{(k+1)!}\max_{s\in[a,b]}\Abs{\partial^{k+1} u(s)}
\end{equation*}
\end{thrm}

%-------------------------------------------------------------------------------
\section{Construction of the Finite Element space}

%-------------------------------------------------------------------------------
\subsection{A nodal element}

Let us take $\Fam{\Pnode_1, \cdots, \Pnode_N}$ a family of points of $K$ such that $\sigma_i(p)= p(\Pnode_i)$, $1\leq i \leq N$:
\begin{itemize}
\item $\Fam{\Pnode_i}_{1\leq i \leq N}$ is the set of \textit{geometric nodes},
\item $\Fam{\varphi_i}_{1\leq i \leq N}$ is a \textit{nodal basis} of $\SpaceP$, \ie $\varphi_i(\Pnode_j) = \delta_{ij}$.
\end{itemize}
We can verify, for any $p \in\SpaceP$ that:
\begin{equation*}
p(\Pnode_j) = \sum_{i=1}^{N} \sigma_i(p)\; \underbrace{\varphi_i(\Pnode_i)}_{\delta_{ij}}\quad,\; 1 \leq i,j \leq N
\end{equation*}
which reduces to:
\begin{equation*}
p(\Pnode_j) = \sigma_i(p)
\end{equation*}

\medskip
\begin{rmrk}[Support of shape functions]
The polynomial basis being defined such that $\varphi_j(\Pnode_i) = \delta_{ij}$ then any shape function $\varphi_i$ has support on the union of cells containing the node $\Pnode_i$.
\end{rmrk}

\subsection{Reference Finite Element}

To make the connection between the abstract Definition \ref{def:finite_element} and a simple concrete example, the reference element for Lagrange $\xP_1$ in one dimension is given by the following triple $\RefFE$.

\medskip
\begin{dfntn}[$\RefFE$ for Lagrange $\xPone$ 1D]\label{fem:lagrange_p1_1d}
The Finite Element space Lagrange $\xPone$ 1D approximating
\begin{equation*}
\xVh = \Set{ v \in \xCzero(\bar\dom)\cap\xHone(\dom) : v|_K \in\xP_1(K),\forany K \in\meshT }
\end{equation*}
is given by
\begin{itemize}
\item $\hat K$ is the unit interval $[0,1]$.
\item $\hat\SpaceP$ is the space of linear polynomials $\xPone([0,1])$ with the basis $(\hat\varphi_0, \hat\varphi_1)$ by Definition \ref{def:lagrange_poly} of $\Lgr{1}{i}$,
\begin{equation*}
\hat\varphi_0(\hat x) = \Lgr{1}{0}(\hat x) = 1 - \hat x, \qquad \hat\varphi_1(\hat x) = \Lgr{1}{1}(\hat x) = \hat x
\end{equation*}
\item $\hat\Sigma$ is the set of linear forms evaluating the function at Lagrange nodes $\xi_0 = 0$ and $\xi_1 = 1$, $\Fam{\hat\sigma_0:v\mapsto v(\xi_0), \hat\sigma_1:v\mapsto v(\xi_1)}$.
\end{itemize}
\end{dfntn}

Consequently the local interpolation operator is
\begin{equation*}\label{fem:lagrange_local_interpolator}
\begin{array}{lclcl}
\IopK{\xV} &:& \xV(K) &\rightarrow& \xVh(K)\\[2ex]
           & & v   &\mapsto    & \displaystyle v(\xi_0) \varphi_0 + v(\xi_1) \varphi_1
\end{array}
\end{equation*}

\begin{prpstn}
$\FE$ by Definition \ref{fem:lagrange_p1_1d} is a unisolvent $\xHone$-conformal Finite Element.
\end{prpstn}
\begin{proof}
During the lecture we proved that:
\begin{itemize}
\item $\xVh\subset\xHone(\dom)$, since piecewise linear and piecewise constant functions belong to $\xLtwo(\dom)$.
\item $(\varphi_i)$ is a basis of $\xPone(K)$ since shape functions are linearly independent as they satisfy $\varphi_i(\xi_j) = \delta_{ij}$, and they generate the space $\xVh$ as any piecewise linear function coincides with its interpolate.
\item $(\sigma_i)$ is a dual basis of $\xPone(K)$ since $\sigma_i(\varphi_j) = \delta_{ij}$.
\end{itemize}
\end{proof}

\subsection{Lagrange $\xP_k$ elements}

The extension to Lagrange $\xP_k$ is natural as it boils down to construct a basis of $\xP_k$ using $(\Lgr{k}{i})_{0\leq i \leq k}$ and choose the degrees of freedom at the corresponding Lagrange nodes $\Fam{\xi_i}_{0\leq i \leq k}$.
Given that the unisolvence for Lagrange $\xPone$ is a direct consequence of the construction from a nodal basis, the same argument applies for higher polynomial order as long as degrees of freedom are located at Lagrange nodes.
As illustration Table \ref{tab:lagrange_1d} depicts the shape functions for $k=1,2,3$ in one dimension with equidistributed nodes (but other distributions are possible and also other polynomials can be used to build nodal elements).

\clearpage
\begin{table}[hbt!]\label{tab:lagrange_1d}
\centering
\begin{tabular}{|cc|}
\hline
$\xPone$
&
\begin{tikzpicture}
  \begin{axis}[
    xmin=0.,
    xmax=1.,
    ymin=-0.4,
    ymax=1.2,
    axis x line*=middle,
    axis y line*=middle,
    xtick={0.0,1.0},
    xticklabels={$0$, $1$},
    /pgfplots/ytick={0., 1.},
    width=\axisdefaultwidth,
    height=\axisdefaultheight
    ]
    \addplot[black, domain=0:1]{1.-x} node[above] at (10,140) {$\hat{\varphi}_{0}$};
    \addplot[black, domain=0:1]{x} node[above] at (90,140) {$\hat{\varphi}_{1}$};
    \addplot[black, dotted, thin, domain=0:1, samples=2]{1};
    \addplot[black, dotted, thin] coordinates { (1,0) (1,1)};
  \end{axis}
\end{tikzpicture}
\\ \hline
$\xPtwo$
&
\begin{tikzpicture}
  \begin{axis}[
    xmin=0.,
    xmax=1.,
    ymin=-0.4,
    ymax=1.2,
    axis x line*=middle,
    axis y line*=middle,
    xtick={0.0,1.0},
    xticklabels={$0$, $1$},
    /pgfplots/ytick={0., 1.},
    width=\axisdefaultwidth,
    height=\axisdefaultheight
    ]
    \addplot[black, domain=0:1]{2*(x-1)*(x-0.5)} node[above] at (10,140) {$\hat{\varphi}_{0}$};
    \addplot[black, domain=0:1]{4*x*(1-x)} node[above] at (50,140) {$\hat{\varphi}_{1}$};
    \addplot[black, domain=0:1]{2*x*(x-0.5} node[above] at (90,140) {$\hat{\varphi}_{2}$};
    \addplot[black, dotted, thin, domain=0:1, samples=2]{1};
    \addplot[black, dotted, thin] coordinates { (1,0) (1,1)};
    \addplot[darkblue, dotted, thin] coordinates { (0.5,0) (0.5,1)};
  \end{axis}
\end{tikzpicture}
\\ \hline
$\xP_3$
&
\begin{tikzpicture}
  \begin{axis}[
    xmin=0.,
    xmax=1.,
    ymin=-0.4,
    ymax=1.2,
    axis x line*=middle,
    axis y line*=middle,
    xtick={0.0,1.0},
    xticklabels={$0$, $1$},
    /pgfplots/ytick={0., 1.},
    width=\axisdefaultwidth,
    height=\axisdefaultheight
    ]
    \addplot[black, domain=0:1, samples=100]{-4.5*(x-1)*(x-2/3)*(x-1/3))} node[above] at (10,140) {$\hat{\varphi}_{0}$};
    \addplot[black, domain=0:1, samples=100]{+13.5*(x-1)*(x-2/3)*x} node[above] at (40,140) {$\hat{\varphi}_{1}$};
    \addplot[black, domain=0:1, samples=100]{-13.5*(x-1)*(x-1/3)*x} node[above] at (60,140) {$\hat{\varphi}_{2}$};
    \addplot[black, domain=0:1, samples=100]{+4.5*(x-2/3)*(x-1/3)*x} node[above] at (90,140) {$\hat{\varphi}_{3}$};
    \addplot[black, dotted, thin, domain=0:1, samples=2]{1};
    \addplot[black, dotted, thin] coordinates { (1,0) (1,1)};
    \addplot[darkblue, dotted, thin] coordinates { (1/3,0) (1/3,1)};
    \addplot[darkblue, dotted, thin] coordinates { (2/3,0) (2/3,1)};
  \end{axis}
\end{tikzpicture}
\\ \hline
\end{tabular}
\caption{Shape functions for Lagrange $\xPone,\xPtwo,\xP_3$ on the interval $\hat{K} = [0,1]$.}
\end{table}

%-------------------------------------------------------------------------------
\section{Extension to multiple dimensions}

\subsection{Barycentric coordinates}

Lagrange polynomials \eqref{def:lagrange_poly} construct directly one-dimensional shape functions, while in higher dimensions they can be reformulated in terms of barycentric coordinates.
High-order Lagrange basis can also be expressed as polyomials of barycentric coordinates.

\begin{dfntn}[Barycentric coordinates]
Let us consider $K$ a $d$-simplex with vertices $\Fam{\vertex_i}_{0\leq i \leq d}$, any point $\xx \in K$ satisfies
\begin{equation*}
\xx = \sum_{i} \lambda_i(\xx) \vertex_i
\end{equation*}
where barycentric coordinates are obtained by relation
\begin{eqnarray*}
\lambda_i:& \xR^d &\fromto \xR\\
          & \bfx  &\mapsto \lambda_i(\bfx) = 1 - \dfrac{(\bfx - \vertex_i)\cdot\n_i}{(\vertex_{f} - \vertex_i)\cdot\n_i}
\end{eqnarray*}
with $\n_i$ the unit outward normal to the facet opposite to $\vertex_i$, and $\vertex_f$ a vertex belonging to this facet.
\end{dfntn}

The geometric interpretation of barycentric coordinates is given by
\[
\displaystyle\lambda_i = \frac{\mathrm{meas}(K_i)}{|K|}
\]
with $\mathrm{meas}(K_i)$ the signed measure of $K_i(\xx)$ the $d$-simplex constructed with point $\xx$ and the facet opposite to $\vertex_i$.
In particular, points located within $K_i$ have non-negative $\lambda_i$, and the point satisfying equal weight $\lambda_i = (d+1)^{-1}$ is the isobarycentre.
In practice this property can be used to check if a point is inside a simplex: if the signed measure of one $\lambda_i$ is negative or if $\sum_i |K_i(\xx)| > |K|$ then the point is outside of the simplex.

\begin{xmpl}[Lagrange $\xPone$ 1D]
In one dimension of space, barycentric coordinates on $K = [x_0, x_1]$ are
\begin{equation*}
\left\lbrace
\begin{array}{lcl}
\lambda_0(x) &=& 1 - \displaystyle\frac{x - x_0}{x_1 - x_0} = \frac{x_1 - x}{x_1 - x_0}\\[2ex]
\lambda_1(x) &=& 1 - \displaystyle\frac{x - x_1}{x_0 - x_1} = \frac{x - x_0}{x_1 - x_0}\\
\end{array}
\right.
\end{equation*}
which is exactly the same expression as linear shape functions $\varphi_0$ and $\varphi_1$.
\end{xmpl}

\begin{xmpl}[Lagrange $\xPone$ 2D]
In two dimensions of space, barycentric coordinates on the unit triangle $\hat K$ depicted Figure \ref{fig:barycentric_coords_2d} are
\begin{equation*}
\left\lbrace
\begin{array}{lclcl}
\lambda_0(\xx) &=& 1 - \displaystyle\frac{(\hat x,\hat y)\cdot(1,1)}{(1,0)\cdot(1,1)} &=& 1 - \hat x -\hat y\\[2ex]
\lambda_1(\xx) &=& 1 - \displaystyle\frac{(\hat x-1,\hat y)\cdot(-1,0)}{(-1,0)\cdot(-1,0)} &=& \hat x\\[2ex]
\lambda_2(\xx) &=& 1 - \displaystyle\frac{(\hat x,\hat y-1)\cdot(0,-1)}{(0,-1)\cdot(0,-1)} &=& \hat y\\
\end{array}
\right.
\end{equation*}
which is the linear Lagrange basis on $\hat K$ (since the normal is at the numerator and the denominator, there was no need to normalize the vector).

\begin{figure}[h!]\label{fig:barycentric_coords_2d}
\centering
\begin{tikzpicture}[scale=2.0]
\coordinate (hv0) at (0,0);
\coordinate (hv1) at (1,0);
\coordinate (hv2) at (0,1);
\fill[black] (hv0) circle [radius=1pt];
\fill[black] (hv1) circle [radius=1pt];
\fill[black] (hv2) circle [radius=1pt];
\draw (hv0) node [below] {$\hat\vertex_0 = (0,0)$}
      (hv1) node [below] {$\hat\vertex_1 = (1,0)$}
      (hv2) node [above] {$\hat\vertex_2 = (0,1)$}
      node at (0.33,0.33) {$\hat K$};
\draw [-,thin] (hv0)  --  (hv1)   --  (hv2) --  (hv0);
\draw [-latex, darkblue] (0.5,0.5)  --  (1.207,1.207) node [right] {$\n_0$};
\draw [-latex, darkblue] (0.0,0.5)  --  (-1.,0.5) node [left] {$\n_1$};
\draw [-latex, darkblue] (0.5,0.0)  --  (0.5,-1.) node [below] {$\n_2$};
\end{tikzpicture}
\caption{Unit triangle with outward facet normals}
\end{figure}

We can verify easily that shape functions
\begin{equation*}
\left\lbrace
\begin{array}{lclcl}
\hat\varphi_0(\xx) &=& 1 - \hat x -\hat y\\[2ex]
\hat\varphi_1(\xx) &=& \hat x\\[2ex]
\hat\varphi_2(\xx) &=& \hat y\\
\end{array}
\right.
\end{equation*}
form a nodal basis and that $\hat\varphi_0(\xx) + \hat\varphi_1(\xx) + \hat\varphi_2(\xx) = 1$ for any $\hat\xx$.

\end{xmpl}

Lagrange elements of polynomial degree $k=1,2,3$ can be expressed using barycentric coordinates in higher dimensions, the shape functions are given by:
\begin{equation*}
\begin{array}{lll}
k = 1, & \displaystyle\lambda_i &,\;0 \leq i \leq d\\[4ex]
k = 2, & \displaystyle\lambda_i\;(2 \lambda_i - 1) & ,\;0 \leq i \leq d\\[2ex]
       & \displaystyle4\;\lambda_i\;\lambda_j       & ,\;0 \leq i < j \leq d\\[4ex]
k = 3, & \displaystyle\frac{1}{2}\;\lambda_i\;(3\lambda_i - 1)\;(3 \lambda_i - 2) & ,\;0 \leq i \leq d\\[2ex]
       & \displaystyle\frac{9}{2}\;\lambda_i\;(3\lambda_i - 1)\;\lambda_j & ,\;0 \leq i,j \leq d, i \neq j\\[2ex]
       & \displaystyle27\;\lambda_i\;\lambda_j\;\lambda_k & ,\;0 \leq i < j < k \leq d\\
\end{array}
\end{equation*}

\subsection{Affine transformation}

In Chapter \ref{sec:fem} the one-dimensional affine mapping between the unit interval and any subinterval $K = [x_{i}, x_{i+1}]$ of a one-dimensional mesh $\meshT$ was given by Equation \eqref{eq:affine_mapping_1d}; the link to higher dimensions in space was then briefly discussed.
The following example describes how the affine mapping $T_K : \hat K \rightarrow K$ is defined for a triangle in $\xR^2$.

\begin{figure}[h!]\label{fig:affine_mapping_2d}
\centering
\begin{tikzpicture}[scale=2.0]
\coordinate (hv0) at (0,0);
\coordinate (hv1) at (1,0);
\coordinate (hv2) at (0,1);
\fill[black] (hv0) circle [radius=1pt];
\fill[black] (hv1) circle [radius=1pt];
\fill[black] (hv2) circle [radius=1pt];
\draw (hv0) node [below] {$\hat\vertex_0 = (0,0)$}
      (hv1) node [below] {$\hat\vertex_1 = (1,0)$}
      (hv2) node [above] {$\hat\vertex_2 = (0,1)$}
      node at (0.33,0.33) {$\hat K$};
\draw [-,thin] (hv0)  --  (hv1)   --  (hv2) --  (hv0);
\coordinate (v0) at (3,0.6);
\coordinate (v1) at (4,0.1);
\coordinate (v2) at (4.5,1.4);
\fill[black] (v0) circle [radius=1pt];
\fill[black] (v1) circle [radius=1pt];
\fill[black] (v2) circle [radius=1pt];
\draw (v0) node [below] {$\vertex_0$}
      (v1) node [below] {$\vertex_1$}
      (v2) node [above] {$\vertex_2$}
      node at (3.83,0.7) {$K$};
\draw [-,thin] (v0)  --  (v1)   --  (v2) --  (v0);
\draw[->, darkblue, thick] (0.43,0.33) to [out=30,in=165] (3.73,0.7);
\draw (1.87,0.9) node[black,above] {$T_K$};
\end{tikzpicture}
\end{figure}

The shape of the reference triangle $\hat K$ is defined by vectors $\hat\vertex_1 - \hat\vertex_0$ and $\hat\vertex_2 - \hat\vertex_0$, and in the same fashion the shape of any triangle $K$ is defined by vectors $\vertex_1 - \vertex_0$ and $\vertex_2 - \vertex_0$.
The affine mapping is a simple change of coordinates but the detail is given below for the sake of completness.
\begin{equation*}
\left\lbrace
\begin{array}{lclcl}
\vertex_1 &=& T_K(\hat\vertex_1) &=& \vertex_0 + (\vertex_1 - \vertex_0)\\
\vertex_2 &=& T_K(\hat\vertex_2) &=& \vertex_0 + (\vertex_2 - \vertex_0)\\
\end{array}
\right.
\end{equation*}
and any point $\xx$ of $K$ can be expressed in terms of the relation
\begin{equation*}
\xx = \vertex_0 + \lambda (\vertex_1 - \vertex_0) + \mu (\vertex_2 - \vertex_0)
\end{equation*}
with given $\lambda$ and $\mu$.
The reference triangle is defined by the canonical basis of $\xR^2$ as $(\hat\vertex_1 - \hat\vertex_0, \hat\vertex_1 - \hat\vertex_0) = (\bfe_x, \bfe_y)$ so that the affine mapping
\begin{equation*}
\xx = \vertex_0 + \matB_K \hat\xx
\end{equation*}
satisfies $T_K(\bfe_x) = (\vertex_1 - \vertex_0)$ and $T_K(\bfe_y) = (\vertex_2 - \vertex_0)$.
The matrix $\matB_K$ is then the matrix of the corresponding change of basis composed of column vectors $\vertex_j - \vertex_0$, thus
\begin{equation*}
\xx = \vertex_0 +
\begin{bmatrix}
\vertex_{1,x} - \vertex_{0,x} & \vertex_{2,x} - \vertex_{0,x}\\
\vertex_{1,y} - \vertex_{0,y} & \vertex_{2,y} - \vertex_{0,y}\\
\end{bmatrix}
\hat\xx
\end{equation*}

\medskip
\begin{dfntn}[Affine mapping from reference simplex in $\xR^d$]
The generalization of the affine mapping in $\xR^d$ from the reference simplex $\hat K = \Fam{\hat\vertex_i}_{0\leq i\leq d}$ to $K = \Fam{\vertex_i}_{0\leq i\leq d}$ is given by $\xx = \vertex_0 + \matJ_{T_K} \hat\xx$,
with
\begin{equation*}
\matJ_{T_K} = \left[\frac{\partial T^i_{K}}{\partial x_j}  \right]_{ij}
\end{equation*}
given by colum vectors $(\vertex_j - \vertex_0)$.
\end{dfntn}

While the change of coordinates for the mass matrix does not pose any difficulty, the case of the stiffness matrix requires some precisions.
The derivation of the composition of two functions reads
\begin{equation*}
\Grad{\varphi} = \Grad{(\hat\varphi\circ T_K^{-1})} = (\Grad{\hat\varphi\circ T_K^{-1}})\cdot\matJ_{T_K^{-1}}
\end{equation*}
which can be also written formally component by component
\begin{equation*}
\frac{\partial\varphi}{\partial x_i}
= \sum_{j} \displaystyle\frac{\partial\varphi}{\partial \hat x_j} \frac{\partial \hat x_j}{\partial x_i}
= \sum_{j} \frac{\partial\varphi}{\partial \hat x_j} \matJ_{T_K^{-1}}
\end{equation*}
and can be interpreted as the decomposition of variation $\md \xx$ along each axis in terms of $\md \hat\xx$.
Moreover the Jacobian matrix of the inverse mapping is the inverse of the Jacobian matrix
\begin{equation*}
\matJ_{T_K^{-1}} = (\matJ_{T_K}\circ T_K^{-1})^{-1}
\end{equation*}
so that
\begin{equation*}
\Grad{\varphi} = [(\matJ_{T_K}\circ T_K^{-1})^{-1}]^\trans (\Grad{\hat\varphi\circ T_K^{-1}})
\end{equation*}
and since the Jacobian matrix is constant on each cell $K$, it can be simplified as
\begin{equation*}
\Grad{\varphi} = [\matJ_{T_K}^{-1}]^\trans (\Grad{\hat\varphi\circ T_K^{-1}})
\end{equation*}
%-------------------------------------------------------------------------------
\section{Local equation for Lagrange $\xPone$ in one dimension}\label{sec:local_equation_poisson_p1}

The approximation of Problem \eqref{pb:weak_poissonHone} by Lagrange $\xP_1$ elements on domain $\dom = (0,1)$ reads:
\begin{subequations}\label{pb:poisson_p1}
\begin{equation}
\left\lvert
\begin{array}{ll}
\mbox{Find $u \in \xVh$, given $f \in \xLtwo(\dom)$, such that:}\\[2ex]
\displaystyle\int_\dom \Grad u\xDot \Grad v\dx = \int_\dom f v  \dx\quad,\;\forany  v\in \xVh
\end{array}
\right .
\end{equation}
with the approximation space $\xVh$ chosen as:
\begin{equation}
\xVh = \Set{ v \in \xCzero(\bar\dom)\cap\xHonec(\dom) : v|_K \in\xP_1(K),\forany K \in\meshT }
\end{equation}
\end{subequations}

The interval $\bar\dom = [0,1]$ is discretized by partitioning into disjoints subintervals $[x_{n}, x_{n+1}]$, $1\leq n \leq N_K$ of length $h = 1 / N_K$.
Steps to obtain a weak formulation and deriving a discrete problem were detailed in Section \ref{sec:poisson_lagrange_p1}.

Expressing the local equation for any subinterval $K = [x_{n}, x_{n+1}]$ consists of assembling a matrix corresponding to contributions
\begin{equation*}
\matA_{ij} = \int_{K} \partial_{x}\varphi_j(x)\;\partial_{x}\varphi_i(x) \md x
\end{equation*}
for shape functions $\varphi_j$ and $\varphi_i$ which have support on $K$.
Given that the dimension of the Lagrange $\xPone$ element in one dimension is two, with two shape functions $\varphi_n$ and $\varphi_{n+1}$, the local matrix is of dimension $2\times 2$.
The derivative of $\varphi_n$ and $\varphi_{n+1}$ is constant on $K$ and of opposite signs: $\varphi_n(x) = -1$ and $\varphi_{n+1}(x) = +1$.
\begin{equation*}
\matA_{K} = \frac{1}{h}
\begin{bmatrix}
+1 & -1 \\
-1 & +1 \\
\end{bmatrix}
\end{equation*}
with row and column indices of the local matrix mapping to row and column indices $(n , n+1)$ of the global matrix.
Therefore assembling the local equation into the global matrix consists of adding entries of $A_K$ to the submatrix with row and column indices $(n , n+1)$.
Since each node $x_n$ has two adjacent subintervals $[x_{n-1}, x_{n}]$ and $[x_{n}, x_{n+1}]$, inner nodes (which are not on the boundary) will see two contributions $+1$ on the diagonal, one contribution $-1$ for columns $n-1$, and one contribution $-1$ for columns $n+1$, scaled by factor $1/h$.
\begin{equation*}
\matA_{i} = \frac{1}{h}
\begin{bmatrix}
0 & \cdots & 0 & -1 & +2 & -1 & 0 & \cdots & 0
\end{bmatrix}
\end{equation*}
If the partition of the interval is not uniform then the assembly should be modified,
\begin{equation*}
\matA_{K} = \frac{1}{|K|}
\begin{bmatrix}
+1 & -1 \\
-1 & +1 \\
\end{bmatrix}
\end{equation*}
with $|K| = |x_{n+1} - x_{n}|$.
\newpage

\section{Exercises}
