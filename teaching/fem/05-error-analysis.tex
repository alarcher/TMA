
\chapter{Error analysis}\label{sec:error_analysis}

The goal of this section is to bound the approximation error $e_h = u - u_h$ in a Sobolev or Lebesgue norm.
To this purpose we have already two ingredients:

--- on the one hand, in the analysis of Ritz and Galerkin methods, consistency estimates like Cea's Lemma give a control on the approximation error in the solution space $\xV$ in term of ``distance'' between the solution space and the approximation space:
\begin{equation*}
\norm{u - u_h}_\xV  \leq C \norm{u - v_h}_\xV\quad,\; \forany v_h \in \xVh
\end{equation*}
with $C > 0$ a constant real number,

--- on the other hand, the pointwise interpolation inequality of Theorem \eqref{th:polyinterpol} gives a control on the interpolation error $e_{\pi} = u - \Projh{k} u$, \ie the difference between any function and its interpolate by Lagrange polynomials of order $k$.

\medskip
For consistency with the notation introduced in Chapter \ref{sec:fem} for the Lagrange interpolation operator in the Finite Element setting, $\Iop{h}^k$ will stand for the Lagange $\xP_k$ interpolation operator in this section.

\medskip
\Question Can we control the approximation error by bounding the right-hand side of the consistency inequality using interpolation properties?

%-------------------------------------------------------------------------------
\section{Preliminary discussion on the Poisson problem}

In the previous chapters, to the weak formulation derived for the Poisson problem with homogeneous Dirichlet boundary conditions and $f\in\xLtwo(\dom)$,
\begin{equation*}
\left\lvert
\begin{array}{ll}
\mbox{Find $u\in \xV = \xHonec(\dom)$ such that:}\\[2ex]
\displaystyle\int_\dom \Grad{u}\xDot\Grad{v} \dx = \int_\dom f\;v \dx\;,\quad\forany v \in\xV\\[2ex]
\end{array}
\right.
\end{equation*}
we associated a Galerkin problem,
\begin{equation*}
\left\lvert
\begin{array}{ll}
\mbox{Find $u_h\in\xVh\subset\xHonec(\dom)$ such that:}\\[2ex]
\displaystyle\int_\dom \Grad{u_h}\xDot\Grad{v} \dx = \int_\dom f\;v \dx\;,\quad\forany v \in\xVh\\[2ex]
\end{array}
\right.
\end{equation*}
by simply looking for a finite dimensional solution.

\medskip
In a first stage, we verified that the problem is well-posed for the chosen solution and test spaces, and that the exact solution $u$ and discrete solution $u_h$ satisfy both a relation of the type
\begin{equation*}
\snorm{u}_{\xHone(\dom)} \leq C \norm{f}_{\xLtwo(\dom)}
\end{equation*}
with $C > 0$, which we denoted as \textit{a priori} estimate.

\medskip
In a second stage, without making any further assumption on the approximation space $\xVh$, a general result was introduced showing that the distance between the exact solution and the approximate solution $\norm{u - u_h}_{\xV}$ is bounded by the distance of $u$ to the best approximation of $u$ in $\xVh$; this is Cea's Lemma.
For the sake of completeness, the estimate corresponding to Cea's Lemma is given in the case of the Poisson problem with homogeneous Dirichlet boundary conditions.
\begin{equation*}
E = \snorm{u - u_h}^2_{\xHone(\dom)} = \int_\dom \Abs{\Grad(u - u_h)}^2\md x = \int_\dom \Grad(u - u_h)\xDot\Grad(u - u_h) \md x
\end{equation*}

For any $v_h\in\xVh$
\begin{equation*}
E = \int_\dom \Grad(u - u_h)\xDot\Grad(u - v_h) \md x + \int_\dom \Grad(u - u_h)\xDot\Grad(v_h - u_h) \md x
\end{equation*}
and the second term cancels by virtue of consistency (Galerkin orthogonality), so that Cauchy--Scwarz gives
\begin{equation*}
E = \snorm{u - u_h}^2_{\xHone(\dom)} \leq \left(\int_\dom \Abs{\Grad(u - u_h)}^2 \md x\right)^{\sfrac{1}{2}} \left(\int_\dom \Abs{\Grad(u - v_h)}^2 \md x\right)^{\sfrac{1}{2}}
\end{equation*}
thus
\begin{equation*}
\snorm{u - u_h}_{\xHone(\dom)} \leq \snorm{u - v_h}_{\xHone(\dom)}
\end{equation*}


As introduced, the relation
\begin{equation}\label{eq:error_estimate_h1}
\snorm{e_h}_{\xHone(\dom)} \leq \snorm{u - v_h}_{\xHone(\dom)}
\end{equation}
means that the \textit{approximation error} $e_h = u - u_h$ is controlled as soon as we are able to bound the distance between $u$ and its best representant in $\xVh$.

\medskip
In a third stage, we introduced Finite Element approximation spaces based on Lagrange polynomials, Definition \ref{def:lagrange_poly}.
Their local interpolation operator given here in one dimension,
\begin{equation*}
\LgrK{k} v = \sum_{j=0}^k v(\xi_j) \Lgr{k}{j}
\end{equation*}
satisfies Inequality \eqref{th:polyinterpol} controlling the \textit{interpolation error} $e_\imath = v - \LgrK{k} v$ pointwise on $K$.

\medskip
Inequality \eqref{eq:error_estimate_h1} can be written
\begin{equation}\label{eq:error_estimate_h1_2}
\snorm{e_h}_{\xHone(\dom)} \leq \snorm{e_\imath}_{\xHone(\dom)}
\end{equation}
which means that the \textit{approximation error} is bounded by the \textit{interpolation error}.
From pointwise polynomial interpolation estimates we need to derive inequalities in Sobolev and Lebesgue norms, so that the right-hand side of Equation \eqref{eq:error_estimate_h1_2} can be bounded by a function $\epsilon$ of the mesh size $\sizeT$, such that $\epsilon(\sizeT) \tendsto 0$ as $\sizeT \tendsto 0$; convergence of the approximation is then ensured.

\medskip
Inequalities of the form
\begin{equation*}
\norm{e_h} \leq C \epsilon(\sizeT)
\end{equation*}
for some constant $C$ depending on the domain and the exact solution, are called \textit{a priori error estimates}.
Usually $\epsilon(\sizeT) \sim O(\sizeT^r)$, with $r > 0$ the convergence rate of the method, which represents the \textit{accuracy}.
In this case sequences of approximate solution $(u_h)_{\sizeT}$ converge to the exact solution $u$, as fast as the function $\epsilon$ allows, when $\sizeT$ tends to zero.

\medskip
The goal of this section is to derive such error estimates after proving required interpolation inequalities.
In particular the space of continuous linear Lagrange elements
\begin{equation}\label{sp:lagrangeP1}
\xVh = \lbrace v \in \xC^0(\bar\dom)\cap\xHonec(\dom) : \forany K \in \meshT, \Restrict{v}{K} \in \xPone(K) \rbrace
\end{equation}
will be considered, then more general results will be stated without proof.

%-------------------------------------------------------------------------------
\section{Stability of the Lagrange interpolation operator}

Before moving to interpolation inequalities, the \textit{stability} of the interpolation operator should be proved: such estimate shows that any interpolate of a function in $\xV$ is also in $\xV$.
The exposé is restricted to the one-dimensional case and detailed so that anybody without prior experience with estimates should be able to follow the procedure.

\medskip
Interpolation properties need to be reformulated in terms of estimates in $\xLtwo$ norms.
First they are expressed elementwise, then on the entire domain by collecting contributions over the mesh.
To give a better idea, let us introduce below an ingredient used for subsequent estimates, considering a function $v\in\xHone(I)$ with $I$ an interval, and two points $\xi, x \in I$.
\begin{equation*}
\begin{array}{lcl}
\Abs{v(x) - v(\xi)} &=&    \displaystyle\Abs{\int_\xi^x v'(s) \md s}\\[2ex]
                    &\leq& \displaystyle\int_\xi^x \Abs{v'(s)} \md s\\[2ex]
                    &\leq& \displaystyle|x - \xi|^{\sfrac{1}{2}}\left(\int_\xi^x \Abs{v'(s)}^{2} \md s\right)^{\sfrac{1}{2}}\\[2ex]
\end{array}
\end{equation*}
using Cauchy--Schwarz with $\Abs{v'(s)}$ and $\indic{[\xi, x]}$, the indicator function on $[\xi, x]$.
The right-hand side is bounded since we assumed that $v\in\xHone(I)$), so that
\begin{equation}\label{eq:interpolation_estimate_1}
\Abs{v(x) - v(\xi)} \leq \displaystyle|x - \xi|^{\sfrac{1}{2}}\snorm{v}_{\xHone([\xi, x])}
\end{equation}
gives a bound involving the $\xHone$ semi-norm.

\medskip
Now let us consider that $\xi$ realizes the minimum of $|v|$ on $I$, then
\begin{equation*}
\Abs{v(x)} \leq |x - \xi|^{\sfrac{1}{2}}\snorm{v}_{\xHone([\xi, x])} + \Abs{v(\xi)},\quad \xi = \argmin{s\in I} |v(s)|
\end{equation*}
using the second triangle inequality.

\medskip
Immediately if $|v(\xi)| = 0$ the estimate gives for any $x\in I$
\begin{equation}\label{eq:interpolation_estimate_2}
\Abs{v(x)} \leq |I|^{\sfrac{1}{2}}\snorm{v}_{\xHone(I)}
\end{equation}
but otherwise we can bound $|v(\xi)|$ using
\begin{equation*}
\Abs{v(\xi)} = \Abs{x-\xi}^{-1} \int_\xi^x \Abs{v(s)} \md s \leq |x - \xi|^{-\sfrac{1}{2}}\norm{v}_{\xLtwo([\xi, x])}
\end{equation*}
so that
\begin{equation*}
\Abs{v(x)} \leq |x - \xi|^{\sfrac{1}{2}}\snorm{v}_{\xHone([\xi, x])} + |x - \xi|^{-\sfrac{1}{2}}\norm{v}_{\xLtwo([\xi, x])}
\end{equation*}
thus for any $x\in I$
\begin{equation}\label{eq:interpolation_estimate_2}
\Abs{v(x)} \leq |I|^{\sfrac{1}{2}}\snorm{v}_{\xHone(I)} + |I|^{-\sfrac{1}{2}}\norm{v}_{\xLtwo(I)}
\end{equation}
in other words
\begin{equation}\label{eq:interpolation_estimate_3}
\norm{v}_{\xLinfty(I)} \leq |I|^{\sfrac{1}{2}}\snorm{v}_{\xHone(I)} + |I|^{-\sfrac{1}{2}}\norm{v}_{\xLtwo(I)}
\end{equation}

Inequality \eqref{eq:stability_lagrange_p1} is also called \textit{uniform continuity} in zero of the interpolation operator since $C$ should not depend on $\meshT$; but possibly depends on $\dom$.

\begin{prpstn}[$\xHone$-stability of the Lagrange $\xPone$ interpolation operator]
\label{prpstn:stability_lagrange_p1}
There exists a constant $C>0$ such that,
\begin{equation}\label{eq:stability_lagrange_p1}
\norm{\Lgrh{1} v}_{\xHone(\dom)} \leq C\; \norm{v}_{\xHone(\dom)}
\end{equation}
\end{prpstn}
\begin{proof}
Since $\norm{\;\cdot\;}^2_{\xHone} = \norm{\;\cdot\;}^2_{\xLtwo} + \snorm{\;\cdot\;}^2_{\xHone}$ then the $\xLtwo$ norm of the interpolate and its derivative should be controlled. Since controlling the derivative of a function in $\xLtwo$ gives a control on the function in $\xLtwo$ (Poincaré Inequality), it is natural to start looking for an estimate of $\snorm{\;\cdot\;}_{\xHone}$, then move to $\norm{\;\cdot\;}_{\xLtwo}$.
\begin{tmaproofitems}
\item Estimate in $\snorm{\;\cdot\;}_{\xHone}$:\\
Let us consider the restriction of the interpolation operator to any $K = [x_{i}, x_{i+1}] \in \meshT$
\[
\Restrict{\bigl(\Lgrh{1}v\bigr)'}{K} = \bigl(\LgrK{1}v\bigr)' = h_K^{-1}\bigl(v(x_{i+1})- v(x_{i}))
\]
which is constant over $K$.
The elementwise $\xHone$ semi-norm of the interpolate can be bounded using Inequality \eqref{eq:interpolation_estimate_1}
\begin{equation*}
\begin{array}{lcl}
\snorm{\LgrK{1}v}_{\xHone(K)} &=&\displaystyle \left(\int_K h_K^{-2}\Abs{v(x_{i+1})- v(x_{i})}^2 \md x\right)^{\sfrac{1}{2}}\\[2ex]
                              &\leq& h_K^{-\sfrac{1}{2}}\Abs{v(x_{i+1})- v(x_{i})}\\[2ex]
                              &\leq& h_K^{-\sfrac{1}{2}} h_K^{\sfrac{1}{2}}\snorm{v}_{\xHone(K)}\\[2ex]
                              &\leq& \snorm{v}_{\xHone(K)}\\
\end{array}
\end{equation*}
Summing over $K$ and using the definition of $\sizeT$,
\begin{equation*}
\snorm{\Lgrh{1}v}_{\xHone(\dom)} \leq \snorm{v}_{\xHone(\dom)}
\end{equation*}
\item Estimate in $\norm{\;\cdot\;}_{\xLtwo}$:\\
In this case we do not need to prove an elementwise estimate first as we already introduced a control of pointwise values in terms of the $\xHone$ seminorm and the $\xLtwo$ norm: it boils down to estimate the maximum attained by the linear interpolate.
\begin{equation*}
\begin{array}{lcl}
\norm{\Lgrh{1}v}_{\xLtwo(\dom)} &=&\displaystyle \left(\int_\dom \Abs{\Lgrh{1}v}^2 \md x\right)^{\sfrac{1}{2}}\\[2ex]
                              &\leq&\displaystyle \left(\int_\dom \norm{\Lgrh{1}v}_{\xLinfty(\dom)}^2 \md x\right)^{\sfrac{1}{2}}\\[2ex]
                              &\leq& |\dom|^{\sfrac{1}{2}} \norm{\Lgrh{1}v}_{\xLinfty(\dom)}\\[2ex]
                              &\leq& |\dom|^{\sfrac{1}{2}} \norm{v}_{\xLinfty(\dom)}\\
\end{array}
\end{equation*}
the last line is given by the $\xLinfty$ stability of the linear interpolation: the function and its linear interpolate coincide at Lagrange nodes so that $\norm{\Lgrh{1}v}_{\xLinfty(\dom)} \leq \norm{v}_{\xLinfty(\dom)}$.
Therefore, using Inequality \eqref{eq:interpolation_estimate_3}
\begin{equation*}
\norm{\Lgrh{1}v}_{\xLtwo(\dom)}  \leq |\dom|^{\sfrac{1}{2}}\snorm{v}_{\xHone(\dom)} + |\dom|^{-\sfrac{1}{2}}\norm{v}_{\xLtwo(\dom)}
\end{equation*}
we can conclude using the Poincaré Inequality with constant $c_p$,
\begin{equation*}
\norm{\Lgrh{1}v}_{\xLtwo(\dom)}  \leq (|\dom| + c_p^{-1})\snorm{v}_{\xHone(\dom)}
\end{equation*}
\end{tmaproofitems}
\end{proof}

%-------------------------------------------------------------------------------
\section{\textit{A priori} error estimate with Lagrange $\xPone$}

\begin{thrm}[Interpolation Inequality in $\xHonec(\dom)$ and $\xLtwo(\dom)$]
There exists two positive constants $C_1$ and $C_0$ such that for any $v\in\xHtwo(\dom)$
\begin{equation}
\snHoneD{v - \Lgrh{1}v}  \leq C_1 \sizeT\;\snHtwoD{v}
\end{equation}
\begin{equation}
\nLtwo{v - \Lgrh{1}v}  \leq C_0 \sizeT^2 \snHtwoD{v}
\end{equation}
with $\sizeT = \max_{K\in\meshT}(h_K)$
\end{thrm}
\begin{proof}
The proof is based on a decomposition of the error per element and on the Mean-Value Theorem (also known as Rolle Theorem).
The global interpolation error is then recovered by summing over the cells.
This makes sense since the polynomial interpolation estimate is defined pointwise, it is then a local property.
In the same spirit as the stability of the interpolation operator which we proved first elementwise for Proposition \ref{prpstn:stability_lagrange_p1}.

\medskip
The main technical ingredient is given by Inequality \eqref{eq:interpolation_estimate_1} for a given function $w\in\xHone(\dom)$,
\begin{equation}
\Abs{w(x) - w(\xi)} \leq \displaystyle|x - \xi|^{\sfrac{1}{2}}\snorm{w}_{\xHone([\xi, x])}
\end{equation}
with $\xi, x\in\dom$.
If we choose $\xi$ such that $w(\xi)$ cancels and if we integrate the square of the expression over any $K\in\meshT$, then we get a control of $w$ in $\xLtwo(\dom)$,
\begin{equation}
\Abs{w(x) - w(\xi)} \leq \displaystyle|x - \xi|^{\sfrac{1}{2}}\snorm{w}_{\xHone([\xi, x])}
\end{equation}

\medskip
\begin{tmaproofitems}
\item Estimate in $\snorm{\;\cdot\;}_{\xHone}$:\\
First, an elementwise estimate is derived with the same argument as Inequality \eqref{},
\begin{equation*}
\begin{array}{lcl}
\snorm{v - \Lgrh{1}v}^2_{\xHone(K)} &=&\displaystyle\sum_{K\in\meshT} \int_K h_K^{-2}\Abs{v(x_{i+1})- v(x_{i})}^2 \md x\\[2ex]
                              &\leq& h_K^{-\sfrac{1}{2}}\Abs{v(x_{i+1})- v(x_{i})}\\[2ex]
                              &\leq& h_K^{-\sfrac{1}{2}} h_K^{\sfrac{1}{2}}\snorm{v}_{\xHone(K)}\\[2ex]
                              &\leq& \snorm{v}_{\xHone(K)}\\
\end{array}
\end{equation*}
\item Estimate in $\norm{\;\cdot\;}_{\xLtwo}$:
\end{tmaproofitems}

\end{proof}

The approximation error being bounded in $O(\sizeT)$ the method is first order in $\xHonec(\dom)$.

\begin{rmrk}[Convergence order in $\xHone(\dom)$]
Using the definition of the norm
\begin{equation*}
\nHoneD{v - \Lgrh{1} v}^2 = \nLtwo{v - \Lgrh{1} v}^2 + \snHoneD{v - \Lgrh{1} v}^2
\end{equation*}
we get
\begin{equation*}
\nHoneD{v - \Lgrh{1} v}^2 \leq C_I^2\; (\sizeT^4 \snHtwoD{v}^2 + \sizeT^2 \snHtwoD{v}^2)
\end{equation*}
\begin{equation*}
\nHoneD{v - \Lgrh{1} v} \leq C_I\; \sizeT\;(1 + \sizeT^2)^{\sfrac{1}{2}} \snHtwoD{v}
\end{equation*}
Thus, we verify that the approximation is first order in $\xHone(\dom)$.
\end{rmrk}

%-------------------------------------------------------------------------------
\section{Superconvergence}

The following result shows the the convergence properties of the method is not only limited by the interpolation inequality.
Indeed, using a result by Aubin--Nitsche, we show that even if the approximation is not $\xHtwo$-conformal, we can improve the error estimate by one order: the convergence order in $\xLtwo(\dom)$ becomes then two.

\begin{thrm}[Superconvergence]
Let $\dom$ be a convex polygonal subset of $\xR^d$, $d \geq 1$, $f \in \xLtwo(\dom)$, $u$ solution to the Dirichlet Problem \eqref{pb:poisson} and $u_h$ approximate solution, $\sizeT\;= \max_{K\in\meshT}(h_K)$:
\begin{equation*}
\nHoneD{u - u_h}  \leq C_1\;\sizeT\quad\mbox{and}\quad\nLtwo{u - u_h}  \leq C_0\;\sizeT^2
\end{equation*}
\end{thrm}

\begin{proof}
If $u$ is solution to the Poisson problen then $u \in \xHonec(\dom)$, then by regularity Theorem \eqref{th:poisson_regularity} (by density of $\xHtwo(\dom)$ in $\xHone(\dom)$), $u \in \xHtwo(\dom)$, thus $\exists\; C_1 > 0$ such that:
\begin{equation*}
\nHtwoD{u}  \leq C_1\;\nLtwo{f}
\end{equation*}
Thus replacing the $\xHtwo$-seminorm in the right-hand side of the error estimate, we have
\begin{equation}\label{eq:h1err_eh}
\nHoneD{u - u_h}  \leq C_1\;\sizeT\;\nLtwo{f}
\end{equation}
Let us introduce the following auxiliary problem:
\begin{subequations}\label{pb:aubin_nitsche}
\begin{equation}\label{pb:aubin_nitsche_eq}
- \Lap \varphi(\x) = e_h(\x)\quad,\;x\in\dom
\end{equation}
\begin{equation}\label{pb:aubin_nitsche_bc}
\varphi(\x) = 0\quad,\;x\in\partial\dom
\end{equation}
\end{subequations}
and its weak formulation:
\begin{equation}\label{pb:weak_aubin_nitsche}
\left\lvert
\begin{array}{ll}
\mbox{Find $\varphi \in \xHonec(\dom)$, given $e_h \in \xLtwo(\dom)$, such that:}\\[2ex]
\displaystyle\int_\dom \Grad \varphi\xDot \Grad \phi\dx = \int_\dom e_h \phi  \dx\quad,\;\forany  \phi\in \xHonec(\dom)
\end{array}
\right .
\end{equation}
Since $e_h$ is bounded in $\xLtwo(\dom)$ then the same regularity result holds for the auxiliary Problem \eqref{pb:aubin_nitsche}, $\exists C_2 > 0$ such that:
\begin{equation*}
\nHtwoD{\varphi}  \leq C_2\;\nLtwo{e_h}
\end{equation*}
and thus:
estimate, we have
\begin{equation}\label{eq:h1err_phih}
\nHoneD{\varphi - \varphi_h}  \leq C_2\;\sizeT\;\nLtwo{e_h}
\end{equation}

\medskip
Let us try to bound the $\xLtwo$-norm of the approximation error by noticing that its amounts to take $\phi = e_h$ in \eqref{pb:weak_aubin_nitsche}:
\begin{equation*}
\nLtwo{e_h} = \int_\dom |e_h|^2 \dx = \int_\dom \Grad \varphi\xDot \Grad e_h\dx
\end{equation*}
If we consider the approximate of Problem \eqref{pb:weak_aubin_nitsche} by Galerkin method, with $\varphi_h \in \xVh$ its solution, then the Galerkin orthogonality reads:
\begin{equation*}
\int_\dom \Grad \varphi_h\xDot \Grad e_h\dx  = 0
\end{equation*}
Thus we can substract and add this latter to the previous expression:
\begin{equation*}
\nLtwo{e_h} = \int_\dom \Grad (\varphi - \varphi_h)\xDot \Grad e_h\dx + \underbrace{\int_\dom \Grad \varphi_h\xDot \Grad e_h\dx}_{0}
\end{equation*}
First we use Cauchy-Schwarz and make the $\xHone$-norm of the approximation errors appear since we control them by Equation \eqref{eq:h1err_eh} and \eqref{eq:h1err_phih}:
\begin{equation*}
\nLtwo{e_h} \leq \nHoneD{\varphi - \varphi_h} \nHoneD{e_h}
\end{equation*}

\medskip
Replacing by the bounds from the interpolation inequalities we get:
\begin{equation*}
\nLtwo{e_h} \leq C_1\;C_2\;\sizeT^2\;\nLtwo{f}
\end{equation*}
which concludes the proof. We have then a second order error estimate in $\xLtwo$.
\end{proof}

\newpage
%-------------------------------------------------------------------------------
\section{Exercises}


