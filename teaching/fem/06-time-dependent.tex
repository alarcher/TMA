
\chapter{Time-dependent problems}\label{chap:td}

The objective of this section is to introduce the \apriori stability analysis of time-dependent problems on several examples to obtain estimates similar to Lemma \ref{lm:stability_elliptic}.

\section{Time marching schemes}

In this section, problems describing the evolution in time of an unknown $u$ will be considered.
Since the unknown depends on the space coordinates and on the time, such evolution problem will be posed on a domain which is the open cylinder $\cyl = \dom \times (0,T)$.

\medskip
The Initial and Boundary Value problem for a Partial Differential Equation will therefore take a form such as
\begin{equation*}
\left\lvert
\begin{array}{ll}
\mbox{Find $u(\xx, t)$ satisfying:}&\\[2ex]
\partial_t u(\xx,t) + \matA\;u(\xx,t) = f(\xx,t) &, \forany(\xx,t)\in\cyl\\
u(\xx,t) = g(\xx,t) &, \forany\xx\in\bound, t\in(0,T) \\
u(\xx,0) = u_0(\xx) &, \forany\xx\in\bound
\end{array}
\right.
\end{equation*}
in the case of a Dirichlet problem which is first-order in time, and with $\matA$ a differential operator in space.
The differential operator and the right-hand side may depend on $u$, in which case the problem becomes non-linear.

\medskip
The equation can be recast under the form of a Cauchy problem
\begin{equation}\label{eq:td_cauchy}
\dot u(\xx,t) = F(\xx,t; u)
\end{equation}
so that an evolution problem can be seen as the coupling between a partial differential equation in space, and an ordinary differential equation in time.

This suggests that two discretizations may be considered: a Finite Element discretization in space which was discussed for elliptic problems in Chapter \ref{chap:rg} and Chapter \ref{chap:fem}, and a time discretization.
Similarly to the one-dimensional spatial case, the time discretization consists of solving the problem on a partition of $(0,T)$.
Let us define a family $\Fam{t^n}_{0\leq n \leq N}$ of $N+1$ discrete times, with $t^0 < \cdots < t^N$, and integrate the equation on each subinterval $[t^{n-1}, t^{n}]$, $1\leq n \leq N$, characterized by the time-step $\kt^n = t^{n} - t^{n-1}$.
Marching in time consists of solving a succession of problems at discrete times $t^n$, $1\leq n \leq N$ given solutions at previous discrete times.

\medskip
For example, in the case of a first order approximation in time, the discrete time-derivative is
\begin{equation}
\pd{t,n} u = \dfrac{u^{n} - u^{n-1}}{\kt^n}
\end{equation}
for $n = 1,\dots,N$, so that Relation \eqref{eq:td_cauchy} reads
\begin{equation}\label{eq:td_scheme}
u^n = u^{n-1} + \kt^n F(\xx,t; u)
\end{equation}
and the treatment of term $F(\xx,t; u)$ is left to be determined as it can be expressed as a function of $u^n$ but also of solutions $u^{n-k}$, $k = 1,\dots, n$ at previous time-steps.

\medskip
The choice of the time-derivative and the way $F(\xx,t; u)$ is expressed will define the type of numerical scheme.
For example, in Relation \eqref{eq:td_scheme} $F(\xx,t; u)$ can be evaluated at time $t^n$,
\begin{equation}\label{eq:td_implicit}
u^n = u^{n-1} + \kt^n F^n(\xx,t; u)
\end{equation}
or at time $t^{n-1}$,
\begin{equation}\label{eq:td_implicit}
u^n = u^{n-1} + \kt^n F^{n-1}(\xx,t; u)
\end{equation}
which correspond respectively to Backward Euler and Forward Euler schemes.
The former is an \textit{implicit scheme} as the term $F(\xx,t; u)$ depends on the unknown $u^n$, while the latter is an \textit{explicit scheme} as the term $F(\xx,t; u)$ is expressed in terms of $u^{n-1}$ which is known.
In a more general fashion, the theta-scheme reads,
\begin{equation}\label{eq:td_implicit}
u^n = u^{n-1} + \kt^n \bigl[\theta F^n(\xx,t; u) + (1-\theta)F^{n-1}(\xx,t; u)\bigr]
\end{equation}
with parameter $\theta \in [0,1]$, so that Backward Euler is recovered for $\theta = 1$, Forward Euler is recovered for $\theta = 0$, and Crank--Nicolson corresponds to the choice $\theta = \sfrac{1}{2}$.

\medskip
Stability and accuracy properties of the numerical scheme will depend on which terms are chosen as implicit or explicit: without going into the details and as a general rule implicit schemes tend to be more stable while explicit schemes will be limited by a condition on the time-step.
You can refer Von Neumann stability analysis for ordinary differential equations, and the Courant--Friedrichs--Levi (CFL) condition.

\medskip
Regardless of the numerical scheme, properties of solutions will also depend on the regularity of the initial condition and the nature of the differential operator.
Parabolic equations involving an elliptic operator will enjoy a smoothing property, given that energy dissipation is induce by diffusion-type operators, while hyperbolic equations may give rise to discontinuities and see the propagation of shocks.


\section{\textit{A priori} stability estimate}

In a similar fashion as elliptic problems, \textit{a priori} estimates can be derived by a careful choice of test function.

\subsection{Heat equation}

In the continuity of the Poisson problem the following unsteady problem is considered
\begin{equation*}
\left\lvert
\begin{array}{l}
\partial_t u(\xx,t) - \Delta u(\xx,t) = f(\xx,t) \\
u(\xx,t) = 0 \\
u(\xx,0) = u_0(\xx)
\end{array}
\right.
\end{equation*}
which corresponds to the case $\matA\;u = - \Delta\;u$.

\begin{equation*}
\InnerP{\matA\;u}{v} = - \intD \Delta\;u v \dxx = \intD \Grad u \cdot \Grad v \dxx
\end{equation*}

Firstly, let us derive the energy estimate for the heat equation in the by recalling the weak form and then taking the test function to be the unknown $u$:
\begin{eqnarray*}
\intD \pdt u\, v \dxx + \kappa \intD \Grad u \cdot \Grad v \dxx &=& \intD f\,v \dxx \\[2ex]
\intD \pdt u\, u \dxx + \kappa \intD | \Grad u |^2 \dxx &=& \intD f\,u \dxx \\
\dfrac{1}{2}\intD \pdt |u|^2 \dxx + \kappa \intD | \Grad u |^2 \dxx &=& \intD f\,u \dxx \\[2ex]
\dfrac{1}{2} \ddd{t} \intD  |u|^2 \dxx + \kappa \intD | \Grad u |^2 \dxx &=& \intD f\,u \dxx \\[2ex]
\dfrac{1}{2} \ddd{t} \nLtwo{u}^2 + \kappa \snHone{u}^2 \dxx &=& \intD f\,u \dxx
\end{eqnarray*}

In the case of an homogeneous equation, the latest relation is directly the instantaneous conservation of energy
\begin{equation}
\dfrac{1}{2} \ddd{t} \nLtwo{u}^2 + \kappa \snHone{u}^2 = 0
\end{equation}
with the first term being the variation of kinetic energy and the second term being the dissipation of energy with diffusion coefficient $\kappa$.
Integrating over the time interval, we get the energy budget over $[0,T]$:
\begin{equation}\label{eq:energy_budget}
\dfrac{1}{2} \nLtwo{u}^2 + \kappa \int_{0}^T \snHone{u}^2 \dt = 0
\end{equation}

\medskip
Let us consider now a non-zero source term $f$, then using the Cauchy--Schwarz inequality yields the following relation:
\begin{equation}
\dfrac{1}{2} \ddd{t} \nLtwo{u}^2 + \kappa \snHone{u}^2 \leq \nLtwo{f} \nLtwo{u}
\end{equation}
Since the bound should depend only on the data, the name of the game is to absorb any term involving the unknown in the left-hand side.
To this purpose, inequalities like Hölder, Korn, Sobolev injections are to be used in order to get a power of the proper $\xL^p$ or $\xH^s$ norm of the unknown.
In the case of coercive problems, the diffusion term giving directly the $\xHone$ seminorm (to a factor depending on the diffusive coefficient), we should try to make it pop from the right-hand side.
Using first the Poincaré inequality (Lemma \ref{lm:poincare}) and then the Young inequality (Lemma \ref{lm:young}), we can bound the right-hand side by the data and the $\xHone$ seminorm,
\begin{equation}\label{eq:apriori_rhs_bound}
\nLtwo{f} \nLtwo{u} \leq \dfrac{1}{2 \gamma^2 c_P^2} \nLtwo{f}^2 + \dfrac{\gamma^2}{2}\snHone{u}^2
\end{equation}
with $\gamma$ a positive real number which can be chosen arbitrarily.
Therefore, as soon as we choose $\gamma < \sqrt{2 \kappa}$, it is possible to subtract the second term of \eqref{eq:apriori_rhs_bound} to the left-hand side of the estimate, given that
\begin{equation}
\dfrac{1}{2} \ddd{t} \nLtwo{u}^2 + \dfrac{2\kappa - \gamma^2}{2}  \snHone{u}^2 \leq \dfrac{1}{2 \gamma^2 c_P^2} \nLtwo{f}^2
\end{equation}

Consequently, taking $\gamma= \sqrt{\kappa}$ there exists a constant $C > 0$ depending on the Poincaré constant, such that
\begin{equation}
\ddd{t} \nLtwo{u}^2 + \kappa  \snHone{u}^2 \leq C(c_P) \nLtwo{f}^2
\end{equation}

This inequality yields a control of the $\xLtwo$ norm and $\xHone$ seminorm of the solution at any time $t$ of the time interval $[0,T]$.
Similarly to Equation \eqref{eq:energy_budget}, if we integrate over the time, we get
\begin{equation*}
\nLtwo{u(T)}^2 - \nLtwo{u(0)}^2 + \kappa \int_{0}^T \snHone{u}^2 \dt \leq C(c_P) \int_{0}^T \nLtwo{f}^2 \dt
\end{equation*}
which, by defining,
\begin{equation}
\norm{v}_{\xL^{r}(0,T;\xL^{p}(\dom))} = \Grp{\int_{0}^T \norm{v}_{\xL^p(\dom)}^{r} \dt}^{1/r}
\end{equation}
can be rewritten as
\begin{equation*}
\nLtwo{u(T)}^2 + \kappa \norm{u}_{\xLtwo(0,T;\xHonec(\dom))}^2 \leq C(c_P) \norm{f}_{\xLtwo(0,T;\xLtwo(\dom))}^2 + \nLtwo{u(0)}^2
\end{equation*}
The solution is said to be bounded in $\xLinfty(0,T;\xLtwo(\dom))$, \ie $u \in \xLtwo(\dom)$ for almost every $t \in [0,T]$, and is it also bounded in $\xLtwo(0,T;\xHone(\dom))$ by the data (provided that $f \in \xLtwo(0,T;\xLtwo(\dom))$ of course).

\medskip
Now, if we turn to the discrete case the estimate is not different aside from the the discrete time derivative.
The term for the discrete time derivative in the case of backward Euler reads
\begin{equation*}
\dfrac{1}{\kt}\intD (u - u^\ast)\, u \dxx
\end{equation*}
with $\kt$ the current time step, $u$ and $u^\ast$ respectively the solution at the current and previous time stepping.

\subsubsection{$\xHtwo$}

Take $v = - t \Delta u$.

\begin{equation*}
- \intD \bigl(\partial_t u - \Delta u\bigr)\,t\;\Delta u \dxx
\end{equation*}


\begin{equation*}
\intD t \Grad(\partial_t u)\cdot\Grad u  \dxx + \intD t \snorm{\Delta u}^2 \dxx
\end{equation*}

\begin{equation*}
t \intD \partial_t(\Grad u)\cdot\Grad u \dxx + t \intD \snorm{\Delta u}^2 \dxx
\end{equation*}

\begin{equation*}
\dfrac{t}{2}\dfrac{\md}{\dt}\snHone{u}^2 + \snHtwo{u}^2
\end{equation*}

\begin{equation*}
\dfrac{1}{2}\dfrac{\md}{\dt}\bigl(t \snHone{u}^2\bigr) - \dfrac{1}{2} \snHone{u}^2 + t\snHtwo{u}^2
\end{equation*}

\begin{equation*}
\dfrac{T}{2}\snHone{u(T)}^2 + \int_0^T t\snHtwo{u}^2 \dt - \dfrac{1}{2}\int_0^T \snHone{u}^2 = \underbrace{\dfrac{0}{2}\snHone{u(0)}^2}_{0}
\end{equation*}

\begin{equation*}
+ t\snHtwo{u}^2 = + \dfrac{1}{2}\int_0^T \snHone{u}^2 - \dfrac{T}{2}\snHone{u(T)}^2
\end{equation*}

Using the previous control in $\xHone$ allows us to conclude.






















