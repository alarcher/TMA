
\chapter{Adaptive error control}\label{chap:aec}

In Section \ref{chap:ea}. we derived \apriori error estimates which give a control of the discretization error for any approximate solution.
The order of convergence given by the exponent $O(\sizeT^\alpha)$ is an indication on ``how close'' to the continuous solution any approximate solution is expected to be.
Provided that we are able to compute an approximate solution $\uh$, we want now to evaluate the ``quality'' of this solution in the sense of the residual of the equation: such an estimate is thus called \aposteriori as it gives a quality measure of a computed solution.

\medskip
\Question How can we evaluate the quality of an approximate solution computed on a given mesh to improve the accuracy?

\section{\textit{A posteriori} estimates}

In Chapter



\[
\Inner{\Res{\uh}}{v} = L(v) - a(\uh,  v)
\]

\[
\Inner{\Res{\uh}}{v} = a(u - \uh,  v)
\]

and the \textit{approximation error} $e_h = u - \uh$ belongs to $\xV$.


\begin{enumerate}
\item Coercivity:
\begin{equation}\label{eq:adapt_coercivity}
\alpha \norm{u - \uh}_{\xV}^2 \leq a(u - \uh,  u - \uh)
\end{equation}


\[
a(u - \uh,  u - \uh) = \Inner{\Res{\uh}}{u -\uh}
\]

\begin{equation}\label{eq:adapt_lower}
\alpha \norm{e_h}_{\xV} \leq \norm{\Res{\uh}}_{\xV'}
\end{equation}


\item Continuity:
\begin{equation}\label{eq:adapt_continuity}
a(u - \uh,  u - \uh) \leq M \norm{u - \uh}_{\xV}
\end{equation}


\[
\norm{\Res{\uh}}_{\xV'} = \sup_{\substack{v\in\xV\\v\neq 0}} \dfrac{a(u - \uh, v)}{\norm{v}_\xV}
\]

\begin{equation}\label{eq:adapt_upper}
\norm{\Res{\uh}}_{\xV'} \leq M \norm{e_h}_{\xV}
\end{equation}


\end{enumerate}

Using only the coercivity \eqref{eq:adapt_coercivity} and the continuity \eqref{eq:adapt_continuity} of the bilinear form, it follows from relations \eqref{eq:adapt_lower} and \eqref{eq:adapt_upper} that estimating $\Res{\uh}$ is equivalent to estimating the \textit{approximation error} in the norm of $\xV$.
Given that $\uh$ is know the quantity $\Res{\uh}$ is computable and can be used to derive \textit{a posteriori} error estimators.



\begin{itemize}
\item $h$-adaptivity
\item $p$-adaptivity
\item $r$-adaptivity
\end{itemize}

\section{Residual-based error estimator for Poisson}

Let $u$ and $\uh$ be respectively the solutions to Problem \eqref{pb:weak_poissonHone} and the approximate Problem \eqref{pb:poisson_p1} by a Lagrange $\xPone$ discretization.
Therefore the ideas of the previous section are applied with $\xV = \xHonec(\dom)$ and $\xVh$ the space of continuous piecewise linear functions vanishing on $\bound$.

\medskip
The objective of this section is to exhibit a control of the $\xHone$ seminorm of the error
\begin{equation*}
\snHoneD{e_h}^2 = \intD \Grad e_h \xDot \Grad e_h \dxx
\end{equation*}
in terms of the residual of the equation $\Res{\uh}$.

\medskip
By Galerkin orthogonality $a(e_h, \vh) = 0$ for $\vh \in \xVh$, in particular testing against $\vh = \Iop{h} e_h$ is possible so that
\begin{equation*}
\snHoneD{e_h}^2 =  = \intD \Grad e_h \xDot \Grad (e_h - \Iop{h} e_h)\dxx
\end{equation*}

%                &=& \intD \Grad u \xDot  \Grad (e_h - \Iop{h} e_h)  \dxx - \intD \Grad \uh \xDot \Grad (e_h - \Iop{h} e_h) \dxx \\
%                &=& \intD f\; (e_h - \Iop{h} e_h)  \dxx - \intD \Grad \uh \xDot \Grad (e_h - \Iop{h} e_h) \dxx
%\end{eqnarray*}
To obtain the residual $\Res \uh$ we need to consider the equation element-wise, then integrating by part on any cell $K \in \meshT$, we obtain
\begin{equation*}
\int_K \Grad \uh \xDot \Grad (e_h - \Iop{h} e_h) \dxx = \int_{\partial K} \Grad \uh \xDot \n\;(e_h - \Iop{h} e_h) \dxx - \int_K \Delta \uh \; (e_h - \Iop{h} e_h) \dxx
\end{equation*}
with
\begin{equation*}
\ResK{\uh} = (f + \Delta \uh)\rvert_K
\end{equation*}
Summing again over the domain yields
\begin{equation*}
\snHoneD{e_h}^2 = \sum _{K\in\meshT}\left[\int_K \ResK{\uh}\; \Grad (e_h - \Iop{h} e_h) \dxx + \int_{\partial K} \Grad \uh \xDot \n\;(e_h - \Iop{h} e_h) \dxx \right]
\end{equation*}
noting that in the case of continuous elements the boundary term cancels.
Using first the Cauchy--Schwarz inequality
\begin{equation*}
\snHoneD{e_h}^2 \leq \nLtwo{\Res{\uh}} \nLtwo{e_h - \Iop{h} e_h}
\end{equation*}
then the interpolation inequality with constant $C_I$
\begin{equation*}
\snHoneD{e_h}^2 \leq C_I \nLtwo{\Res{\uh}} \snHoneD{h e_h}
\end{equation*}
Consequenlty, we conclude
\begin{equation*}
\snHoneD{e_h} \leq C_I \sizeT \nLtwo{\Res{\uh}}
\end{equation*}

\section{Dual weighted residual estimate}

\subsection{Adjoint operator}

\begin{dfntn}[Adjoint operator]
Let us define $\opA\dual$, the adjoint operator of $\opA$ as:
\begin{equation*}
\Inner{\opA u}{v} = \Inner{u}{\opA\dual v}
\end{equation*}
\end{dfntn}

\begin{xmpl}[Matrix of $\xMatR{N}$]
Let $\opA = \matA$ be a real square matrix of dimension $N\times N$ and $x,y \in \xR^N$:
\begin{equation*}
\Inner{\opA x}{y} = \Inner{\matA x}{y} = \Inner{x}{\matA\trans y} = \Inner{x}{\opA\dual y}
\end{equation*}
with $\Inner{\xDot}{\xDot}$ the scalar product of $\xR^N$, then $\opA\dual = \matA^t$.
\end{xmpl}

\begin{xmpl}[Weak derivative]
Let $\opA = {\mathrm D}_x $ and $u,v \in \xLtwo(\dom)$, with compact support on $\dom$:
\begin{equation*}
\Inner{\opA u}{v} = \Inner{{\mathrm D}_x u}{v} = -\,\Inner{u}{{\mathrm D}_x v} = \Inner{u}{\opA\dual v}
\end{equation*}
with $\Inner{\xDot}{\xDot}$ the scalar product of $\xLtwo(\dom)$, then $\opA\dual = - {\mathrm D}_x$.
\end{xmpl}

\begin{xmpl}[Laplace operator]
Let $\opA = -\Lap$ and $u,v \in \xHonec(\dom)$:
\begin{equation*}
\Inner{\opA u}{v} = \Inner{- \Lap u}{v} = \Inner{\Grad u}{\Grad v} = \Inner{u}{- \Lap v} = \Inner{u}{\opA\dual v}
\end{equation*}
with $\Inner{\xDot}{\xDot}$ the scalar product of $\xLtwo(\dom)$, then $\opA\dual = - \Lap$. The Laplace operator is said \textit{self-adjoint}.
\end{xmpl}

\subsection{Duality-based \aposteriori error estimate}

We define the dual problem as seeking $\eta$ satisfying $\opA^\star\eta = e_h$, which gives a control on the discretization error, using the definition of the adjoint operator $\opA^\star$:
\begin{eqnarray*}
\nLtwo{e_h} &=& \Inner{e_h}{e_h}\\
            &=& \Inner{e_h}{\opA^\star\eta}\\
            &=& \Inner{\opA e_h}{\eta}\\
            &=& \Inner{\opA u}{\eta} - \Inner{\opA \uh}{\eta}\\
            &=& \Inner{f - \opA \uh}{\eta}\\
            &=& \Inner{\Res \uh}{\eta}
\end{eqnarray*}
with $\Res \uh = f -  \opA \uh$. Moreover, if the dual problem is stable then there exists a constant $\Stab$ such that the dual solution $\eta$ is bounded:
\begin{equation*}
\nLtwo{\eta} \leq \Stab \nLtwo{e_h}
\end{equation*}
with the stability factor $\Stab$ satisfying
\begin{equation*}
\Stab = \max_{\theta\in\xLtwo(\dom)}\dfrac{\snHtwoD{\eta}}{\nLtwo{\theta}}
\end{equation*}
Thus we can obtain a bound of the form:
\begin{equation*}
\nLtwo{e_h} \leq \Stab \nLtwo{\Res \uh}
\end{equation*}

\medskip
Combining this estimate with an interpolation inequality in $\xH^\alpha$, we can bound the discretization error in terms of the residual and the stability factor.
For instance, if we control the second derivatives of the dual solution, \ie $\alpha = 2$,
\begin{eqnarray*}
\nLtwo{e_h} \leq C_I\;\nLtwo{h^2\Res \uh}\dfrac{\snHtwoD{\eta}}{\nLtwo{e_h}}
\end{eqnarray*}
Consequently,
\begin{eqnarray*}
\nLtwo{e_h} \leq C_I\;\Stab\nLtwo{h^2\Res \uh}
\end{eqnarray*}

\section{Method}

\begin{dfntn}[$h$-adaptivity]
Given a tolerance parameter $\Etol > 0$ defining a quality criterion for the computed solution $\uh$, adapt the mesh such that it satisfies:
\begin{equation*}
\EindT = \sum_{K\in\meshT} \EindK < \Etol
\end{equation*}
\end{dfntn}

\medskip
\begin{lgrthm}[Adaptive mesh strategy] The following
procedure applies:
\begin{itemize}
\item Generate an initial coarse mesh $\meshT^0$.
\item Perform adaptive iterations for levels $\ell = 0, \cdots,\ell_{\mathrm max}$\;:
\begin{enumerate}
\item Solve the primal problem with solution $\uh^0 \in \xVh^\ell$.
\item Compute the residual of the equation $\Res{\uh^\ell}$.
\item If dual weighted, solve the dual problem with solution $\eta \in \xW_h^\ell$.
\item Compute error indicators $\EindK$, $\forany K \in \meshT^\ell$.
\item
If ($\EindT \geq \Etol$)\;:\\
$\rightarrow$ Generate mesh $\meshT^{\ell+1}$ by refining cells with largest values of $\EindK$.\\
Else\;:\\
$\rightarrow$ Terminate adaptive iterations, $\ell_{\mathrm max} = \ell$.
\end{enumerate}
\end{itemize}
\end{lgrthm}

\section{Exercises}

\begin{tmaxrcs}{Diffusion--Reaction problem on the unit interval}{7.1}

Consider the following one-dimensional problem:
\begin{equation*}
- \pdx\bigl(a(x)\;\pdx u(x)\bigr) + c(x)\;u(x) = f(x)\quad,\;\forany x\in \dom = [0,1]
\end{equation*}
with $a > 0$, $c \geq 0$, and supplemented with homogeneous Dirichlet boundary conditions
\begin{equation*}
u(0) = u(1) = 0
\end{equation*}
\begin{enumerate}
\item Write the weak formulation for the given problem and its approximation by piecewise linear Lagrange elements.
\item Write the dual problem for unknown $\eta$.
\item Obtain the following estimate:
\begin{equation*}
\nLtwo{e_h} \leq \nLtwo{h^2 \Res \uh} \nLtwo{h^{-2} (\eta - \Lgrh{1}\eta)}
\end{equation*}
with the discretization error $e_h = u - \uh$, the equation residual $\Res \uh = f + \pdx\bigl(a\,\pdx \uh\bigr) - c\,\uh$ and the Lagrange $\xPone$ interpolation operator $\Lgrh{1}$. First you should test the dual equation against $e_h$, then write the expression element-wise to be able to define the residual.
\item Conclude that the \aposteriori error estimate holds
\begin{eqnarray*}
\nLtwo{e_h} \leq C_I\;\Stab\nLtwo{h^2\Res \uh}
\end{eqnarray*}
with $C_I$ the interpolation constant and $\Stab$ a stability factor that you will define.
\end{enumerate}
\end{tmaxrcs}


