
\section{Chapter 1}

\begin{tmasltn}{1.1}
\begin{tmatsks}
\item False. Justify by verifying that linear combination of functions of $S$ do not satisfy $v(\frac{1}{2}) = 1$.
\item True. Verify that the functional is linear, $L(\lambda u + \mu v) = \lambda L(u) + \mu L(v)$,
\item False. Show that $\InnerP{\cdot}{\cdot}{\xV}$ is not bilinear.
\item False. Verify that $\snorm{u}_{\xHone(\dom)} = 0$ for $u = c$ so it is not positive definite, only positive: $\snorm{\cdot}_{\xHone(\dom)}$ is a semi-norm for $\xHone(\dom)$ but a norm for $\xHonec(\dom)$ since the only constant satisfying the trace is $u = 0$.
\item Verify that the function and derivatives are square integrable:
\begin{equation*}
 \norm{v}^2_{\xLtwo((0,1))} = \int_{0}^{1} x^3 \md x = \frac{2}{3}
\end{equation*}
\begin{equation*}
 \snorm{v}^2_{\xHone((0,1))} = \int_{0}^{1} \frac{9}{16} x^{-1/2} \md x = \frac{9}{8}
\end{equation*}
\begin{equation*}
 \snorm{v}^2_{\xHtwo((0,1))} = \int_{0}^{1} \frac{-9}{32} x^{-3/2} \md x\quad \mbox{not bounded}
\end{equation*}
\item False. Express both semi-norm and compare that they are different since the derivation of the exponential gives a factor $-10$.
\end{tmatsks}
\end{tmasltn}

\begin{tmasltn}{1.2}
\begin{tmatsks}
\item To get the strong problem of \eqref{pb:adr} integate the second by part to get all the derivatives on the solution. Since there is no linear form associated with Neumann boundary conditions and the solution space is $\xHonec(\dom)$ the boundary condition of the problem is an homogeneous Dirichlet condition on $\bound$.
\begin{equation*}
 \kappa u'(x) - u''(x) + u(x) = f(x), x\in\dom,\quad u(0) = u(1) = 0
\end{equation*}
\item Verify that the Problem \eqref{pb:adr} statisfies the hypothesis of the Lax--Milgram theorem.
\end{tmatsks}
\end{tmasltn}

\begin{tmasltn}{1.3}
\begin{tmatsks}
\item To derive a weak formulation, test the equation against $v \in \xCinfc(\dom)$ and integrate by part to report derivatives on $v$.
\begin{equation*}
\DualP{\Lap^2 u}{v} = \Inner{\Lap u}{\Lap v}
\end{equation*}
Given that $\xCinfc(\dom)$ is dense in $\xHtwo(\dom)$ it is then possible to find 
\begin{equation*}\label{pb:weak_biharmonic}
\left\lvert
\begin{array}{ll}
\mbox{Find $u \in \xHtwo_0(\dom)$ such that:}\\[2ex]
\displaystyle\int_\dom \Lap u \Lap v\dx = \int_\dom f v  \dx\quad,\;\forany  v\in \xHtwo_0(\dom)
\end{array}
\right .
\end{equation*}
\item The solution and test space are suggested to be $\xHtwo_0(\dom)$ given that the bilinear form is the $\xLtwo$ inner product of $\Lap u$ when $v = u$, and the boundary conditions cancel the function and first derivative.
\item Given that the bilinear form is the inner product of $\xHtwo_0(\dom)$ the Riesz theorem applies directly.
\end{tmatsks}
\end{tmasltn}


%\begin{tmasltn}{1.4}
%\begin{tmatsks}
%\item Derive a weak formulation (WF) of Problem \eqref{pb:helmoltz}.
%\item Specify the solution and test spaces.
%\item What is the nature of the bilinear form for $\kappa = 1$?
%\item Prove that the problem is well-posed for $\kappa = 0$ and $\kappa > 0$.
%\item Comment on the difficulty posed by the case $\kappa < 0$.
%\item The boundary condition is now given by:
%\begin{equation}
% u(0) - u'(0) = 0, u'(1) = -1
%\end{equation}
%Derive a weak formulation for the Problem \eqref{pb:helmoltz_eq}--\eqref{pb:helmoltz_mixed} and show that it admits a unique solution.
%To prove the coercivity the following relation can be used:
%\[
%v(1) = v(x) + \int_0^1 v'(t) \md t
%\]
%\end{tmatsks}
%\end{tmasltn}

