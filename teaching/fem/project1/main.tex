\documentclass[assignment]{tmanotes}
\settmacourse{TMA4220}
{Numerical Solution of Partial Differential Equations Using Finite Element Methods}
{Numerical Solution of PDEs Using Finite Element Methods}

%-------------------------------------------------------------------------------
% @collection	Discretization
% @description
% @version		1
% @author		Aurélien Larcher <aurelien.larcher@gmail.com>
%-------------------------------------------------------------------------------
% @category		Discretization
%---------------------------------------
% @subcategory	Discretization subscript
% @macro disc		admissible discretization
\newcommand{\disc}{{\mathcal M}}
% @macro disc		space-time discretization
\newcommand{\discd}{{\mathcal D}}
% @macro disc		simplicial discretization
\newcommand{\disct}{{\mathcal T}}
%---------------------------------------
% @subcategory	Mesh
% @macro mesh		admissible finite volume mesh
\newcommand{\mesh}{{\mathcal M}}
% @macro xD
\newcommand{\xD}{{\rm D}}
% @macro size		size
\newcommand{\size}{{\rm size}}
%---------------------------------------
% @subcategory	Mesh topology
\newcommand{\mgraph}[2]{\bigl(\;#1,\:#2\;\bigr}
% @macro connect		connectivity
\newcommand{\connect}[2]{{\mathcal C}_{#1, #2}}
% @macro edges		set of cells of the discretization
\newcommand{\cells}{{\mathcal{K}}}
\newcommand{\cell}{{K}}
% @macro edges		set of vertices of the discretization
\newcommand{\vertices}{{\mathcal{V}}}
\newcommand{\vertex}{{\mathrm v}}
%---------------------------------------
% @subcategory	Edges
% @macro edge		edge
\newcommand{\edge}{\sigma}
% @macro edged		edge of diamond cell
\newcommand{\edged}{\xi}
% @macro edges		set of edges of the discretization
\newcommand{\edges}{\mathcal{E}}
% @macro edgesint	set of internal edges
\newcommand{\edgesint}{\edges_{{\rm int}}}
% @macro edgesext	set of external edges
\newcommand{\edgesext}{\edges_{{\rm ext}}}
% @macro edgesextD	set of external edges of the diamond mesh
\newcommand{\edgesextD}{{\cal E}_{{\rm ext,D}}}
% @macro nK			outward normal from K across edge
\newcommand{\nK}{\bfn_{\edge,K}}
% @macro nL			outward normal from L across edge
\newcommand{\nL}{\bfn_{\edge,L}}
% @macro nedge		normal vector on edge
\newcommand{\nedge}{\bfn_\edge}
% @macro nKL 		outward normal from K to L
\newcommand{\nKL}{\bfn_{KL}}

% @macro 			points	set of points of the discretization
\newcommand{\points}{\mathcal{P}}
% @macro neigh		neighbouring cells of K
\newcommand{\neigh}{{\mathcal N}(K)}

%-------------------------------------------------------------------------------
% @collection	Finite Element Method
% @description
% @version		1
% @author		Aurélien Larcher <aurelien.larcher@gmail.com>
%-------------------------------------------------------------------------------

\newcommand{\uh}{{u_h}}
\newcommand{\vh}{{v_h}}
\newcommand{\uuh}{{\boldsymbol u}_h}
\newcommand{\vvh}{{\boldsymbol v}_h}
\newcommand{\meshT}{{\mathcal T}_h}
\newcommand{\sizeT}{h_{\mathcal T}}
\newcommand{\CellK}{{K}}
\newcommand{\SpaceP}{{\mathcal P}}
\newcommand{\DualBasis}{{\Sigma}}
\newcommand{\RefK}{{\hat \CellK}}
\newcommand{\RefP}{{\hat \SpaceP}}
\newcommand{\RefSigma}{{\hat\DualBasis}}
\newcommand{\FE}{{(\CellK,\SpaceP, \DualBasis)}}
\newcommand{\RefFE}{{(\RefK,\RefP, \RefSigma)}}
\newcommand{\Ndof}{{N_{\mathrm dof}}}
\newcommand{\NxVh}{{N_{\xVh}}}
\newcommand{\Res}[1]{{\mathcal{R}(#1)}}
\newcommand{\ResK}[1]{{\mathcal{R}_K(#1)}}
\newcommand{\Stab}{\mathcal{S}}
\newcommand{\Etol}{\epsilon_{\mathrm tol}}
\newcommand{\EindK}{\epsilon_K}
\newcommand{\EindT}{\epsilon_{\mathcal T}}

\newcommand{\Iop}[1]{\mathop{\mathcal{I}_{#1}}}
\newcommand{\IopK}[1]{\mathop{\mathcal{I}_{K,#1}}}

% @category		Discrete spaces
%---------------------------------------
\newcommand{\xM}{{M}}
\newcommand{\xMh}{{\xM}_h}
\newcommand{\xMker}{{\xM}_0}
\newcommand{\xMz}{{\mathrm{\tilde M}}}
\newcommand{\xMzh}{{\mathrm{\tilde M}_h}}
\newcommand{\xV}{{V}}
\newcommand{\xVh}{{\xV}_h}
\newcommand{\xVdisc}{{\xV}_h}
\newcommand{\xVdiscm}{{\xVdisc}^{(m)}}
\newcommand{\xVd}{\boldsymbol{\xV}}
\newcommand{\xVhd}{\boldsymbol{\xV}_h}
\newcommand{\xVV}{\boldsymbol{W}}
\newcommand{\xWd}{\boldsymbol{\xVV}}
\newcommand{\xWdisc}{{\boldsymbol W}_h}
\newcommand{\xWdiscm}{{\xWdisc}^{(m)}}
\newcommand{\xPzero}{{\xP}_1}
\newcommand{\xPone}{{\xP}_1}
\newcommand{\xPtwo}{{\xP}_2}
\newcommand{\xPn}[1]{{\xP}_{#1}}
\newcommand{\xQone}{{\xQ}_1}
\newcommand{\xisoQone}{\tilde {\xQ}_1}
\newcommand{\xQtwo}{{\xQ}_2}
%-------------------------------------------------------------------------------
% @category		Operators
\newcommand{\Ph}{{\Pi_h}}
\newcommand{\rh}{{r_h}}
\newcommand{\rhv}{\boldsymbol{r}_h}
\newcommand{\Gradh}{{\boldsymbol \nabla_h}}
\newcommand{\Divh}{{\boldsymbol \nabla_h\cdot}}
\newcommand{\Divrh}{{\mathop{\mathrm div}_h}}
\newcommand{\Stabh}[1]{{S_h^{#1}}}
%-------------------------------------------------------------------------------
\def\snormFE{\norm}
\newcommand{\refelm}{\hat K}

%-------------------------------------------------------------------------------
% @collection	Latin Locutions
% @description 
% @version		1
% @author		Aurélien Larcher <aurelien.larcher@gmail.com>
%-------------------------------------------------------------------------------
\newcommand{\apriori}{\textit{a priori\/} }
\newcommand{\afortiori}{\textit{a fortiori\/} }
\newcommand{\aposteriori}{\textit{a posteriori\/} }
\newcommand{\cf}{\textit{cf.\/} }
\newcommand{\eg}{\textit{e.g.\/} }
\newcommand{\etal}{\textit{et al.\/} }
\newcommand{\etc}{\textit{etc\/} }
\newcommand{\ie}{\textit{i.e.\/} }
\newcommand{\loccit}{\textit{a loc. cit.\/} }
\newcommand{\vg}{\textit{v.g.\/} }

%-------------------------------------------------------------------------------
% Operators:
%-------------------------------------------------------------------------------
\newcommand{\Lgr}[2]{\mathcal L^{#1}_{#2}}
\newcommand{\Abs}[1]{\left\lvert #1 \right\rvert}
\newcommand{\card}{\mathrm{card}}
\newcommand{\fromto}{\rightarrow}
\newcommand{\tendsto}{\rightarrow}
\newcommand{\tendstoweak}{\rightharpoonup}
\newcommand{\forany}{{\forall\;}}
\newcommand{\m}{{(m)}}
\newcommand{\seq}[1]{\bigl({#1}^\m\bigr)_{m\in\xN}}
\newcommand{\seqn}[1]{\left({#1}^n\right)_{n\in\xN}}
\newcommand{\Proj}[1]{\,\mathrm{P}_{#1}\,}
\newcommand{\Projh}[1]{\,\mathrm{\pi}_{#1}\,}
\newcommand{\opA}{\mathcal{A}}

\newcommand{\Empty}{\varnothing}
\newcommand{\Union}{\mathop\bigcup}
\newcommand{\Inter}{\mathop\bigcap}
\newcommand{\Interior}[1]{\mathring{#1}}



%-------------------------------------------------------------------------------
% Miscellaneous
\newcommand{\Grp}[1]{\displaystyle\left({#1}\right)}
\newcommand{\indic}[1]{\displaystyle{1\!\!1_{#1}}}
\newcommand{\mean}[1]{\brace\lbrace#1\rbrace\rbrace}
\newcommand{\jump}[1]{[#1]}
% @macro diam		diameter
\newcommand{\diam}{{\mathrm diam}}

%-------------------------------------------------------------------------------
% Derivatives
\newcommand{\ds}{\hspace{.5ex}{\mathrm d}s}
\newcommand{\md}{\hspace{.5ex}{\mathrm d}}
%
\newcommand{\dt}{\delta \hspace{-0.05ex} t}
\newcommand{\pdt}{\partial_t}
\newcommand{\ddt}{\hspace{.5ex}{\mathrm d}t}
%
\newcommand{\pd}[1]{\partial_{#1}\,}
\newcommand{\pdx}{\partial_x}
\newcommand{\dxd}{\hspace{.5ex}{\mathrm d}\bfx}
\newcommand{\dx}{\hspace{.5ex}{\mathrm d}\bfx}
\newcommand{\dy}{\hspace{.5ex}{\mathrm d}\bfy}
\newcommand{\dz}{\hspace{.5ex}{\mathrm d}\bfz}
%\newcommand{\ddx}{\frac{{\mathrm d}}{{\mathrm d}x}}
%\newcommand{\ddy}{\frac{{\mathrm d}}{{\mathrm d}y}}
%\newcommand{\ddz}{\frac{{\mathrm d}}{{\mathrm d}z}}
\newcommand{\ddx}{\frac{{\partial}}{{\partial}x}}
\newcommand{\ddy}{\frac{{\partial}}{{\partial}y}}
\newcommand{\ddz}{\frac{{\partial}}{{\partial}z}}

\newcommand{\pddx}[1]{\frac{{\partial #1}}{{\partial}x}}
\newcommand{\pddy}[1]{\frac{{\partial #1}}{{\partial}y}}
\newcommand{\pddz}[1]{\frac{{\partial #1}}{{\partial}z}}
\newcommand{\dd}[2]{\frac{{\partial #1}}{{\partial#2}}}
\newcommand{\ddd}[1]{\frac{{\mathrm d}}{{\mathrm d #1}}}

%-------------------------------------------------------------------------------
\newcommand{\xDot}{\,{\mathbf\cdot}\,}
\newcommand{\Cross}{\mathbf X}
\newcommand{\DualP}[2]{\mathbf{\langle}\;#1\:,\: #2 \;\mathbf{\rangle}}
\newcommand{\Inner}[2]{{{\scriptstyle\mathbf{(}}\;#1\:,\: #2 \;{\scriptstyle\mathbf{)}}}}
\newcommand{\InnerP}[3]{{{\scriptstyle\mathbf{(}}\;#1\:,\: #2 \;{\scriptstyle\mathbf{)}}}_{\,#3}}
\newcommand{\Span}{\mathop{\mathrm{span}}}
\newcommand{\T}{{^{\mathrm T}}}
\newcommand{\Trace}[2]{{\mathbf{Tr}}^{(#1)}(#2)}
\newcommand{\trans}{{^{\mathsmaller{T}}}}
\newcommand{\Tens}{{\otimes}}
\newcommand{\Vect}{\wedge}

%-------------------------------------------------------------------------------
% Differential Operators
% continuous
\newcommand{\D}{{\boldsymbol{\mathrm D}}} % gradient
\newcommand{\Grad}{{\boldsymbol\nabla}} % gradient
\newcommand{\Gradt}{{\boldsymbol\nabla^t}}
\newcommand{\Gradx}{\Grad_x}
\newcommand{\GradxR}{\Grad_{\xR}}
\newcommand{\Div}{{\boldsymbol\nabla\cdot\,}} % divergence
\newcommand{\Divr}{{\mathrm div}}
\newcommand{\Lap}{{\boldsymbol\Delta}} % laplacian
\newcommand{\Lapl}{{\boldsymbol\nabla^2}} % laplacian
\newcommand{\Curl}{{\Grad\boldsymbol\wedge}}
%
\newcommand{\udiv}{(\vec{\bfv}\cdot\nabla)} % advection
\newcommand{\uhdiv}{(\vec{\bfv_h}^n\cdot\nabla)}

\newcommand{\DualProd}[2]{{\langle #1,\: #2 \rangle}}

%-----------------------------------------------------------------------
%   Thermophysical Properties
%-----------------------------------------------------------------------
\newcommand{\dens}{\varrho}             % density
\newcommand{\cond}{\lambda}             % conductivity
\newcommand{\visc}{\eta}                % dynamic viscosity ($Pa.s$)
\newcommand{\kvisc}{\nu}                % kinematic viscosity ($m^2/s$)
\newcommand{\tension}{\sigma}           % surface tension
\newcommand{\Tt}{\Sigma}                % surface tension
\newcommand{\Ttm}{\underline{\Sigma}}   % constante
\newcommand{\cp}{c_p}                   % chaleur massique
\newcommand{\grav}{\mathbf g}           % gravity
\newcommand{\fric}{\xi}                 % coefficient de friction
\newcommand{\Stress}{\sigma}    % stress tensor
\newcommand{\Tvisc}{\tau}       % viscous stress tensor
\newcommand{\Strain}{\varepsilon}    % strain tensor

%-----------------------------------------------------------------------
%   Non dimensional numbers
%-----------------------------------------------------------------------
\newcommand{\CFL}{{\rm CFL}}            % CFL number
\newcommand{\Fr}{{\rm\bf Fr}}           % Froude number
\newcommand{\Nu}{{\rm\bf Nu}}		% Nusselt
\newcommand{\Pra}{{\rm\bf Pr}}		% Prandtl
\newcommand{\Rey}{{\rm\bf Re}}		% Reynodls
\newcommand{\We}{{\rm\bf We}}		%
\newcommand{\Ca}{{\rm\bf Ca}}		%
\newcommand{\La}{{\rm\bf La}}		%
\newcommand{\Ra}{{\rm\bf Ra}}		% Rayleigh
\newcommand{\Eo}{{\rm\bf Eo}}		%
%\newcommand{\Eot}{E\"otv\"os }		
\newcommand{\Pe}{{\rm\bf Pe}}		% Peclet
\newcommand{\Mor}{{\rm \bf M}}
\newcommand{\Adim}[1]{#1_{\diamond}}
\newcommand{\Ref}[1]{{#1}_{\rm ref}}    % deference quantity for adim


%-------------------------------------------------------------------------------
% @collection	Spaces (M2AN style)
% @description
% @version		1
% @author		Aurélien Larcher <aurelien.larcher@gmail.com>
%-------------------------------------------------------------------------------
% @category		Sets
%---------------------------------------

\newcommand{\aevery}{\textit{a.e\/} }

\newcommand{\dual}{^\star}
\newcommand{\ortho}{^\perp}
\newcommand{\ball}{\mathfrak B}

\newcommand{\cyl}{\mathrm{Q}} % Q = \dom \times (0,T)
\newcommand{\dom}{\Omega}
\newcommand{\domc}{\overline{\Omega}}
\newcommand{\bound}{{\partial\dom}}

% Integral on the space domain
\newcommand{\intD}{\int_\dom}
% Integral on the time interval
\newcommand{\intI}{\int_0^T}

%-------------------------------------------------------------------------------

\newcommand{\kernel}{\mathfrak N}
\newcommand{\nilset}{\lbrace 0\rbrace}
\newcommand{\range}{\mathfrak R}
\newcommand{\Fam}[1]{{\left\lbrace #1 \right\rbrace}}
\newcommand{\sFam}[1]{{\lbrace #1 \rbrace}}
\newcommand{\Set}[1]{{\left\lbrace #1 \right\rbrace}}
\newcommand{\One}{{\displaystyle{1\!\!1}}}
\newcommand{\xLin}{\mathcal{L}}
\newcommand{\Ker}{\mathrm{Ker}}
\newcommand{\Ima}{\mathrm{Im}}
\newcommand{\Basis}{\mathcal{B}}
\newcommand{\Coords}[1]{\left(#1\right)}

\newcommand{\cyldisc}{\mathrm{Q}_\discd}
\newcommand{\intdisc}[1]{[\hspace{-.02in}[{ #1}]\hspace{-.02in}]}

%-------------------------------------------------------------------------------

\newcommand{\xK}{\mathbb{K}}
\newcommand{\xR}{\mathbb{R}} %M2AN
\newcommand{\xQ}{\mathbb{Q}} %M2AN
\newcommand{\xZ}{\mathbb{Z}} %M2AN
\newcommand{\xN}{\mathbb{N}} %M2AN
\newcommand{\xP}{\mathbb{P}} %M2AN
\newcommand{\xA}{\mathbb{A}} %M2AN
\newcommand{\xComplex}{\mathbb{C}}
\newcommand{\xQuaternion}{\mathbb{H}} %M2AN
\newcommand{\xHil}{H}

% d-dimensional versions
\newcommand{\xKd}{{\xK^d}}
\newcommand{\xRd}{{\xR^d}}
\newcommand{\xQd}{{\xQ^d}}
\newcommand{\xZd}{{\xZ^d}}
\newcommand{\xNd}{{\xN^d}}
\newcommand{\xPd}{{\xP^d}}
\newcommand{\xAd}{{\xA^d}}

\newcommand{\xRn}{{\xR^n}}
\newcommand{\xRN}{{\xR^N}}

%-------------------------------------------------------------------------------
% @category		Space of Matrices
\newcommand{\xMat}{\mathcal{M}}
\newcommand{\xMatR}[1]{\mathcal{M}_{#1}(\xR)}
\newcommand{\xMnR}{{M_n(\xR)}}
\newcommand{\xMNR}{{M_N(\xR)}}
\newcommand{\xGL}{\mathop{\mathrm GL\,}\nolimits}
\newcommand{\xGLnR}{{\mathop{\mathrm GL_n(\xR)\,}\nolimits}}
\newcommand{\xGLNR}{{\mathop{\mathrm GL_N(\xR)\,}\nolimits}}
\newcommand{\xSL}{\mathop{\mathrm SL\,}\nolimits}
\newcommand{\xPSL}{\mathop{\mathrm PSL\,}\nolimits}
\newcommand{\xSO}{\mathop{\mathrm SO\,}\nolimits}
\newcommand{\xSU}{\mathop{\mathrm SU\,}\nolimits}

%-------------------------------------------------------------------------------
% @category		Continuously Differentiable Functions
\newcommand{\xC}{{\mathrm C}} %M2AN
\newcommand{\xCzero}{\xC^{0}} %M2AN
\newcommand{\xCone}{\xC^{1}} %M2AN
\newcommand{\xCtwo}{\xC^{2}} %M2AN
\newcommand{\xCinfty}{\xC^{\infty}} %M2AN
\newcommand{\xCn}[1]{\xC^{#1}} %M2AN
\newcommand{\xCinfc}{{\xCinfty_c}}

%-------------------------------------------------------------------------------
% @category		Lebesgue Spaces
\newcommand{\xL}{{\mathrm L}}
\newcommand{\xLzero}{\xL^{0}}
\newcommand{\xLone}{\xL^{1}}
\newcommand{\xLoneloc}{\xLone_{loc}}
\newcommand{\nLone}[1]{\norm{#1}_{\xLone(\dom)}}
\newcommand{\nLoned}[1]{\norm{#1}_{\xLone(\dom)^d}}
\newcommand{\xLtwo}{\xL^{2}}
\newcommand{\nLtwo}[1]{\norm{#1}_{\xLtwo(\dom)}}
\newcommand{\nLtwod}[1]{\norm{#1}_{\xLtwo(\dom)^d}}
\newcommand{\xLtwoR}[1]{{\xLtwo(#1)/\xR}}
\newcommand{\xLtwoz}{\xLtwo_0}
\newcommand{\xLp}[1]{\xL^{#1}}
\newcommand{\nLp}[2]{\norm{#2}_{\xLp{#1}(\dom)}}
\newcommand{\xLinfty}{\xL^{\infty}}
\newcommand{\nLinfty}[1]{\norm{#1}_{\infty}}

%-------------------------------------------------------------------------------
% @category		Hardy Spaces
\newcommand{\xH}{{\mathrm H}} %M2AN
\newcommand{\xHd}{\boldsymbol{\mathrm H}}
\newcommand{\xHzero}{\xH^{0}} %M2AN
\newcommand{\xHone}{\xH^{1}} %M2AN
\newcommand{\xHoned}[1]{{\xHone(#1)^d}}
\newcommand{\xHonec}{{\xHone_0}}
\newcommand{\xHonecb}{{\xHone_{0,b}}}
\newcommand{\xHonecd}[1]{{\xHonec(#1)^d}}
\newcommand{\xHtwo}{\xH^{2}} %M2AN
\newcommand{\xHinfty}{\xH^{\infty}} %M2AN
\newcommand{\xHn}[1]{\xH^{#1}} %M2AN
\newcommand{\xHmone}{\xH^{-1}}
\newcommand{\xHmoned}[1]{{\bigl(\xHmone(#1)\bigr)^d}}

%-------------------------------------------------------------------------------
% @category		Sobolev Spaces
\newcommand{\xW}{{\mathrm W}} %M2AN
\newcommand{\xWzero}{\xW^{0}} %M2AN
\newcommand{\xWone}{\xW^{1}} %M2AN
\newcommand{\xWoneq}[1]{\xW^{1,#1}}
\newcommand{\xWtwo}{\xW^{2}} %M2AN
\newcommand{\xWinfty}{\xW^{\infty}} %M2AN
\newcommand{\xWn}[1]{\xW^{#1}} %M2AN

%-------------------------------------------------------------------------------
% @category		Norms
\newcommand{\norm}[1]{\lVert #1 \rVert}
\newcommand{\snorm}[1]{\lvert #1 \rvert}
\newcommand{\norminf}[1]{\norm{#1}_{\infty}}
\newcommand{\normstar}[1]{\norm{#1}_{\ast}}
\newcommand{\normStar}[1]{\norm{#1}^{\ast}}
\newcommand{\normz}[1]{\norm{#1}_z}
\newcommand{\norma}[1]{\norm{#1}_{\mathrm a}}
\newcommand{\normma}[1]{\norm{#1}_{\mathrm -a}}

%-------------------------------------------------------------------------------
% @category		Discrete Norms
\newcommand{\normdisc}[1]{\norm{#1}_{h,\sigma}}
\newcommand{\snormdisc}[2]{\snorm{#2}_{h,#1}}
\newcommand{\normh}[1]{\norm{#1}_{h}}
\newcommand{\snormh}[2]{\snorm{#2}_{h}}
% JCL-style
\newcommand{\normLdHmu}[1]{\|#1\|_{\xLtwo(0,T;\, \xHdisc^{-1})}}
\newcommand{\normLdHu}[1]{\|#1\|_{\xLtwo(0,T;\, \xHdisc^1)}}
\newcommand{\normLds}[1]{\|#1\|_{\xLtwo(0,T;\, {\mathrm H}_{\ast})}}
\newcommand{\normLdms}[1]{\|#1\|_{\xLtwo(0,T;\, {\mathrm H}^\ast)}}

%\newcommand{\InnerLtwo}[2]{\InnerP{#1}{#2}{0,\dom}}
\newcommand{\InnerLtwo}[2]{\InnerP{#1}{#2}{\xLtwo(\dom)}}
% discrete
\newcommand{\nLdisc}[2]{\norm{#2}_{h,#1}}
% norms
%\newcommand{\InnerHone}[2]{\InnerP{#1}{#2}{1,\dom}}
\newcommand{\InnerHone}[2]{\InnerP{#1}{#2}{\xHone(\dom)}}
\newcommand{\nHone}[1]{\norm{#1}_{\xHone(\dom)}}
\newcommand{\snHone}[1]{\snorm{#1}_{\xHone(\dom)}}
\newcommand{\nHtwo}[1]{\norm{#1}_{\xHtwo(\dom)}}
\newcommand{\snHtwo}[1]{\snorm{#1}_{\xHtwo(\dom)}}

\newcommand{\nxHone}[1]{\norm{#1}_{1}}
\newcommand{\snxHone}[1]{\snorm{#1}_{1}}

\newcommand{\nHoneb}[1]{\norm{#1}_{1,b}}
\newcommand{\nHonebK}[1]{\norm{#1}_{1,b,K}}
\newcommand{\snHoneb}[1]{\snorm{#1}_{1,b}}
% discrete
\newcommand{\nHmonedisc}[1]{\norm{#1}_{-1,\disc}}
\newcommand{\snHonedisc}[1]{\snorm{#1}_{1,\disc}}
\newcommand{\nHnb}[2]{\norm{#1}_{1,#2,b}}
%---------------------------------------
% @subcategory	Norms
\newcommand{\nWmp}[2]{\norm{#1}_{#2}}
% w-weighted Sobolev spaces
\newcommand{\nWmpw}[3]{\norm{#1}_{#2;#3}}
\newcommand{\snWmpw}[3]{\snorm{#1}_{#2;#3}}
%\newcommand{\nWmp}[2]{\norm{#1}_{#2,\dom}}
%\newcommand{\nWmpw}[3]{\norm{#1}_{#2,\dom,#3}}
%\newcommand{\snWmpw}[3]{\snorm{#1}_{#2,\dom;#3}}
\newcommand{\xE}{{\mathrm E}}



\bibliographystyle{plain}
\title{Project I -- Helmholtz equation in two dimensions}
\date{}

\begin{document}
\maketitle

The deadline for this assignment is 28. October 2018 and counts towards 15\% of the final grade.
The delivery consists of a report answering all the questions and presenting the results, together with a source code in MATLAB.
Working in pairs is possible but not compulsory.

\section{Problem formulation}

Let us consider the Helmholtz equation posed on $\dom = (0,1)^2$,
\begin{equation}\label{eq:helmholtz}
u(\xx) - \Delta u(\xx) = f(\xx), \qquad \forany \xx = (x,y)\in\dom
\end{equation}
with source term $f(\xx) = (2\pi^2 + 1)\cos(\pi x)\sin(\pi y)$, and supplemented with boundary conditions,
\begin{equation}\label{bc:helmholtz}
\begin{array}{lcccl}
u(\xx)              &=& u_D &,&\quad \xx\in\Gamma_D = \Set{\xx\in\bound : y \in \Fam{0,1}}\\
\Grad u\xDot\n(\xx) &=& 0   &,&\quad \xx\in\Gamma_N = \Set{\xx\in\bound : x \in \Fam{0,1}}\\
\end{array}
\end{equation}
with $\n$ the outward normal to the boundary $\bound$.

\medskip
The goal of this assignment is to write an algorithm to compute approximate solutions to Problem \eqref{eq:helmholtz}--\eqref{bc:helmholtz} using linear Lagrange Finite Elements on a triangular mesh $\meshT$.

\medskip
\begin{tmatsks}
\item Verify that $\tilde u: (x,y) \mapsto \cos(\pi x)\sin(\pi y)$ is solution to \eqref{eq:helmholtz}--\eqref{bc:helmholtz}.
\item Derive a weak formulation of \eqref{eq:helmholtz}--\eqref{bc:helmholtz}, specify the function spaces.
\item Is the solution $\tilde u$ unique?
\end{tmatsks}

\section{Finite Element space}

The construction of the approximation space based on Lagrange $\xPone$ Finite Elements is now considered.

\medskip
\begin{tmatsks}
\item Give the definition of the Lagrange $\xPone$ reference element on the unit triangle $\hat K$ with vertices $\Fam{ \hat\vertex_0 = (0,0), \hat\vertex_1 = (1,0), \hat\vertex_2 = (0,1)}$, and associated local shape functions $(\hat\varphi_0, \hat\varphi_1, \hat\varphi_2)$, then implement it
\item Write a simple test showing that shape functions $(\hat\varphi_0, \hat\varphi_1, \hat\varphi_2)$ form a nodal basis, and that for any $\hat\xx\in\hat K$, $\hat\varphi_0(\hat\xx) + \hat\varphi_1(\hat\xx) + \hat\varphi_2(\hat\xx) = 1$.
\item Implement the affine mapping $T_K: \hat K \rightarrow K$ and verify that the determinant of the Jacobian $\matJ_{T_K}$ is positive for triangle $K_0 = \Fam{ (1,0), (3,1), (3,2)}$. Interpret this result.
\item Implement the inverse of $\matJ_{T_K}$ and verify that the Finite Element obtained by transporting the reference Finite Element $\RefFE$ to $K_0$ is equivalent to $\RefFE$.
\item Formulate the Galerkin problem corresponding to the weak formulation derived at the preceding section, in particular define the approximation space carefully.
\end{tmatsks}

\section{Numerical integration}

Unless efficient exact evaluation is possible, computation of integrals is performed using quadrature rules.
Such approximations are expressed as the weighted sum of integrand values over $N_{q}$ \textit{quadrature points} $\sFam{\QRn{q}}$,
\[
I_K = \int_K \psi(\xx) \md \xx \approx \sum_{q = 1}^{N_q} \psi(\QRn{q})\;\omega_q
\]
with real coefficients $\sFam{\omega_q}$ called \textit{quadrature weights}.
The order $k_q$ of a quadrature rule is the polynomial degree of the integrand for which the evaluation is exact.
In particular, Gauss--Legendre quadratures on a real interval gathered in Table \ref{table:gauss_legendre_1d} satisfy the relation $k_q = 2N_q - 1$.
Quadrature rules can be defined using other polynomials and considering higher dimensions in space.
In the frame of Finite Elements, contributions on the reference simplex $\hat K$ can be written as
\[
\int_\RefK \hat\psi(\hxx) \md \hxx \approx \sum_{q = 1}^{N_q} \hat\psi(\QRn{q})\;\hat\omega_q
\]
with $q$ the index of the quadrature point.
Therefore any contribution on cell $K \in \meshT$ is obtained directly by composition with the affine change of coordinates $T_K$,
\[
\int_K \psi(\xx) \md \xx \approx |\det(J_{T_K})|\sum_{q = 1}^{N_q} \Comp{\psi}{T_K}(\QRn{q})\;\hat\omega_q
\]
with $J_{T_K}$ elementwise constant.
If $\psi$ involves derivatives, the change of variable should take into account that $(\Comp{f}{g})' = (\Comp{f'}{g})\cdot g'$.

\begin{table}
\label{table:gauss_legendre_1d}
\begin{tmatable}{llcc}
$k_q$ & $N_q$ & $\sFam{\rQRn{q}}$ & $\sFam{\rQRw{q}}$\\
\hline
\hline
1     & 1     & $\bar\zeta$                                   & $|I|$             \\
\hline
3     & 2     & $\bar\zeta\pm|I|\frac{\sqrt{3}}{6}$ & $\frac{1}{2}|I|$  \\
\hline
5     & 3     & $\bar\zeta\pm|I|\frac{\sqrt{15}}{10}$ & $\frac{5}{18}|I|$ \\
      &       & $\bar\zeta$                                   & $\frac{8}{18}|I|$ \\
\hline
7     & 4     & $\bar\zeta\pm|I|\frac{\sqrt{525+70\sqrt{30}}}{70}$ & $\frac{18-\sqrt{30}}{36}|I|$ \\
      &       & $\bar\zeta\pm|I|\frac{\sqrt{525-70\sqrt{30}}}{70}$ & $\frac{18+\sqrt{30}}{36}|I|$ \\
\hline
\end{tmatable}
\caption{Gauss--Legendre quadratures on the interval $[a,b]$ with $\bar\zeta = (a+b)/2$, and $|I| = |b -a|$}
\end{table}



\begin{table}
\label{table:gauss_legendre_2d}
\begin{tmatable}{llcc}
$k_q$ & $N_q$ & $\sFam{\rQRn{q}}$ & $\sFam{\rQRw{q}}$\\
\hline
\hline
1     & 1     & $\Coords{\frac{1}{3}, \frac{1}{3}, \frac{1}{3}}$ & $|K|$ \\
\hline
2     & 3     & $\Coords{\frac{1}{2}, \frac{1}{2}, 0}_{3}$ & $\frac{1}{3}|K|$  \\
\hline
3     & 4     & $\Coords{\frac{1}{3}, \frac{1}{3}, \frac{1}{3}}$ & $\frac{-9}{16}|K|$  \\
      &       & $\Coords{\frac{1}{5}, \frac{1}{5}, \frac{3}{5}}_{3}$   & $\frac{25}{48}|K|$ \\
\hline
4     & 7     & $\Coords{\frac{1}{3}, \frac{1}{3}, \frac{1}{3}}$ & $\frac{9}{40}|K|$  \\
      &       & $\Coords{a_i, a_i, 1 - 2a_i}_{3}$  &  $\frac{155 \pm \sqrt{15}}{1200}|K|$ \\
      &       & $a_i = \frac{6 \pm \sqrt{15}}{21}$  &  \\
\hline
\end{tmatable}
\caption{Gauss--Legendre quadratures on a triangle $K$ in barycentric coordinates $\sFam{\rQRn{q}} = (\lambda_0, \lambda_1, \lambda_2)$, with $(\xDot,\xDot,\xDot)_{k}$ the $k$ distinct tuples obtained by permutation, \cite{EG} page 360.}
\end{table}

%\begin{tmaalgo}{Integration by quadrature}
%\State $I_K = 0$
%\For{$q = 1:N_q$}
%\EndFor
%\end{tmaalgo}

\medskip
\begin{tmatsks}
\item Implement Gauss--Legendre quadratures from Table \ref{table:gauss_legendre_1d} and plot the approximation error for
\[
I = \int_1^2 e^x \md x
\]
\item Implement Gauss--Legendre quadratures from Table \ref{table:gauss_legendre_2d} and plot the approximation error for
\[
I = \int_{K_0} \log(x+y) \md x
\]
with $K_0 = \Fam{ (1,0), (3,1), (3,2)}$.
\item Discuss why the choice of quadrature is important for the evaluation of Finite Element contributions. Which properties of the problem should be considered for terms corresponding to the left-hand side and right-hand side of the equation?
\end{tmatsks}

\section{Assembly of the linear system}

For each cell $K \in \meshT$, elementwise contributions for the Helmholtz equation are under the form of a sum of two submatrices, corresponding to contributions of the mass matrix
\[
\matM_K =  \left[\int_K \varphi_j(\xx) \varphi_i(\xx) \md\xx\right]_{i j}
\]
and of the stiffness matrix
\[
\matK_K = \left[\int_K \nabla\varphi_j(\hxx) \nabla\varphi_i(\xx) \md\xx\right]_{ij}
\]
with $j$ indices of the \textit{global shape functions} (solution space), and $i$ indices of the \textit{global basis functions} (test space) with support on $K$.
For any $K\in\meshT$ assembling the local equation consists of computing the contributions for indices $\hat j = 1,\cdots,N_\SpaceP$  of the \textit{local shape functions} (solution space), and $\hat i = 1,\cdots,N_\SpaceP$ indices of the \textit{local basis functions} (test space) with support on $K$, $N_\SpaceP$ the dimension of the Finite Element.
The passage from one to another is performed with a mapping from (cell) local indices $(\hat i, \hat j)$ to (mesh) global indices $(i, j)$.
The obtained submatrix and subvector are then added to the global matrix and load vector.
%\begin{tmaalgo}{Assembly of the local equation matrix}
%\State $K \in \meshT$
%\State $I_K = 0$
%\For{$j = 1:N_{\SpaceP}$}
%	\For{$i = 1:N_{\SpaceP}$}
%  \State $I_K +=  $
%	\EndFor
%\EndFor
%\end{tmaalgo}

\medskip
\begin{tmatsks}
\item Detail the assembly of the local matrix and the local vector for any $K\in\meshT$.
\item Describe the assembly of the Dirichlet and Neumann boundary conditions.
\end{tmatsks}

\section{Convergence analysis}

\medskip
\begin{tmatsks}
\item Implement the computation of the $\xLtwo$ error norm given by
\[
\norm{u - u_h}_{\xLtwo} = \left(\int_{\dom} |u(\xx) - u_h(\xx)|^2 \md\xx\right)^\frac{1}{2}
\]
Why should you be careful with the evaluation of the integral?
\item Solve the problem for different mesh sizes $\sizeT = 1/M$ with $M = 4,8,16$ and plot the $\xLtwo$ error norm with respect to the dimension of the problem.
\end{tmatsks}

\section{Extension to an evolution problem}

Let us consider the evolution problem,
\begin{equation}\label{eq:heat_equation}
\partial_t u(\xx,t) - \nu\Delta u(\xx,t) = f(\xx,t), \qquad \forany (\xx,t)\in\dom\times(0,T)
\end{equation}
with $u(\xx, 0) = u_0$ given initial data, and $\nu$ diffusivity.

%\begin{tmaalgo}{Time stepping}
%\State $\matA = 0$, $\vecb = 0$
%\For{$n = 1:N$}
%\EndFor
%\end{tmaalgo}

\medskip
\begin{tmatsks}
\item Describe how you would modify the algorithm developed for the Helmholtz problem to solve this equation for a given discretization in time. For example use the Backward Euler scheme. The function $\tilde u(\xx,t) = e^{-\nu t} \sin\bigl(x \cos(\theta) + y\sin(\theta)\bigr)$ can be used to verify the implementation for the homogeneous equation (optional); take $\nu = 1$ and $\theta = \pi/4$ for instance.
\end{tmatsks}


\bibliography{../main}

\end{document}

