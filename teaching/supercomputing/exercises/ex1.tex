\documentclass[onecolumn, oneside, a4paper, 11pt]{memoir}

\usepackage[utf8]{inputenc}
\usepackage[T1]{fontenc}

% Paths
\newcommand{\figs}{../figs}
\newcommand{\data}{../data}

% Fonts
\usepackage{newpxtext,newpxmath}
\renewcommand*\sfdefault{cmss}

% Units
\usepackage[detect-weight=true, binary-units=true]{siunitx}
\DeclareSIUnit\flop{Flops}

% Math
\let\openbox\undefined
\usepackage{amsthm}
\usepackage{amsmath}
\usepackage{amssymb}
\usepackage{bm}

\theoremstyle{remark}
\newtheorem{ex}{Exercise}
\newtheorem*{sol}{Solution}

% Graphics
\usepackage{graphicx}
\usepackage{caption}
\usepackage{subcaption}
\graphicspath{{../figs/}}

% Tikz
\usepackage{tikz}
\usetikzlibrary{positioning,shapes,arrows,calc,intersections}
\usepackage{pgfplots}
\usepgfplotslibrary{dateplot}
\pgfplotsset{compat=1.8}

% Colors
\definecolor{darkblue}{HTML}{00688B}
\definecolor{darkgreen}{HTML}{6E8B3D}
\definecolor{cadet}{HTML}{DAE1FF}
\definecolor{salmon}{HTML}{FFB08A}

% Listings
\usepackage{textcomp}
\usepackage{listings}
\lstset{
  keywordstyle=\bfseries\color{orange},
  stringstyle=\color{darkblue!80},
  commentstyle=\color{darkblue!80},
  showstringspaces=false,
  basicstyle=\ttfamily,
  upquote=true,
}
\lstdefinestyle{fortran}{
  language=Fortran,
  morekeywords={for},
  deletekeywords={status},
}
\lstdefinestyle{c}{
  language=C,
  morekeywords={include},
}
\lstdefinestyle{shell}{
  language=bash,
}

\begin{document}

\pagestyle{empty}

\begin{center}
  {\Huge \bfseries \scshape
    Introduction to \\[0.2\baselineskip] Supercomputing} \\[2\baselineskip]
  {\Large TMA4280 $\cdot$ Problem set 1} \\[2\baselineskip]
\end{center}

\begin{ex}
  Exercise 1.1 in the lecture notes.
\end{ex}
\begin{sol}
  We found that single precision numbers have about 7 significant (decimal)
  digits, since they use a 23-bit mantissa, and $2^{-23} \approx 10^{-7}$.
\end{sol}

\begin{ex}
  Exercise 1.2 in the lecture notes.
\end{ex}
\begin{sol}
  \begin{align*}
    4.25 &= 4 + \frac{1}{4} \\
         &= 1 \cdot 2^2 + 0 \cdot 2^1 + 0 \cdot 2^0 + 0 \cdot 2^{-1} + 1 \cdot 2^{-2} \\
         &= 100.01_2 \\
         &= 1.0001_2 \cdot 2^2
  \end{align*}
\end{sol}

\begin{ex}
  Exercise 1.3 in the lecture notes.
\end{ex}
\begin{sol}
  This was computed in the lecture. A double precision floating point number has
  a 52 binary digit mantiassa, so its accuracy is $2^{-52} \approx 10^{-16}$.
  Therefore, about 16 decimal digits.
\end{sol}

\begin{ex}
  Exercise 1.4 in the lecture notes.
\end{ex}
\begin{sol}
  Some options:
  \begin{itemize}
  \item Use nested loops, where none of the loops go higher than about $10^9$
    iterations.
  \item Use a larger datatype. E.g. in C, \texttt{long} (at least 32 bits) or
    \texttt{long long} (at least 64). Note that a C \texttt{int} may be as low
    as 16 bits.
  \item Use an unsigned datatype. This gives us twice the range since the sign
    bit isn't needed.
  \end{itemize}
\end{sol}

\begin{ex}
  Exercise 1.5 in the lecture notes.
\end{ex}
\begin{sol}
  The first case needs $n$ additions and multiplications, so $2n =
  \mathcal{O}(n)$ operations total.

  For the second case, each element of $y$ requires $n$ multiplications and
  $n-1$ additions to form. Therefore the total number of operations should be
  $n(n + n - 1) = n(2n-1) = \mathcal{O}(n^2)$.
\end{sol}

\begin{ex}
  Exercise 1.6 in the lecture notes.
\end{ex}
\begin{sol}
  The matrix requires $n^2$ numbers to store, and each of the vectors requires
  $n$. We assume double precision floating point numbers ($\SI{8}{\byte}$ per
  number). Let us also assume that $n$ is large, so that the total memory
  requirement will be approximately $n^2$ numbers.
  \[
    8n^2 = 10^9 \quad\Longrightarrow\quad n \approx 11180.
  \]
  Therefore, only about $10^4$ unknowns can fit in this memory.
\end{sol}

\begin{ex}
  In the lecture we found that adding a small number to a large number can cause
  problems when the relative difference between the numbers exceed the accuracy
  of the floating point representation.

  With this in mind, suggest an algorithm for summing a list of numbers that is
  more accurate than doing it ``naively''.
\end{ex}
\begin{sol}
  The simplest solution is to sort the numbers before summing them in order from
  smallest to largest. However, this approach has complexity $\mathcal{O}(n\log
  n)$, which is worse than naive summation.

  The \emph{Kahan} algorithm keeps track of the lost bits and tries to add them
  on the following iterations. For each element $s_i$ in the input array, the
  sum is updated as
  \[
    S_i = S_{i-1} + \underbrace{s_i + c_{i-1}}_{y_i},
  \]
  where $S_i$ is the sum of the first $i$ elements, and $c_{i-1}$ is the
  compensation from the first $i-1$ elements. However, since the sum $S_{i-1}$
  is large, and $y_i$ is small, the lower bits may be lost. These lower bits are
  computed and used as compensation for the next iteration:
  \[
    c_i = y_i - \underbrace{(S_i - S_{i-1})}_{\text{high bits of $y_i$}}.
  \]
  Subtracting the high bits of $y_i$ from $y_i$ should leave just the low bits,
  which were exactly what went missing.

  Note that algebraically (in infinite precision arithmetic), $c_i$ should
  always be zero.
\end{sol}

\begin{ex}
  Implement a C or Fortran program that calculates $\bm y = \bm A \bm x$.
  \begin{align*}
    \bm A =
    \begin{pmatrix}
      0.3 & 0.4 & 0.3 \\
      0.7 & 0.1 & 0.2 \\
      0.5 & 0.5 & 0.0
    \end{pmatrix}, \qquad \bm x = \begin{pmatrix} 1.0 \\ 1.0 \\ 1.0 \end{pmatrix}.
  \end{align*}
\end{ex}
\newpage
\begin{sol}
  In C:

  \begin{lstlisting}[style=c]
#include <stdio.h>

const double A[][3] = {{0.3, 0.4, 0.3},
                       {0.7, 0.1, 0.2},
                       {0.5, 0.5, 0.0}};
const double x[3] = {1.0, 1.0, 1.0};

int main(int argc, char **argv)
{
  double y[3];
  for (int i = 0; i < 3; i++) {
    y[i] = 0.0;
    for (int j = 0; j < 3; j++)
      y[i] += A[i][j] * x[j];
  }
  printf("y = %f %f %f\n", y[0], y[1], y[2]);
  return 0;
}
  \end{lstlisting}
\end{sol}

\end{document}
