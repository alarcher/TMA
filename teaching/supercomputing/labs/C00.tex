\documentclass[onecolumn, oneside, a4paper, 11pt]{memoir}


\usepackage[utf8]{inputenc}
\usepackage[T1]{fontenc}

% Paths
\newcommand{\figs}{../figs}
\newcommand{\data}{../data}

% Fonts
\usepackage{newpxtext,newpxmath}
\renewcommand*\sfdefault{cmss}

% Units
\usepackage[detect-weight=true, binary-units=true]{siunitx}
\DeclareSIUnit\flop{Flops}

% Math
\let\openbox\undefined
\usepackage{amsthm}
\usepackage{amsmath}
\usepackage{amssymb}
\usepackage{bm}

\theoremstyle{remark}
\newtheorem{ex}{Exercise}
\newtheorem*{sol}{Solution}

% Graphics
\usepackage{graphicx}
\usepackage{caption}
\usepackage{subcaption}
\graphicspath{{../figs/}}

% Tikz
\usepackage{tikz}
\usetikzlibrary{positioning,shapes,arrows,calc,intersections}
\usepackage{pgfplots}
\usepgfplotslibrary{dateplot}
\pgfplotsset{compat=1.8}

% Colors
\definecolor{darkblue}{HTML}{00688B}
\definecolor{darkgreen}{HTML}{6E8B3D}
\definecolor{cadet}{HTML}{DAE1FF}
\definecolor{salmon}{HTML}{FFB08A}

% Listings
\usepackage{textcomp}
\usepackage{listings}
\lstset{
  keywordstyle=\bfseries\color{orange},
  stringstyle=\color{darkblue!80},
  commentstyle=\color{darkblue!80},
  showstringspaces=false,
  basicstyle=\ttfamily,
  upquote=true,
}
\lstdefinestyle{fortran}{
  language=Fortran,
  morekeywords={for},
  deletekeywords={status},
}
\lstdefinestyle{c}{
  language=C,
  morekeywords={include},
}
\lstdefinestyle{shell}{
  language=bash,
}

\begin{document}

\pagestyle{empty}

\begin{center}
  {\Huge \bfseries \scshape
    Introduction to \\[0.2\baselineskip] Supercomputing} \\[2\baselineskip]
  {\Large TMA4280 $\cdot$ Introduction to UNIX environment and tools} \\[2\baselineskip]
\end{center}

\section{Getting started with the environment and the \texttt{bash} shell interpreter}


Desktop computers are usually operated from a  \textit{graphical user interface} (GUI), but they have historically been interacted with by using a command interpreter: a sequence of commands is enter in a \textit{terminal}.
Since supercomputers do not run a graphical interface,
As of November 2017, 100\% of the supercomputers listed on the Top500 run a flavour of GNU/Linux system and they are exclusively operated without a GUI.
The most common interactive command interpreter, a \textit{shell}, is \texttt{bash}, the Bourne Again SHell. Other \textit{shell} programs exist and offer their own scripting language: \textit{csh}, \textit{tcsh}, \textit{ksh}, \textit{zsh}\dots

The \textit{shell} can be used interactively in a terminal and to run scripts: it is a very powerful tool which allows the user to interact with all the aspects of the system and perform tasks automatically.

\medskip
This section is a basic introduction to the \texttt{UNIX} environment which covers:
\begin{itemize}
\item getting information from the system,
\item navigating in directories,
\item manipulating files,
\item and processing text.
\end{itemize}

\medskip
At first learning \texttt{shell} can seem difficult and may have a steep learning curve, but in the long term it provides an elegant tool to perform tasks in a very efficient way. 

\subsubsection{Some aspects of the interaction with the \texttt{bash} shell}

The computer is operated by entering commands at the \textit{prompt}: it is called a \textit{command line}.
By convention the \textit{prompt} can be set to either:
\begin{enumerate}
\item $\$$ for a normal user,
\item $\#$ for an administrator.
\end{enumerate}

\medskip
The behaviour of commands can be modified with flags,and arguments can be provided.
For example, the following line can be entered:
\begin{lstlisting}[style=shell]
$ <command> -f0 -f1 <argument0> <argument1>
\end{lstlisting}
to call the command with two flags \texttt{-f0}, \texttt{-f1}, and two arguments.

\bigskip
Interacting with the \textit{bash} command line consists of:
\begin{itemize}
\item typing sequences of commands and arguments on the \textit{command line},
\item auto-completion of arguments with the \texttt{TAB} key to expand the name of a command or file,
\item navigation in the history of commands entered can be done with \texttt{UP}/\texttt{DOWN} arrow keys, or searched with \texttt{CTRL-R},
\item built-in functions that can be used for scripting.
\end{itemize}

\bigskip
All commands are documented and the manual pages can be displayed using the tools listed in Table \ref{tab:doc}.
\begin{table}[h!]
\centering
\caption{Basic commands: documentation}
\label{tab:doc}
\begin{tabular}{|c|l|}
  \hline
  Command & Description \\
  \hline
  \texttt{info} & Display manual pages in the \texttt{GNU Texinfo} format. \\
  \texttt{man}  & Display \textbf{man}ual pages in the \texttt{troff} format. \\
  \hline
\end{tabular}
\end{table}

The \texttt{man} utility is definitely the most important command on the system.


\subsubsection{Basic commands to get information on the system}

The \textit{command line} offers several tools to get various information of the computer and the users.
Before using commands that perform actions, it is a good idea to start with a few commands that will help you understand the environment.

\medskip
Open a terminal and look at the information provided by the \textit{prompt}, for example:
\begin{lstlisting}[style=shell]
[aurelila@lille-login2 ~]$
\end{lstlisting}
and let us discuss the information presented.
The first concept to understand is that:
\begin{itemize}
\item you are identified by a \textit{user} on a system identified by his \textit{hostname},
\item this \textit{user} belongs to one or more \textit{groups},
\item which define the \textit{permissions} you possess on the system. 
\end{itemize}

\bigskip
\begin{table}[h!]
\centering
\caption{Basic commands: information}
\label{tab:inf}
\begin{tabular}{|c|l|}
  \hline
  Command & Description \\
  \hline
  \texttt{df} & Display status of \textbf{d}isk space on \textbf{f}ile systems. \\
  \texttt{du} & Summarize \textbf{d}isk \textbf{u}sage. \\
  \texttt{env} & Print \textbf{env}ironment variables. \\
  \texttt{groups} &  Print \textbf{group} membership of user.\\
  \texttt{hostname} &  Set or print \textbf{name} of current \textbf{host} system.\\
  \texttt{top} & Display system usage information. \\
  \texttt{uname} & Print \textbf{name} of current system. \\
  \texttt{uptime} & Show how long the system has been \textbf{up}. \\
  \texttt{which} &  Which - locate a command and display its pathname or alias. \\
  \texttt{who} &  Who is logged on the system. \\
  \texttt{whoami} &  Print effective user identity. \\
  \hline
\end{tabular}
\end{table}

\begin{ex}
Before starting to navigate, let us inspect the work environment using the commands listed in Table \ref{tab:inf}.

Open a text editor and create an empty document that you will use to describe your steps and copy sample outputs from the terminal.
From the graphical interface save the file as \texttt{unix.txt}.
\begin{enumerate}
\item Open a terminal and inspect information about your user and the system:
\begin{itemize}
\item Print your user name, the machine name (\texttt{hostname}).
\item Use the manual page of \texttt{uname} to find how to print the machine name.
\item Find out to which groups your user belongs to.
\end{itemize}
\item Inspect the environment variables:
\begin{itemize}
\item Display the environment variables and note the value of \texttt{SHELL}, \texttt{PATH} and \texttt{HOME}.
\item For each of these variables, type \texttt{echo \$VARIABLE}, this is how to return the value of an environment variable.
\item Print the current working directory and observe that it is your home directory, i.e the value returned.
\end{itemize}
\item Find out more about resources:
\begin{itemize}
\item Display the time elapsed since the machine was booted and inspect the resources (memory, processes, disk space).
\item Look at the \texttt{du} manual page and determine the current disk usage of your home directory.
\end{itemize}
\end{enumerate}
\end{ex}

\subsubsection{Permissions}

A set of \textit{permissions} can be applied to files: in the UNIX philosophy, everything is represented as a file (text files, directories, devices).
Any file is attributed to a \textit{user} and a \textit{group}.

\medskip
The set is composed of three subsets describing permissions for:
\begin{enumerate}
\item \textit{user} (\textbf{u})
\item \textit{group} (\textbf{g})
\item \textit{others} (\textbf{o})
\end{enumerate}
and is encoded using an octal triplet.
For example, the triplet \texttt{644}, will give read/write permissions to the \textit{user}, and only read permissions to the \textit{group} and \textit{others}.

\medskip
\begin{table}[h!]
\centering
\caption{Permissions}
\begin{tabular}{|c|l|c|}
  \hline
  Type       & Description            & Octal value\\
  \hline
  \textbf{r} & \textbf{R}ead from the file.& 4 \\
  \textbf{w} & \textbf{W}rite to the file. & 2 \\
  \textbf{x} & E\textbf{x}ecute the file. &  1 \\
  \hline
\end{tabular}
\end{table}

\begin{table}[h!]
\centering
\caption{Basic commands: permissions}
\begin{tabular}{|l|c|r|}
  \hline
  Command & Description \\
  \hline
  \texttt{chmod} & Change the permission set. \\
  \texttt{chown} & Change the user ownership. \\
  \texttt{chgrp} & Change the group ownership. \\
  \hline
\end{tabular}
\end{table}


\begin{ex}
Use the command \texttt{ls -lha} to list the files in the current working directory and understand the information printed.
\end{ex}


\subsubsection{Basic commands to navigate on the system}

The set of commands essential to navigate in a \textit{terminal} is small but provides many possibilities.

\begin{table}[h!]
\centering
\caption{Basic commands: navigation}
\begin{tabular}{|c|l|}
  \hline
  Command & Description \\
  \hline
  \texttt{cd} & \textbf{C}hange working \textbf{d}irectory. \\
  \texttt{cp} & \textbf{C}o\textbf{p}y files. \\
  \texttt{find} & \textbf{Find} files in the directory hierarchy. \\
  \texttt{ls} & \textbf{L}i\textbf{s}t contents of directory. \\
  \texttt{mkdir} & \textbf{M}a\textbf{k}e \textbf{dir}ectories \\
  \texttt{mv} & \textbf{M}o\textbf{v}e files. \\
  \texttt{pwd} & \textbf{P}rint \textbf{w}orking \textbf{d}irectory name. \\
  \texttt{rm, rmdir} & \textbf{R}e\textbf{m}ove (directory) entries. \\
  \texttt{touch, settime} & Change file access and modification times. \\
  \hline
\end{tabular}
\end{table}

\begin{ex}
Now that you are familiar with your system, let us navigate and create directories/files.
\begin{itemize}
\item Create a directory named \texttt{TMA4280} and move inside it.
\item Create a directory named \texttt{shell} and move inside it.
\item Find your \texttt{unix.txt} file and copy it to the current directory.
\item Move one level up (two options possible) and create a directory named \texttt{prog}.
\item Move one level up again and type the command \texttt{ls -lha}, inspect the different folders.
\item Open the manual pages of \texttt{chown}, \texttt{chgrp} and \texttt{chmod}: which one should you use to change the read permission of \textit{others}?
\item Create a \texttt{README} file using the command \texttt{touch}.
\end{itemize}
\end{ex}

\begin{table}[h!]
\centering
\caption{Basic commands: text utilities}
\begin{tabular}{|l|c|r|}
  \hline
  Command & Description \\
  \hline
  \texttt{awk} & Pattern scanning and processing language. \\
  \texttt{cat} & Concatenate and display files. \\
  \texttt{grep} & Search a file for a pattern. \\
  \texttt{more, less} & Display text content with pagination. \\
  \texttt{sed} & Stream editor for text manipulation. \\
  \hline
\end{tabular}
\end{table}

\begin{table}[h!]
\centering
\caption{Basic commands: pipes and redirections}
\begin{tabular}{|l|c|r|}
  \hline
  Command & Description \\
  \hline
  \texttt{|} & Pipe operator. \\
  \texttt{>} & Redirection. \\
  \texttt{<} & Redirection. \\
  \hline
\end{tabular}
\end{table}

\begin{ex}
Just introducing a few concept for text and file manipulation:
\begin{itemize}
\item Enter the \texttt{shell} directory and create a \texttt{cat} subdirectory.
\item Create three files named \texttt{1.txt}, \texttt{2.txt} and \texttt{3.txt} and write the text \texttt{foo}, \texttt{bar} and \texttt{braz} into them using \textit{echo}.
\item Concatenate the content of the three files into the file \texttt{all.txt}.
\item Display it in a pager.
\item Together let us use the different utilities to search and replace text in these files: substitute \texttt{braz} with \texttt{babar}.
\end{itemize}
\end{ex}

\end{document}
