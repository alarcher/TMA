\lstset{inputpath=code/openmp/}

\chapter{Shared memory machines}

\section{Introduction}

Thus far in the course we have considered programming parallel machines using a
distributed memory model, where several processes communicate through message
passing, using the MPI library. This has been the traditional programming model
used in the HPC community since distributed memory machines became popular
during the late eighties. One of the main benefits of this programming model is
that it can be used on all hardware, even machines that actually have a shared
memory architecture. However, it has one major drawback; parallelizing a code
typically requires substantial changes to the serial version.

In these notes we consider programming shared memory machines using a shared
memory programming model. In particular we consider parallelization using the
\emph{OpenMP} standard. In later years, technology have reached a point where
making a single processing core much faster is very challenging. In order to
keep up with Moore's law for the next decades, vendors have started integrating
several processing cores in one chip, even in processors targeted at desktop
computers. At the moment, dual core processors are the norm on the desktop, with
quad and hexacore processors being available in the high end market. All major
vendors have signaled that they expect 8-16 cores to be the norm within a few
years, and a few hundreds within a decade. This means that even for desktop
class programs, programmers need to start developing parallel programs to
utilize the available computing resources. The effect of this is a substantially
increased interest in developing programming tools which allows for code
parallization on shared memory archictures with little effort. While the
OpenMP standard, which originated in the late ninties, initially was
aimed at HPC applications, it has now been adapted much more broadly. The effort
to improve the standard for the desktop market has also largely benefited the
HPC community. The increased activity has resulted in two major revisions of the
standard in the last five years, making it quite extensive. The scope of this
document is not to give a thorough introduction to the whole API, but rather to
give an idea of the main principles, as well as what new challenges and benefits
programming using OpenMP offers compared to the traditional message
passing tools.

\section{The OpenMP programming model}

OpenMP is a C/Fortran language extension for programming shared memory parallel
machines. It is implemented and supported by most major vendors, including
Intel, AMD, IBM and Oracle (SUN). It is also available in open source compilers
such as GCC or Clang, as well as in professional versions of Microsoft Visual
Studio.

\begin{figure}
  \centering
  \begin{tikzpicture}
  \foreach \i in {0,...,2} {
    \draw[darkblue, very thick, -latex] (0,-\i) -- (9,-\i);
    \node[anchor=west] at (9,-\i) {Process \i};
  }
\end{tikzpicture}

  \caption{
    An illustration of the MPI programming model. We have several independent
    processes, and each of these processes have their own separate program
    flow.
  }
  \label{fig:mpi}
\end{figure}

It is built on the threading paradigm in combination with a parallel section
view of the code. This means that the main program flow happens on one processor
only, which is quite a difference from the MPI programming model. Consider
\autoref{fig:mpi}, which is an illustration of the MPI programming model. Here
we have several processes, and each of these processes have their own separate
program flow. That is, each process is a separate instance of the program.
Syncronization of these processes is typically handled implicitly by using
blocking communication calls, i.e. if you call \texttt{MPI\_Send} in one
process, this process halts until it receives a confirmation that the message
has been processed. Likewise, on the receiver end, the process blocks the
program flow in the \texttt{MPI\_Recv} call until it has received the expected
message.

\begin{figure}
  \centering
  \begin{tikzpicture}
  \draw[darkblue, very thick] (0,0) -- (9,0);
  \draw[darkblue, very thick] (3,1) -- (6,1) -- (7,0) -- (6,-1) -- (3,-1) -- (2,0) -- (3,1);
  \node[anchor=south] at (8,0) {\scriptsize Thread 0};
  \node[anchor=south] at (1,0) {\scriptsize Thread 0};
  \node[anchor=south] at (4.5,1) {\scriptsize Thread 0};
  \node[anchor=south] at (4.5,0) {\scriptsize Thread 1};
  \node[anchor=south] at (4.5,-1) {\scriptsize Thread 2};
\end{tikzpicture}

  \caption{
    An illustration of the \texttt{OpenMP} programming model. We only have a
    single process, hence the main program flow only happens on a single
    processor. In sections of the code that we have marked as parallel, the
    program forks into multiple threads which work independently. At the end of
    the parallel section, these threads join together again and the program flow
    is returned to the first processor.
  }
  \label{fig:openmp}
\end{figure}

In the OpenMP programming model, this is different. Consider
\autoref{fig:openmp} which is an illustration of this programming model. OpenMP
is based on a fork/join programming model, where we only have a single instance
of the program, i.e. the main program flow only happens on one processor. We
mark certain sections of the program as being suitable for parallelization. When
the program flow enters these sections of the code, the program \emph{forks}
into several threads which work independently of each other. At the end of the
code sections the threads \emph{join} with the thread running on processor 0 and
the program flow is returned to this single processor. Thus in some sense one
might say that the MPI programming model is embedded in the OpenMP model, but
only in those parts of the code we have marked as parallel sections. This is
only half the truth though. In MPI each process have their own \emph{private}
resources. Here however, this is not true. The threads within the parallel
sections all have access to the \emph{same} resources. This is crucial to keep
in mind when designating the parallel sections of the code. One of the more
common pitfalls is several threads trying to write to the same memory location,
often due to using shared buffers during the calculations. It is thus highly
recommended that you try to design your parallel sections in such a way that
each thread has its own separate working buffers.

\subsection{Critical section}
\label{sec:critsec}

If for some reason you cannot avoid several threads needing write access to the
same resources, your only choice is to construct a critical section in your code
to protect these resources. Consider \autoref{fig:cs}. A critical section of the
code is a section of the program in which only a single thread can be at any
point in time. We construct such sections of the code using a tool known as a
\emph{mutual exclusion lock}, commonly referred to as a \emph{mutex}. We make a
section of the code critical by embedding it in a lock/unlock procedure. Prior
to entering the critical section of the code, the thread requests a lock of the
mutex. If the mutex is open when this is requested, the mutex is locked, and
info about which thread has the key is recorded. This mutex is now locked until
the thread associated with the key requests an unlock. The thread then moves on
executing the critical section of the code. Upon completion of the critical
section the thread unlocks the mutex and continues doing whatever we have told
it to do next. Now, if a second thread attempts to lock the mutex while it still
is locked, the second thread will stall in the lock call until it is able to
obtain the key. Since the mutex only has a single key, it would only be able to
obtain this key after the mutex has been unlocked by the thread which is
currently holding it. Since this unlocking only happens after the critical
section has been executed, we can then guarantee that only a single thread is
within the critical section of the code at any time. We stress that this is
something you should only use if there is no way around it, since the use of a
mutex leads to a \emph{serialization} of the critical section code. This can be
catastrophic for the parallel performance of your code if a large part of the
computation time is spent within such critical sections. Since this is meant to
be a brief introduction, we will not discuss these issues further in the
following.

\begin{figure}
  \centering
  \begin{tikzpicture}[scale=0.5]
  \draw[very thick] (-6,4) -- (-3,1) -- (3,1) -- (6,4);
  \draw[very thick] (-6,-4) -- (-3,-1) -- (3,-1) -- (6,-4);
  \draw[darkblue, ->] (-2.5,0) -- (2.5,0);
  \foreach \i in {-2,...,2} {
    \draw[darkblue, ->] (-8,\i) -- (-5,\i);
    \draw[darkblue, ->] (5,\i) -- (8,\i);
  }
  \node[anchor=south] at (0,1) {\scriptsize Critical section};
\end{tikzpicture}

  \caption{
    Illustration of a critical section. The incoming arrows represents the
    threads. All the threads are running concurrently, until they need to enter
    the critical section. Since only one thread can be inside the critical
    section at any time, the threads which want to enter need to wait until they
    obtain the key to the mutual exclusion lock, and hence their turn to enter
    the code section.
  }
  \label{fig:cs}
\end{figure}

\section{How to use OpenMP}

The idea behind the OpenMP API is that we give the compiler instructions on
which sections of the code we want to be parallelized. This means that, in
contrast to MPI, which works with all compilers, OpenMP requires specific
support in the compiler. The compiler then handles work division between the
available number of threads. This is in stark contrast to MPI where work
division is something the programmer always have to decide up front, before
modifying the serial code accordingly.

These instructions to the compiler are known as \emph{pragma} commands. There
are mainly two classes of OpenMP pragmas. The first class of pragmas are those
which can be used in combination with loop constructs such as \emph{for} loops.
This is the most useful case in context of HPC, since the programs typically
consist of multiple loops which apply the same operation to large datasets.
Consider the serial snippet
\lstinputlisting[style=c]{serial-for.c}
or the equivalent in Fortran
\lstinputlisting[style=fortran]{serial-for.f}

For simplicity, we here assume that \texttt{DoSomething(i)} does not depend on
any global resources, such as temporary working buffers. Hence this loop is
highly suitable for parallelization using OpenMP. In addition we first assume
that \texttt{DoSomething(i)} has a constant cost. To divide this loop among
several threads, we simply do
\lstinputlisting[style=c]{openmp-for.c}
Here we have our first example of an OpenMP directive. The pragma can be broken
down into three parts.
\begin{description}
\item[\texttt{\#pragma omp}] all \texttt{OpenMP} directives start with this.
\item[\texttt{parallel for}] instructs the compiler that we want the following
  \texttt{for}-construct parallelized.
\item[\texttt{schedule(static)}] instructs the compiler to hand each thread
  approximately the same number of loop iterations up front. This is a good
  solution here since (we have assumed that) each call to
  \texttt{DoSomething(i)} has the same cost. Hence such a simple division will
  give good load balancing between the threads.
\end{description}
The ingredients in the Fortran version is the same, but the syntax is slightly
different.
\lstinputlisting[style=fortran]{openmp-for.f}
To avoid having to restate everything twice, we only give C examples in the
following.

We can also run into situations where each call to \texttt{DoSomething(i)} has a
different cost. One example where you can run into this scenario is if
\texttt{DoSomething(i)} consists of an iterative method such as conjugate
gradients. Each solution process can have a different solution time. In this
case, if we give each thread a fixed number of loop iterations up front, we end
up with poor load balancing between the threads. Fortunately OpenMP offers a
mechanism to handle these situations. We simply do
\lstinputlisting[style=c]{openmp-for-dynamic.c}
The only difference is the change of the schedule parameter in the pragma from
static to dynamic. We here instruct the compiler to use a dynamic workload
division between the threads. This means that we reserve one thread as a
bookkeeper/negotiator. Within this parallel section this thread has a simple
task; keep track of which loop iterations have been performed and hand out a new
one to a thread when it requests it. Initially each thread is given a single
loop iteration to perform. Once the thread finishes this, it asks the negotiator
thread for a new one. The threads keep doing this until all work has been
performed.

If \texttt{DoSomething(i)} is fairly costly, this works very well. However, in
some cases each DoSomething(i) may be rather cheap. In this case the cost of
asking the negotiator for a new loop iteration between every calculation may
dominate the actual computation time. OpenMP also offers a mechanism to try to
minimize this problem. Instead of having the negotiator hand out a single loop
iteration when a thread asks for more work, it can hand out loop iterations in
chunks. Consider
\lstinputlisting[style=c]{openmp-for-dynamic-chunk.c}
The second parameter in the schedule (5) is the \emph{chunk size}. This is the
number of loop iterations a thread is (at most) given when it requests more work
from the negotiator thread. Thus we can limit the number of times a thread has
to communicate with the negotiator, hopefully making this part of the process
less dominating.

OpenMP also offers a third scheduling mode, called \emph{guided}. Consider
\lstinputlisting[style=c]{openmp-for-guided-chunk.c}
This is essentially a variant of dynamic scheduling, where we start out with a
large chunk size. The chunks are then exponentially decreased until we reach a
minimum, as specified in the chunk size. The idea here is that allocating large
chunks initially is good for performance, since these will typically overlap
fairly well. However, when the number of loop iterations left is small, we may
end up in a situation where all the remaining loop iterations are allocated to a
single thread. This is bad for performance since the other threads would then be
left idle. By using progressively smaller chunks, the chance of this happening
is reduced.

The second class of OpenMP directives are not tied to loop constructs. Instead
they are to be used if we have sections of the code which are completely
independent of each other. Consider the snippet
\lstinputlisting[style=c]{serial-sections.c}
If we are certain that these jobs are independent of each other, we can tell the
compiler this fact, and ask for the different \emph{sections} to be executed in
parallel on several threads. We do
\lstinputlisting[style=c]{openmp-sections.c}
Here each section of the code would be performed in a separate thread, before
the program flow again returns to processor 0 once all sections have been
completed. This directive is not as useful as those used in combination with
loop constructs, in particular we cannot as easily utilize a large number of
threads. The reason for this is fairly straight forward. In this example we can
at most use three threads, since there are only three sections of code
specified. It is often hard to find a large number of independent code sections
to allow for a larger number of threads. Large, expensive loops, however, are
typically present in most codes.

\section{$\pi$ --- OpenMP style}
We have previously calculated $\pi$ in both serial and MPI codes. We can
certainly use OpenMP for this as well. In the original serial code we have a
loop
\lstinputlisting[style=c]{serial-integrate.c}
This is the loop where the main work happens, and is what we should focus our
effort on. In this case, it is embarassingly simple to parallelize the loop
since there are no dependencies between the loop iterations. We can simply hand
a number of iterations to each thread, and then sum up the results afterwards.
OpenMP makes this convenient through the reduction directive
\lstinputlisting[style=c]{openmp-integrate.c}

\section{Compiling and running an OpenMP application}

We have a source file called \emph{openmp.c}, and we want to compile this with
OpenMP directives enabled. On a standard Linux computer this can be achieved by
using the \texttt{-fopenmp} directive, i.e.
\begin{lstlisting}
  gcc -O3 -o openmp -fopenmp -c openmp.c
\end{lstlisting}
while using the Intel compiler as we do on Kongull or Vilje, the directive is
simply \texttt{-openmp}, i.e.
\begin{lstlisting}
  icc -O3 -o poisson -openmp -c poisson.c
\end{lstlisting}
As long as we do not use any OpenMP utility function calls, we can still compile
this code into a completely serial code, simply by removing the compiler
switches. The compiler then simply ignore the pragmas, which makes the code look
exactly as the serial code from its point of view. This is in stark contrast to
a MPI version of the code where we would have to do substantial changes which
makes the program dependent on the MPI libraries.

To run our program using for instance four threads we do
\begin{lstlisting}[style=c]
  OMP_NUM_THREADS=4 ./openmp 2048
\end{lstlisting}

\section{Final remarks}

Modern supercomputers typically consist of multiple SMP nodes interconnected in
a NUMA organization, see the lecture notes. Clusters also fall into the same
category, since even the commodity processors used here have several cores
integrated in their chips as discussed in the introduction. In particular, Vilje
is an example of such a machine. Here each SMP node consists of 16
hyper-threaded processor cores. This means that an application which is
parallelized using OpenMP can at most use 32 threads, although for floating
point dominated programs, running only one thread per physical core is adviced.
If more computing resources is needed, we have no choice but to resort to a
distributed memory approach using MPI. The same applies to Kongull, except here
each SMP has 12 cores, which are not hyperthreaded.

A very natural question to ask is whether or not the two approaches can be
combined. The answer to this is yes. In fact this approach often allows us the
best of both worlds. Fine-grain parallelism is often intricate to exploit using
a distributed memory model, while the convenience offered by the shared memory
model often makes it fairly trivial to express. In addition, it is not always
easy to say up front where exploiting fine-grained parallelism actually will
improve the performance of your program. Since the modifications to the program
using the OpenMP approach is minimal, we do not have to invest much effort just
to benchmark whether or not parallizing a particular loop improves performance.

Coarse-grain parallelism, however, is often fairly involved to exploit using a
shared memory model, in particular due to the complications involved in
protecting shared resources such as working buffers. This often lead to
excessive memory usage or serialization of substantial parts of the code through
usage of critical sections, see \autoref{sec:critsec}. Using a distributed
memory model, this problem is nonexistent. Each process have their own private
resources which are inaccessible from the other processes. Here expressing
coarse grain parallelism is often just a matter of adjusting the limits on some
loops, as well as adding the appropriate library calls for data exchanges
between the processes when such exchanges are needed.

Hence a program where we utilize a message passing based approach, i.e. MPI, to
express the coarse grain parallelism, while utilizing OpenP pragmas to express
the fine grain parallelism within each MPI process allows use to use each
approach for what they are best at, while avoiding their weak sides.

\section{Further reading}

You can find the official OpenMP homepage at \url{http://openmp.org}. This page
is a great resource for those who are interested in more details. In addition to
having the description of the standard, it also contains links to several books
on subject, as well as discussion forums where you can ask questions. If a
source of tutorials are to be suggested, we can recommend
\url{https://computing.llnl.gov/tutorials/openMP/}.

\textbf{Acknowledgements:} Stephan Diederich helped with proof reading and gave
some valuable input while this chapter was written. The chapter is written by
Arne Morten Kvarving. Your assistance was greatly appreciated.
