\input{preamble}

\title{The Message Passing Interface (MPI)}
\institute{NTNU, IMF}
\date{February 9. 2018}
%\author{Aurélien Larcher}

\maketitle


\begin{frame}
  \frametitle{Recap: Parallelism with MPI}

An MPI execution is started on a set of processes $\mathcal{P}$ with:
\begin{center}
\texttt{mpirun -n $N_P$ ./exe}
\end{center}
with $\texttt{$N_P$} = card(\mathcal{P})$ the number of processes.

\bigskip
A program is executed in parallel on each process:
\begin{center}
    \scalebox{0.8}{\input{\figs/tma4280/mpi-model}}
\end{center}
where each process in $\mathcal{P} = \lbrace P0, P1, P2 \rbrace$ has a access to data in its memory space: remote data exchanged through interconnect.

\medskip
An ordered set of processes defines a \textbf{Group} (\texttt{MPI\_Group}): the inital group (\texttt{WORLD}) consists of \textbf{all} processes.

\end{frame}

\begin{frame}
  \frametitle{Recap: Parallelism with MPI}

An MPI execution is enclosed between calls of function to initialize and finalize the subsystem:
\begin{center}
\texttt{MPI\_Init}, \dots $\lbrace$ Program body $\rbrace$ \dots, \texttt{MPI\_Finalize}.
\end{center}
\bigskip

Processes in a \textbf{Group} can communicate through a \textbf{Communicator} (\texttt{MPI\_Comm}) with attributes:
\begin{itemize}
\item rank: \texttt{MPI\_Comm\_rank(MPI\_COMM\_WORLD, \&rank);}
\item size: \texttt{MPI\_Comm\_size(MPI\_COMM\_WORLD, \&size);}
\end{itemize}

\bigskip
The default communicator for \textbf{all} processes is \texttt{MPI\_COMM\_WORLD}.

\bigskip
New \textbf{Groups} can be created on subsets of $\mathcal{P}$ to restrict the communication between selected processes.\\[2ex]
Example: Monte-Carlo simulation, structure implementation of adjacent exchange. 

\end{frame}


\begin{frame}
  \frametitle{Recap: Parallelism with MPI}

Process communicate using the \textbf{Message Passing} paradigm:
\begin{enumerate}
\item Envelope: how to process the message,
\item Body: data to exchange. 
\end{enumerate}

\bigskip
MPI routines are categorized depending on their relation:
\begin{center}
\begin{tabular}{|l|c|}
\hline
Point-to-Point & \textcolor{red}{one-to-one}\\
\hline
\hfill & \textcolor{red}{one-to-all}\\
Collective & \textcolor{red}{all-to-one}\\
\hfill & \textcolor{red}{all-to-all}\\
\hline
\end{tabular}
\end{center}
where \textbf{all} is defined by the \textbf{Communicator} used.

\bigskip
Point-to-Point operations are the base of \textbf{Message Passing}: \texttt{MPI\_Send}, \texttt{MPI\_Recv} to exchange data.

\medskip
Collective operations are implemented in terms of Point-to-Point operations but may be optimized.

\end{frame}

\begin{frame}
\frametitle{Message buffers}

\begin{center}
\texttt{\textcolor{red}{MPI\_Send}($\underbrace{buffer, count, datatype}_{\textcolor{blue}{data}}$,
    $\underbrace{dest, tag, comm}_{\textcolor{blue}{envelope}}$); }
\end{center}

\begin{tabular}{|c|l|}
\hline
\texttt{buffer}    & memory location defining the begining of the buffer\\
\texttt{count}     & max length in bytes of the send buffer (\textbf{bigger} than actual)\\
\texttt{dest} & receiving process rank\\
\texttt{tag}        & arbitrary identifier \\
\texttt{comm}        & communicator \\
\hline
\end{tabular}

\begin{center}
\texttt{\textcolor{red}{MPI\_Recv}($\underbrace{buffer, count, datatype}_{\textcolor{blue}{data}}$,
    $\underbrace{source, tag, comm,\&status}_{\textcolor{blue}{envelope}}$); }
\end{center}

\begin{tabular}{|c|l|}
\hline
\texttt{buffer}    & memory location defining the begining of the buffer\\
\texttt{count}  & maxlength in bytes of received data\\
\texttt{datatype}  & MPI data type\\
\texttt{source}& sending process rank\\
\texttt{tag}        & arbitrary identifier \\
\texttt{comm}        & communicator \\
\texttt{status}     & metadata: actual length \\
\hline
\end{tabular}

\end{frame}

\begin{frame}
  \frametitle{Recap: Parallelism with MPI}

Two types of MPI commmunication: block and non-blocking.
\begin{center}
    \scalebox{0.8}{\begin{tikzpicture}
  \node[anchor=south] at (1,-1) {send(x, 1)};
  \draw[darkblue, very thick] (0,-1) -- (9,-1);
  \node[anchor=west] at (9,-1) {Process 0};
  \draw[orange, ultra thick, -latex] (1,-1) -- (3,-3);
  \node[anchor=north] at (3,-3) {recv(y, 0)};
  \draw[darkblue, very thick] (0,-3) -- (9,-3);
  \node[anchor=west] at (9,-3) {Process 1};
  \draw[darkblue, very thick] (0,-5) -- (9,-5);
  \node[anchor=south] at (5.5,-3) {req = irecv(y, 2)};
  \node[anchor=south] at (8,-3) {wait(req)};
  \draw[orange, ultra thick, -latex] (5.5,-3) -- (5.5, -3.5) -- (8, -3.5) -- (8,-3);
  \draw[orange, ultra thick, -latex] (1,-1) -- (3,-3);
  \node[anchor=west] at (9,-5) {Process 2};
  \draw[orange, ultra thick, -latex] (5,-5) -- (5, -4.5) -- (7.5, -4.5) -- (7.5,-5);
  \draw[orange, ultra thick, -latex] (1,-1) -- (3,-3);
  \draw[orange, ultra thick, -latex] (6.25,-4.5) -- (6.75,-3.5);
  \node[anchor=north] at (5,-5) {req = isend(x, 1)};
  \node[anchor=north] at (7.5,-5) {wait(req)};
\end{tikzpicture}
}
\end{center}
\begin{enumerate}
\item blocking: the function returns once the buffer is processed (what ``processed'' (\textcolor{red}{$\ast$})
\item non-blocking: the function returns immediately
\end{enumerate}

\bigskip
MPI implementations use an internal buffer:
\begin{itemize}
\item data may be buffered internally on both sides,
\item sending process can modify the send buffer once it is copied to the internal buffe regardless of receiver status,
\item receiver process can buffer the data internally before creation of the recv buffer.
\end{itemize}

\end{frame}

\begin{frame}
  \frametitle{Recap: Parallelism with MPI}

(\textcolor{red}{$\ast$}) What ``processed'' means depends on the communication mode:
\begin{center}
\texttt{MPI\_Send}, \texttt{MPI\_SSend}, \texttt{MPI\_RSend}, \texttt{MPI\_BSend}
\end{center}.

\bigskip
MPI may provide several versions of a routine for different communication modes
\begin{itemize}
\item Synchronous: send blocks until handshake is done and matching receive has started.
\item Ready: send blocks until matching receive has started (no handshake).
\item Buffered: send returns as soon as data is buffered internally.
\item Standard: Synchronous or Buffered depending on message size and resources.
\end{itemize}

\textcolor{red}{Important}: do not assume anything about the implementation, different handling of buffers can hide bugs!
\end{frame}

\begin{frame}
  \frametitle{Parallelism}

Pairwise send--receive:
\begin{center}
    \scalebox{0.8}{
\begin{tikzpicture}
  \node[anchor=south] at (2,-1) {send(x1, 1)};
  \draw[darkblue, very thick] (0,-1) -- (9,-1);
  \node[anchor=west] at (9,-1) {Process 0};
  \node[anchor=south] at (2,-3) {send(x2, 2)};
  \draw[orange, ultra thick, -latex] (2,-1) -- (7,-3);
  \node[anchor=north] at (7,-3) {recv(y0, 0)};
  \draw[darkblue, very thick] (0,-3) -- (9,-3);
  \node[anchor=west] at (9,-3) {Process 1};
  \draw[darkblue, very thick] (0,-5) -- (9,-5);
  \node[anchor=west] at (9,-5) {Process 2};
  \node[anchor=north] at (7,-5) {recv(y1, 1)};
  \draw[orange, ultra thick, -latex] (2,-3) -- (7,-5);
\end{tikzpicture}
}

\bigskip
A safe choice for Point-to-Point communication and optimized by vendors for their hardware.
\end{center}

\end{frame}


\begin{frame}[fragile]
  \frametitle{Collective operations}
  In general,
  \begin{itemize}
  \item \textcolor{red}{MUST} involves all of the processes in a group, and
  \item are more efficient and less tedious to use compared to point-to-point
    communication.
  \end{itemize}

\begin{enumerate}
\item Barrier synchronization
\item Broadcast (one-to-all)
\item Scatter (one-to-all)
\item Gather (all-to-one)
\item Global reduction operations: min, max, sum (all-to-all)
\item All-to-all data exchange
\item Scan across all processes
\end{enumerate}

  An example (synchronization between processes):
  \begin{center}
    \begin{tabular}{c}
\begin{lstlisting}[style=c,morekeywords={MPI_Barrier}]
MPI_Barrier(comm);
\end{lstlisting}
    \end{tabular}
  \end{center}
\end{frame}

\begin{frame}
  \frametitle{Collective operations}
  \begin{center}
    \begin{tikzpicture}[scale=0.6]
      \foreach \i in {0,...,4} {
        \foreach \j in {0,9} {
          \foreach \k in {0,6} {
            \draw[darkblue, thick] (\i+\j,0+\k) -- (\i+\j,4+\k);
            \draw[darkblue, thick] (0+\j,\i+\k) -- (4+\j,\i+\k);
          }
        }
      }
      \node at (0.5,9.5) {$A_0$};
      \node at (9.5,9.5) {$A_0$};
      \node at (9.5,8.5) {$A_0$};
      \node at (9.5,7.5) {$A_0$};
      \node at (9.5,6.5) {$A_0$};

      \node at (0.5,3.5) {$A_0$};
      \node at (0.5,2.5) {$A_1$};
      \node at (0.5,1.5) {$A_2$};
      \node at (0.5,0.5) {$A_3$};
      \node at (9.5,3.5) {$A_0$};
      \node at (10.5,3.5) {$A_1$};
      \node at (11.5,3.5) {$A_2$};
      \node at (12.5,3.5) {$A_3$};
      \node[color=black!30] at (9.5,2.5) {$A_1$};
      \node[color=black!30] at (9.5,1.5) {$A_2$};
      \node[color=black!30] at (9.5,0.5) {$A_3$};

      \draw[red, very thick, ->] (4.5,8) -- (8.5,8);
      \draw[red, very thick, ->] (4.5,2) -- (8.5,2);
      \node[anchor=north] at (6.5,8) {\texttt{MPI\_Bcast}};
      \node[anchor=south] at (6.5,8) {\scriptsize One-to-all broadcast};
      \node[anchor=north] at (6.5,2) {\texttt{MPI\_Gather}};
      \node[anchor=south] at (6.5,2) {\scriptsize All-to-one gather};

      \node[anchor=west, color=darkblue] (data) at (0,10.5) {\scriptsize Data};
      \node[anchor=west, color=darkblue, rotate=-90] (procs) at (-0.5,10) {\scriptsize Processes};
      \draw[darkblue, ->] (data.east) -- ($ (data.east) + (1.5,0) $);
      \draw[darkblue, ->] (procs.east) -- ($ (procs.east) - (0,1.5) $);
    \end{tikzpicture}
  \end{center}
\end{frame}

\begin{frame}
  \frametitle{Global reduction}
  \begin{center}
    \begin{tikzpicture}[scale=0.6]
      \foreach \i in {0,4,8,12} {
        \foreach \j in {0,4,8} {
          \draw[darkblue, thick] (\i,\j) rectangle (\i+2,\j+1);
          \draw[darkblue, thin] (\i+1,\j) -- (\i+1,\j+1);
        }
      }

      \node[anchor=south] at (7,9) {Processes with initial data};
      \node[anchor=north] at (1,8) {\scriptsize $p=0$};
      \node[anchor=north] at (5,8) {\scriptsize $p=1$};
      \node[anchor=north] at (9,8) {\scriptsize $p=2$};
      \node[anchor=north] at (13,8) {\scriptsize $p=3$};
      \node at (0.5,8.5) {$2$}; \node at (1.5,8.5) {$4$};
      \node at (4.5,8.5) {$5$}; \node at (5.5,8.5) {$7$};
      \node at (8.5,8.5) {$0$}; \node at (9.5,8.5) {$3$};
      \node at (12.5,8.5) {$6$}; \node at (13.5,8.5) {$2$};

      \node[anchor=south] at (7,5) {After \texttt{MPI\_Reduce(..., MPI\_MIN, 0, ...)}};
      \node at (0.5,4.5) {$0$}; \node at (1.5,4.5) {$2$};
      \node at (4.5,4.5) {$-$}; \node at (5.5,4.5) {$-$};
      \node at (8.5,4.5) {$-$}; \node at (9.5,4.5) {$-$};
      \node at (12.5,4.5) {$-$}; \node at (13.5,4.5) {$-$};

      \node[anchor=south] at (7,1) {After \texttt{MPI\_Allreduce(..., MPI\_MIN, ...)}};
      \node at (0.5,0.5) {$0$}; \node at (1.5,0.5) {$2$};
      \node at (4.5,0.5) {$0$}; \node at (5.5,0.5) {$2$};
      \node at (8.5,0.5) {$0$}; \node at (9.5,0.5) {$2$};
      \node at (12.5,0.5) {$0$}; \node at (13.5,0.5) {$2$};
    \end{tikzpicture}
  \end{center}
\end{frame}

\begin{frame}
  \frametitle{Global reduction}
  \texttt{\textcolor{red}{MPI\_Reduce}($\underbrace{sbuf, rbuf, count, datatype}_{\textcolor{blue}{data}}$,
    $\underbrace{op, root, comm}_{\textcolor{blue}{envelope}}$); } \\
  \texttt{\textcolor{red}{MPI\_Allreduce}($\underbrace{sbuf, rbuf, count, datatype}_{\textcolor{blue}{data}}$,
    $\underbrace{op, comm}_{\textcolor{blue}{envelope}}$); }

  Examples of predefined operations (C):
  \begin{itemize}
  \item \texttt{MPI\_SUM}
  \item \texttt{MPI\_PROD}
  \item \texttt{MPI\_MIN}
  \item \texttt{MPI\_MAX}
  \end{itemize}
\end{frame}


\begin{frame}
  \frametitle{Numerical integration}
  \begin{center}
    \begin{tikzpicture}[scale=0.7]
      \begin{axis}[
        xmin=0,
        xmax=1.1,
        ymin=0,
        ymax=5,
        axis lines=middle,
        xlabel={$x$},
        ylabel={$f(x)$},
        xtick={0.001,0.4,1},
        xticklabels={$0$, $x_i$, $1$},
        ]
        \draw[darkblue, fill=cadet] (axis cs:0.36,0) rectangle (axis cs:0.44,3.4482);
        \addplot[darkblue, thick, domain=0:1, samples=100]{4/(1+x^2)};
      \end{axis}
    \end{tikzpicture}
  \end{center}
  \[
    A_i = \left ( \frac{4}{1+x_i^2} \right ) \cdot h, \qquad
    \text{with} \quad x_i = \left (i+\frac{1}{2} \right )\cdot h
  \]
  where $i=0,\ldots,n-1$, and $h=1/n$.
\end{frame}

\begin{frame}
  \frametitle{Numerical integration}
  \begin{align*}
    \pi = \int_0^1 \frac{4}{1 + x^2} \dif{x} \approx
    h \sum_{i=0}^{n-1} \frac{4}{1 + x_i^2} = \pi_n
  \end{align*}
\end{frame}

\begin{frame}[fragile]
  \frametitle{Calculating pi with MPI in C}
  \begin{center}
    \begin{tabular}{c}
      \scalebox{0.65}{
      \lstinputlisting
      [style=c, firstline=1, lastline=22, morekeywords={
      MPI_Init, MPI_Comm_size, MPI_Comm_rank, MPI_Wtime}]{\code/pi/pi.c}
      }
    \end{tabular}
  \end{center}
\end{frame}

\begin{frame}[fragile]
  \frametitle{Calculating pi with MPI in C (cntd.)}
  \begin{center}
    \begin{tabular}{c}
      \scalebox{0.65}{
      \lstinputlisting
      [style=c, firstline=24, lastline=45, morekeywords={
      MPI_Reduce, MPI_Finalize, MPI_Wtime}]{\code/pi/pi.c}
      }
    \end{tabular}
  \end{center}
\end{frame}

\begin{frame}
  \frametitle{Things to consider}
  \begin{enumerate}
  \item Is the program correct, e.g., is the convergence rate as expected?
  \item Is the program load-balanced?
  \item Do we get the same value of $\pi_n$ for different values of $P$?
  \item Is the program scalable?
  \end{enumerate}
\end{frame}

\begin{frame}
  \frametitle{Convergence test}
  \begin{center}
    \bgroup\def\arraystretch{1.2}
    \begin{tabular}{cc}
      \hline
      $n$ & $\text{error} = |\pi - \pi_n |$
      \\ \hhline{==} 10 & $8.33\cdot 10^{-4}$
      \\ \hline $10^2$ & $8.33\cdot 10^{-6}$
      \\ \hline $10^3$ & $8.33\cdot 10^{-8}$
      \\ \hline $10^4$ & $8.33\cdot 10^{-10}$
      \\ \hline $10^5$ & $8.37\cdot 10^{-12}$
      \\ \hline
    \end{tabular}
    \egroup
  \end{center}
  Hence, $|\pi - \pi_n | \sim {\cal O}(h^2)$ where $h = 1/n$.
\end{frame}

\begin{frame}
  \frametitle{Scalability: timing results on Vilje}
  \begin{center}
    \begin{tikzpicture}
      \begin{axis}[
        xmin=1,
        xmax=40,
        ymin=0,
        ymax=10,
        axis lines=middle,
        xlabel={{\footnotesize $P$}},
        ylabel={{\footnotesize Speedup, $T_1/T_P$}},
        xtick={1,2,4,8,16,32},
        xticklabels={1,2,4,8,16,32},
        xmode=log,
        legend style={
          at={(1,0.50)},
          draw=none,
        }
        ]
        \addplot[domain=1:32, dashed] {x};
        \addplot[only marks, draw=darkblue, fill=cadet, mark=square*]
        coordinates {(1,1) (2,1.083) (4,1.020) (8,0.782)};
        \addplot[only marks, draw=darkblue, fill=salmon, mark=*]
        coordinates {(1,1) (2,1.966) (4,3.729) (8,6.586) (16,9.548) (32,9.196)};
        \addplot[only marks, draw=darkblue, fill=salmon, mark=otimes*] coordinates {(32,6.791)};
        \legend{Ideal, $n=10^4$, $n=10^6$}
      \end{axis}
    \end{tikzpicture}
  \end{center}
\end{frame}

\begin{frame}
  \frametitle{Inner product}
  \begin{align*}
    \sigma = \bm x^\intercal \bm y = \bm x \cdot \bm y = \sum_{m=0}^{N-1} x_m y_m
  \end{align*}
  \begin{center}
    \begin{tikzpicture}[scale=0.7]
      \foreach \i in {0,...,9} {
        \node at (\i+0.5,2.5) {$x_\i$};
        \node at (\i+0.5,0.5) {$y_\i$};
        \draw[darkblue, thick] (\i,0) rectangle (\i+1,1);
        \draw[darkblue, thick] (\i,2) rectangle (\i+1,3);
      }
      \node at (-0.5,0.5) {$\bm y$};
      \node at (-0.5,2.5) {$\bm y$};
    \end{tikzpicture}
  \end{center}
\end{frame}

\begin{frame}
  \frametitle{Distribution of work}
  \begin{center}
    \begin{tikzpicture}[scale=0.45]
      \foreach \i in {0,...,2} {
        \draw[darkblue, fill=cadet, thick] (\i,0) rectangle (\i+1,1);
        \draw[darkblue, fill=cadet, thick] (\i,1.5) rectangle (\i+1,2.5);
      }
      \foreach \i in {3,...,9} {
        \draw[darkblue, thick] (\i,0) rectangle (\i+1,1);
        \draw[darkblue, thick] (\i,1.5) rectangle (\i+1,2.5);
      }
      \foreach \i in {0,...,9} {
        \node at (\i+0.5,2.0) {${\scriptstyle x_\i}$};
        \node at (\i+0.5,0.5) {${\scriptstyle y_\i}$};
      }
      \node[anchor=west] at (10.5,1.25) {$\omega_0 = \sum_{m=0}^2 x_m y_m$};
    \end{tikzpicture}
    \\~\\
    \begin{tikzpicture}[scale=0.45]
      \foreach \i in {3,4,5} {
        \draw[darkblue, fill=cadet, thick] (\i,0) rectangle (\i+1,1);
        \draw[darkblue, fill=cadet, thick] (\i,1.5) rectangle (\i+1,2.5);
      }
      \foreach \i in {0,1,2,6,7,8,9} {
        \draw[darkblue, thick] (\i,0) rectangle (\i+1,1);
        \draw[darkblue, thick] (\i,1.5) rectangle (\i+1,2.5);
      }
      \foreach \i in {0,...,9} {
        \node at (\i+0.5,2.0) {${\scriptstyle x_\i}$};
        \node at (\i+0.5,0.5) {${\scriptstyle y_\i}$};
      }
      \node[anchor=west] at (10.5,1.25) {$\omega_1 = \sum_{m=3}^5 x_m y_m$};
    \end{tikzpicture}
    \\~\\
    \begin{tikzpicture}[scale=0.45]
      \foreach \i in {6,7} {
        \draw[darkblue, fill=cadet, thick] (\i,0) rectangle (\i+1,1);
        \draw[darkblue, fill=cadet, thick] (\i,1.5) rectangle (\i+1,2.5);
      }
      \foreach \i in {0,1,2,3,4,5,8,9} {
        \draw[darkblue, thick] (\i,0) rectangle (\i+1,1);
        \draw[darkblue, thick] (\i,1.5) rectangle (\i+1,2.5);
      }
      \foreach \i in {0,...,9} {
        \node at (\i+0.5,2.0) {${\scriptstyle x_\i}$};
        \node at (\i+0.5,0.5) {${\scriptstyle y_\i}$};
      }
      \node[anchor=west] at (10.5,1.25) {$\omega_2 = \sum_{m=6}^7 x_m y_m$};
    \end{tikzpicture}
    \\~\\
    \begin{tikzpicture}[scale=0.45]
      \foreach \i in {8,9} {
        \draw[darkblue, fill=cadet, thick] (\i,0) rectangle (\i+1,1);
        \draw[darkblue, fill=cadet, thick] (\i,1.5) rectangle (\i+1,2.5);
      }
      \foreach \i in {0,...,7} {
        \draw[darkblue, thick] (\i,0) rectangle (\i+1,1);
        \draw[darkblue, thick] (\i,1.5) rectangle (\i+1,2.5);
      }
      \foreach \i in {0,...,9} {
        \node at (\i+0.5,2.0) {${\scriptstyle x_\i}$};
        \node at (\i+0.5,0.5) {${\scriptstyle y_\i}$};
      }
      \node[anchor=west] at (10.5,1.25) {$\omega_3 = \sum_{m=8}^9 x_m y_m$};
    \end{tikzpicture}
  \end{center}
\end{frame}

\begin{frame}
  \frametitle{Program on processor $p$}
  \begin{align*}
    \bm x, \bm y &: \text{ vectors of dimension } N \\
    \omega_p &= \sum_{m \in \mathcal{N}_p} x_m y_m \\
                 & \text{Send } \omega_p  \text{ to processor } q \not= p \\
                 & \text{Receive } \omega_q  \text{ from processor } q \\
    \sigma &= \sum_{q=0}^{P-1} \omega_q.
  \end{align*}
\end{frame}

\begin{frame}
  \frametitle{Local and global numbering}
  Global indices: $\mathcal{N} = \{0,1,2,\ldots,N-1\}$.
  Cardinality $|\mathcal{N}| = N$.

  Subdivision:
  \begin{align*}
    \mathcal{N} = \bigcup_{p=0}^{P-1} \mathcal{N}_p, \qquad
    \mathcal{N}_p \cap \mathcal{N}_q = \emptyset, \qquad
    p \not= q.
  \end{align*}

  \begin{itemize}
  \item $\mathcal{N}_p$: a subset of global indices
  \item $N_p = |\mathcal{N}_p|$: the number of global indices assigned to
    process $p$.
  \item $\mathcal{I}_p = \{0,1,\ldots,N_p-1\}$: a \emph{local} index set
  \end{itemize}

  Note $|\mathcal{I}_p| = |\mathcal{N}_p| = N_p$.
\end{frame}

\begin{frame}
  \frametitle{Local and global numbering}
  \begin{center}
    \begin{tikzpicture}
      \node[circle, thick, fill=cadet, draw=darkblue, minimum size=1cm] (g) at (-3,0) {$m$};
      \node[circle, thick, fill=cadet, draw=darkblue, minimum size=1cm] (l) at (3,0) {$p,i$};
      \draw[->, draw=darkblue, thick] (g) edge[bend left] node [above] {$\mu$} (l);
      \draw[->, draw=darkblue, thick] (l) edge[bend left] node [below] {$\mu^{-1}$} (g);
      \node at (-3,-1) {$m\in\mathcal{N}$};
      \node (p) at (3,-1) {$0 \leq p < P$};
      \node[anchor=north] at (p.south) {$i \in \mathcal{I}_p$};
    \end{tikzpicture}
  \end{center}
\end{frame}

\begin{frame}
  \frametitle{Local and global numbering}

Requirements for the numbering of entities:
\begin{itemize}
\item The local-to-global numbering is a one-to-one relationship.
\item Local indices are found in the interval $\mathcal{I}_p = \{0,1,\ldots,N_p-1\}$.
\item Global indices are not necessarily alway contiguous: this is the case when the global number of entities is not known.
\end{itemize}

\medskip
The entities are usually renumbered globally such that global indices are packed contiguously.

\medskip
Each process $p$ is then assigned a \textbf{range} $[ i^p_0, i^p_{N_p}  [$ with
\begin{enumerate}
\item $i^p_0$ is the process range \textit{offset},
\item $i^p_{N_p}$ is the \textit{local size}.
\end{enumerate}

\end{frame}

\begin{frame}
  \frametitle{Local and global numbering}

How to compute the offset with MPI?
\begin{enumerate}
\item MPI 1: \texttt{MPI\_Scan}
\item MPI 2: \texttt{MPI\_Exscan}
\end{enumerate}
\end{frame}

\begin{frame}
  \frametitle{Distribution of work and data}
  \begin{center}
    \begin{tikzpicture}[scale=0.45]
      \foreach \i in {0,1,2} {
        \draw[darkblue, fill=cadet, thick] (\i,0) rectangle (\i+1,1);
        \draw[darkblue, fill=cadet, thick] (\i,1.5) rectangle (\i+1,2.5);
        \node at (\i+0.5,2.0) {${\scriptstyle \hat{x}_\i}$};
        \node at (\i+0.5,0.5) {${\scriptstyle \hat{y}_\i}$};
      }
      \node[anchor=west] at (4.5,1.25) {$\omega_0 = \sum_{i=0}^2 \hat{x}_i \hat{y}_i$};
    \end{tikzpicture}
    \\~\\
    \begin{tikzpicture}[scale=0.45]
      \foreach \i in {0,1,2} {
        \draw[darkblue, fill=cadet, thick] (\i,0) rectangle (\i+1,1);
        \draw[darkblue, fill=cadet, thick] (\i,1.5) rectangle (\i+1,2.5);
        \node at (\i+0.5,2.0) {${\scriptstyle \hat{x}_\i}$};
        \node at (\i+0.5,0.5) {${\scriptstyle \hat{y}_\i}$};
      }
      \node[anchor=west] at (4.5,1.25) {$\omega_1 = \sum_{i=0}^2 \hat{x}_i \hat{y}_i$};
    \end{tikzpicture}
    \\~\\
    \begin{tikzpicture}[scale=0.45]
      \foreach \i in {0,1} {
        \draw[darkblue, fill=cadet, thick] (\i,0) rectangle (\i+1,1);
        \draw[darkblue, fill=cadet, thick] (\i,1.5) rectangle (\i+1,2.5);
        \node at (\i+0.5,2.0) {${\scriptstyle \hat{x}_\i}$};
        \node at (\i+0.5,0.5) {${\scriptstyle \hat{y}_\i}$};
      }
      \node[anchor=west] at (4.5,1.25) {$\omega_2 = \sum_{i=0}^1 \hat{x}_i \hat{y}_i$};
    \end{tikzpicture}
    \\~\\
    \begin{tikzpicture}[scale=0.45]
      \foreach \i in {0,1} {
        \draw[darkblue, fill=cadet, thick] (\i,0) rectangle (\i+1,1);
        \draw[darkblue, fill=cadet, thick] (\i,1.5) rectangle (\i+1,2.5);
        \node at (\i+0.5,2.0) {${\scriptstyle \hat{x}_\i}$};
        \node at (\i+0.5,0.5) {${\scriptstyle \hat{y}_\i}$};
      }
      \node[anchor=west] at (4.5,1.25) {$\omega_3 = \sum_{i=0}^1 \hat{x}_i \hat{y}_i$};
    \end{tikzpicture}
  \end{center}
\end{frame}

\begin{frame}
  \frametitle{Program on processor $p$}
  \begin{align*}
    \hat{\bm x}, \hat{\bm y} &: \text{ vectors of dimension } I_p \\
    \omega_p &= \sum_{i=0}^{I_p-1} \hat{x}_i \hat{y}_i
    = \sum_{i=0}^{I_p-1} x_{\mu^{-1}(p,i)} y_{\mu^{-1}(p,i)}
    = \sum_{m \in \mathcal{N}_p} x_m y_m. \\
                 & \text{Send } \omega_p  \text{ to processor } q \not= p \\
                 & \text{Receive } \omega_q  \text{ from processor } q \\
    \sigma &= \sum_{q=0}^{P-1} \omega_q.
  \end{align*}
\end{frame}

\begin{frame}
  \frametitle{Global reduction}
  Reduction algorithm for the global sum
  \[
    \sigma = \sum_{q=0}^{P-1} \omega_q.
  \]
  \texttt{\textcolor{red}{MPI\_Reduce}(}$\omega$, $\sigma$,
  \texttt{1, MPI\_DOUBLE, MPI\_SUM, 0, MPI\_COMM\_WORLD)} \\
  (the answer will be known to process zero), or \\~\\
  \texttt{\textcolor{red}{MPI\_Allreduce}(}$\omega$, $\sigma$,
  \texttt{1, MPI\_DOUBLE, MPI\_SUM, MPI\_COMM\_WORLD)}
  (the answer will be known to every process)
\end{frame}

\begin{frame}
  \frametitle{Global sum $P=4$, \texttt{MPI\_Reduce}}
  \begin{center}
    \begin{tikzpicture}[scale=0.7]
      \node (a0) at (0,3) {$p=0: \quad \omega_0$};
      \node (a1) at (0,2) {$p=1: \quad \omega_1$};
      \node (a2) at (0,1) {$p=2: \quad \omega_2$};
      \node (a3) at (0,0) {$p=3: \quad \omega_3$};

      \node[right=1 of a0] (b0) {$\omega_0 + \omega_1$};
      \node[right=1 of a2] (b2) {$\omega_2 + \omega_3$};

      \node[right=1 of b0] (c0) {$\omega_0 + \omega_1 + \omega_2 + \omega_3 = \sigma$};

      \draw[thick, darkblue, ->] (a1.east) -- (b0.west);
      \draw[thick, darkblue, ->] (a3.east) -- (b2.west);
      \draw[thick, darkblue, ->] (b2.east) -- (c0.west);
    \end{tikzpicture}
  \end{center}
  Can be completed in $\log_2 P$ steps.
\end{frame}

\begin{frame}
  \frametitle{Global sum $P=4$, \texttt{MPI\_Allreduce}}
  \begin{center}
    \begin{tikzpicture}[scale=0.7]
      \node (a0) at (0,3) {$p=0: \quad \omega_0$};
      \node (a1) at (0,2) {$p=1: \quad \omega_1$};
      \node (a2) at (0,1) {$p=2: \quad \omega_2$};
      \node (a3) at (0,0) {$p=3: \quad \omega_3$};

      \node[right=1 of a0] (b0) {$\omega_0 + \omega_1$};
      \node[right=1 of a1] (b1) {$\omega_1 + \omega_0$};
      \node[right=1 of a2] (b2) {$\omega_2 + \omega_3$};
      \node[right=1 of a3] (b3) {$\omega_3 + \omega_2$};

      \node[right=1 of b0] (c0) {$\omega_0 + \omega_1 + \omega_2 + \omega_3 = \sigma$};
      \node[right=1 of b1] (c1) {$\omega_1 + \omega_0 + \omega_3 + \omega_2 = \sigma$};
      \node[right=1 of b2] (c2) {$\omega_2 + \omega_3 + \omega_0 + \omega_1 = \sigma$};
      \node[right=1 of b3] (c3) {$\omega_3 + \omega_2 + \omega_1 + \omega_0 = \sigma$};

      \draw[thick, darkblue, ->] (a0.east) -- (b1.west);
      \draw[thick, darkblue, ->] (a1.east) -- (b0.west);
      \draw[thick, darkblue, ->] (a2.east) -- (b3.west);
      \draw[thick, darkblue, ->] (a3.east) -- (b2.west);
      \draw[thick, darkblue, ->] (b0.east) -- (c2.west);
      \draw[thick, darkblue, ->] (b1.east) -- (c3.west);
      \draw[thick, darkblue, ->] (b2.east) -- (c0.west);
      \draw[thick, darkblue, ->] (b3.east) -- (c1.west);
    \end{tikzpicture}
  \end{center}
  Can be completed in $\log_2 P$ steps.
\end{frame}

\begin{frame}
  \frametitle{Program on processor $p$}
  \begin{align*}
    \hat{\bm x}, \hat{\bm y} &: \text{ vectors of dimension } I_p \\
    \sigma &= \sum_{i=0}^{I_p-1} \hat{x}_i \hat{y}_i
             = \sum_{i=0}^{I_p-1} x_{\mu^{-1}(p,i)} y_{\mu^{-1}(p,i)}
             = \sum_{m \in \mathcal{N}_p} x_m y_m. \\
                             & \text{for } d=0,\ldots,\log_2 P - 1 \\
    & \quad \text{Send } \sigma \text{ to processor } q = p \;\overline{\vee}\; 2^d \\
    & \quad \text{Receive } \sigma_q \text{ from processor } q = p \;\overline{\vee}\; 2^d \\
    & \quad \sigma = \sigma + \sigma_q \\
    & \text{end}
  \end{align*}
  Here, $\overline{\vee}$ is exclusive or. $p$ and $p\;\overline{\vee}\;2^d$
  differ only in bit $d$.
\end{frame}

\input{postamble}
