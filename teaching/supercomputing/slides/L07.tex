\input{preamble}

\title{Solving PDEs: the Poisson problem}
\institute{NTNU, IMF}
\date{March 9. 2018}
\author{Based on 2016v slides by Eivind Fonn}
%\author{Aurélien Larcher}

\maketitle

%------------------------------------------------------------------------------
\begin{frame}
  \frametitle{The Poisson problem}
\begin{itemize}
  \item The Poisson equation is an elliptic partial differential equation.
  \item The Poisson \emph{problem} is the solution of the Poisson
    \emph{equation} equipped with a set of boundary conditions.
  \item The equation is
    \[
      -\Delta u = f, \qquad \text{ in } \Omega
    \]
    where $u$ is the unknown, $f$ is the load on the system and $\Omega$ denotes
    some domain.
  \item We remark that $\Delta$ is the sum of the second order partial
    derivatives (i.e trace of Hessian matrix), e.g. in one and two dimensions the equation is
    \[
      -u_{xx} = f, \qquad -\left( u_{xx} + u_{yy} \right) = f.
    \]
\end{itemize}
\end{frame}

%------------------------------------------------------------------------------
\begin{frame}[fragile]
  \frametitle{The Poisson problem}
  \begin{itemize}
  \item This problem is important because a number of physical processes are
    modelled either entirely or in part by the Poisson equation.
  \item The technical term is a \emph{diffusion process}: $u$ is to be
    interpreted as the concentration of a physical quantity, and $f$ as the rate at
    which  is introduced ($f > 0$) or removed ($f < 0$) from the
    domain.
  \item The physical quantity (typically intensive) may be a concentration, temperature, potentials, \dots
  \item The solution in $\Omega$ is uniquely determined by the boundary data and
    $f$: this problem involves an elliptic equation.
  \item Intuitively, the differential operator has a \textit{smoothing} effect.
  \end{itemize}
\end{frame}

%------------------------------------------------------------------------------
\begin{frame}[fragile]
  \frametitle{Steady heat transfer}
  Energy transfered out of an arbitrary domain $V$ can be expressed as
  \[
    \int_{\partial V} \bm q \cdot \bm n \dif{S} = \int_V f \dif{V}.
  \]
  where $\bm q$ is the heat flux, $\bm n$ is the outward surface normal along
  the boundary $\partial V$ and $f$ represents a volumetric heat source. This
  basically says that the net energy generation inside the domain must equal the
  net energy flowing out of the domain.
  \begin{center}
    \input{\figs/tma4280/domain}
  \end{center}
\end{frame}

%------------------------------------------------------------------------------
\begin{frame}[fragile]
  \frametitle{Steady heat transfer}
  \begin{itemize}
  \item Applying the Gauss divergence theorem, we can write
    \[
      \int_{\partial V} \bm q \cdot \bm n \dif{S}
      = \int_V \nabla \cdot \bm q \dif{V}
      = \int_V f \dif{V},
    \]
    yielding
    \[
      \nabla \cdot \bm q = f.
    \]
  \item Applying Fourier's law, i.e. $\bm q = -\kappa \nabla u$, $\kappa > 0$,
    we get
    \[
      - \nabla\cdot\kappa\nabla u = f \qquad \text{ in } \Omega,
    \]
    to be solved for the temperature $u$.
  \item For a constant isotropic heat conductivity $\kappa$, we regain the
    Poisson equation,
    \[
      - \kappa\Delta u = f.
    \]
  \end{itemize}
\end{frame}

%------------------------------------------------------------------------------
\begin{frame}[fragile]
  \frametitle{Applications: Electrostatics}
  \begin{itemize}
  \item The differential forms for the electric field $\bm E$ are
    \[
      \begin{split}
        \nabla \cdot \bm E &= 4\pi\rho \\
        \nabla \times \bm E &= 0,
      \end{split}
    \]
    where $\rho$ is the charge density.
  \item The electric field $\bm E$ can be expressed as the gradient of
    a scalar potential $\varphi$, i.e. $\bm E = -\nabla\varphi$. Thus
    \[
      \nabla \cdot \bm E = -\nabla\cdot\nabla\varphi = -\Delta\varphi = 4\pi\rho.
    \]
  \end{itemize}
\end{frame}

%------------------------------------------------------------------------------
\begin{frame}
  \frametitle{Applications: Potential flow}
  \begin{itemize}
  \item Likewise, potential flow in fluid mechanics can be modelled by this
    equation.
  \item Given a velocity field $\bm U$ which is irrotational and incompressible,
    \[
      \begin{split}
        \nabla \times \bm U &= 0 \\
        \nabla \cdot \bm U &= 0,
      \end{split}
    \]
    it follows that $\bm U = \nabla\varphi$ where $\varphi$ is a scalar velocity
    potential, which satisfies the Laplace equation
    \[
      \Delta\varphi = 0.
    \]
  \end{itemize}
\end{frame}

%------------------------------------------------------------------------------
\begin{frame}
  \frametitle{Applications: Numerical methods}


Use of the Poisson equation can also be guided by the numerics \\and not explicitly by the physical model:
\begin{itemize}
\item Projection methods for incompressible Navier--Stokes: Helmoltz decomposition of $L^2$ function into divergence free $L^2$ function and gradients of $H^1$ function.
\begin{enumerate}
\item Velocity prediction: solving the momentum equation.
\item Projection step: project the predicted velocity onto solenoidal space.
\end{enumerate}

\bigskip
\item Homogenization methods for multiscale problems: solving a local problem, typically a Poisson problem on a cell, to model small scales.
\end{itemize}

\bigskip
In any case, computing an approximation of a solution to the Poisson requires:
\begin{itemize}
\item a discretization (physical space, frequential space) 
\item and a discretization in time in the case of evolution problems.
\end{itemize}

\end{frame}


%------------------------------------------------------------------------------
\begin{frame}
  \frametitle{Unsteady heat transfer: discretization in time}
  \begin{itemize}
  \item Unsteady heat transfer is modelled by the heat equation
    \[
      \frac{\partial u}{\partial t} = \kappa\Delta u + f \qquad \text{ in } \Omega.
    \]
  \item Discretizing in time using Backward Euler, we otain
    \[
      \frac{1}{\Delta t} (u^{n+1}-u^n) = \kappa\Delta^{n+1}+f^{n+1}
    \]
    where superscript $n$ refers to a quanity at time $t^n, n=0,1,\ldots$.
  \item This can be written as
    \[
      \left(-\kappa\Delta + \frac{1}{\Delta t}\right)u^{n+1} = \frac{u^n}{\Delta t}+f^{n+1}.
    \]
  \item This is a \emph{Helmholtz} equation: Laplacian plus a multiple of the
    identity.
  \end{itemize}
\end{frame}

%------------------------------------------------------------------------------
\begin{frame}
  \frametitle{Solving PDEs: mesh generation}

Depending on the discretization, cartesian grids or simplicial meshes may be used.

\medskip
Complex geometries are usually more suited for simplicial meshes:
\begin{figure}[hbt]
  \centering
  \includegraphics[width=0.6\linewidth]{ctl/wing}
\end{figure}
but numerical methods like immersed boundaries or fictitious domain approaches maybe used.

\end{frame}

%------------------------------------------------------------------------------
\begin{frame}
  \frametitle{Solving PDEs: mesh distribution, domain decomposition}

\begin{enumerate}
\item Solving in parallel requires mesh distribution:
\begin{center}
   \begin{minipage}[bc]{0.3\linewidth}
   \centering
   \begin{figure}
    \centering
    \includegraphics[width=\linewidth]{dolfin/snake}
\end{figure}
   \end{minipage}
   \begin{minipage}[bc]{0.3\linewidth}
   \centering
   \begin{figure}
    \centering
    \includegraphics[width=\linewidth]{dolfin/gear}
    \end{figure}
   \end{minipage}
\end{center}
Different partitioning software packages: ParMetis, Zoltan, Scotch.

\bigskip
\item Domain decomposition: not to be confused with mesh distribution! Such methods can be used for:
\begin{itemize}
\item accelerate the resolution,
\item deal with multiphysics,
\item solve on composite domains.
\end{itemize}

\end{enumerate}
\end{frame}

%------------------------------------------------------------------------------
\begin{frame}
  \frametitle{Solving PDEs: adaptivity}

\begin{enumerate}
\item \textit{h-adaptivity}: refine the mesh by dividing cells
\item \textit{p-adaptivity}: increase the polynomial order,
\item \textit{r-adaptivity}: move mesh to increase accuracy locally,
\end{enumerate}

\medskip
Goal-oriented adaptivity: adaptation w.r.t a goal functional
\begin{figure}[hbt]
  \centering
  \href{run:\figs/ctl/banc2.mp4}{\includegraphics[width=0.6\linewidth]{ctl/banc2_stages}}
  \caption{Lighthill tensor. Vilela De Abreu/N. Jansson/Hoffman (2012)}
\end{figure}

\end{frame}

\begin{frame}
  \frametitle{Finite difference methods}
  \begin{itemize}
    \item Consider a continuous function 1D function $u(x)$.
    \item Introduce a grid, $\left\{x_i\right\}_{i=0}^N$, with $x_i = x_0+ih$.
    \item Here $h$ is the grid spacing. For simplicity we consider equidistant
      grids (constant $h$).
      \begin{center}
        \scalebox{0.7}{\input{\figs/tma4280/sampling}}
      \end{center}
    \item Want to approximate derivatives of the function only using data on the grid.
  \end{itemize}
\end{frame}

\begin{frame}
  \frametitle{Finite differences}
  \begin{itemize}
  \item First idea:  linear approximation of slope
    \[
      u'(x_i) \approx \frac{1}{h} \left( u(x_i+h)-u(x_i) \right).
    \]
  \item This is called a one-sided difference (a \emph{forward} difference).
    Invoking Taylor we find
    \begin{align*}
      \frac{1}{h} \left( u(x_i+h)-u(x_i) \right)
      &= \frac{1}{h} \left( u(x_i) + hu'(x_i) + \mathcal{O}(h^2) - u(x_i) \right) \\
      &= u'(x_i) + \mathcal{O}(h)
    \end{align*}
    In other words, this is a \emph{first order} approximation to $u'(x_i)$.
  \end{itemize}
\end{frame}

\begin{frame}
  \frametitle{Finite differences}
  \begin{itemize}
  \item Second idea: a centered difference
    \[
      u'(x_i) \approx \frac{1}{2h} \left( u(x_i + h) - u(x_i - h) \right).
    \]
  \item Invoking Taylor we find
    \[
      \frac{1}{2h} \left( u(x_i + h) - u(x_i - h) \right)
      = u'(x_i) + \mathcal{O}(h^2).
    \]
    In other words, this is a \emph{second order} approximation to $u'(x_i)$.
  \end{itemize}
\end{frame}

%------------------------------------------------------------------------------
\begin{frame}
  \frametitle{Finite difference methods}
  \begin{itemize}
  \item A centered difference for the second derivative:
    \begin{align*}
      u''(x_i) &\approx \frac{1}{h}\left( u'(x_i + \nicefrac{h}{2}) - u'(x_i - \nicefrac{h}{2})\right) \\
      &\approx \frac{1}{h^2}\left( u(x_i + h) - 2u(x_i) + u(x_i - h) \right)
    \end{align*}
  \item Invoking Taylor we find
    \[
      \frac{1}{h^2}\left( u(x_i + h) - 2u(x_i) + u(x_i - h) \right)
      = u''(x_i) + \mathcal{O}(h^2).
    \]
    In other words, this is a \emph{second order} approximation to $u''(x_i)$.
  \end{itemize}
\end{frame}

%------------------------------------------------------------------------------
\begin{frame}
  \frametitle{Stencils}
  Finite differences are often illustrated using \emph{stencils}.
  \begin{figure}
    \begin{center}
      \input{\figs/tma4280/three-point-stencil-first}
    \end{center}
    \caption{Stencil for the first derivative (central difference)}
  \end{figure}
  \begin{figure}
    \begin{center}
      \input{\figs/tma4280/three-point-stencil-second}
    \end{center}
    \caption{Stencil for the second derivative}
  \end{figure}
\end{frame}

%------------------------------------------------------------------------------
\begin{frame}[fragile]
  \frametitle{Discretization in 1D}
  \begin{itemize}
  \item We now consider the 1D Poisson problem
    \[
      \begin{split}
        -u_{xx} &= f, \qquad \text{ in } \Omega = (0,1), \\
        u(0) &= u(1) = 0.
      \end{split}
    \]
    \begin{center}
      \scalebox{0.7}{\input{\figs/tma4280/poisson-solution}}
    \end{center}
  \item These are called homogenous Dirichlet boundary conditions: the solution
    is prescribed to be zero on the boundaries of the domain.
  \item Introduce the grid, $\left\{x_i\right\}_{i=0}^N$, with $x_i = x_0+ih$,
    $h = \nicefrac{1}{N}$.
    \begin{center}
      \input{\figs/tma4280/finite-difference}
    \end{center}
  \end{itemize}
\end{frame}

%------------------------------------------------------------------------------
\begin{frame}
  \frametitle{Discretization in 1D}
  \begin{itemize}
  \item Let $u_i$ denote $u(x_i)$ and $f_i$ denote $f(x_i)$.
  \item Due to the boundary conditions, we know that $u_0 = u_N = 0$.
  \item We thus have $N-1$ unknowns that we collect in a vector $\bm u$.
  \item We then apply the second order finite difference formula in each grid
    point.
    \begin{align*}
      - \frac{1}{h^2} \left(u_{i+1} - 2u_i + u_{i-1} \right) &= f_i, \qquad i=1,\ldots,n-1, \\
      u_0 &= u_N = 0.
    \end{align*}
  \end{itemize}

\medskip The algebraic unknowns are also called degrees of freedoms: computing a discrete solution consists of determining a numerical for each degree of freedom.
\end{frame}

%------------------------------------------------------------------------------
\begin{frame}
  \frametitle{Discretization in 1D}
  These equations can also be expressed as the system
  \begin{align*}
    2u_1 - u_2 &= h^2 f_2 \\
    -u_1 + 2u_2 - u_3 &= h^2 f_3 \\
               &\vdots \\
    -u_{N-2} + 2u_{N-1} &= h^2 f_{N-1},
  \end{align*}
  or in matrix form
  \begin{align*}
    \underbrace{ \begin{pmatrix}
      2 & -1 & & & \\
      -1 & 2 & -1 & & \\
      & & \ddots & & \\
      & & -1 & 2 & -1 \\
      & & & -1 & 2
    \end{pmatrix}
    }_{\bm A}
    \underbrace{ \begin{pmatrix}
      u_1 \\
      u_2 \\
      \vdots \\
      u_{n-2} \\
      u_{n-1}
    \end{pmatrix}
    }_{\bm u}
    &= h^2
    \underbrace{ \begin{pmatrix}
      f_1 \\
      f_2 \\
      \vdots \\
      f_{n-2} \\
      f_{n-1}
    \end{pmatrix}
    }_{\bm f} ,
  \end{align*}
\end{frame}

%------------------------------------------------------------------------------
\begin{frame}
  \frametitle{Discretization in 1D}
  \begin{itemize}
  \item The matrix $\bm A$ is sparse (tridiagonal).
  \item The matrix $\bm A$ is symmetric, that is $\bm A = \bm A^\intercal$.
  \item The matrix $\bm A$ is positive definite, that is,
    $\bm v^\intercal \bm A \bm v > 0$
    for all vectors $\bm v \in \mathbb{R}^{N-1}$, $\bm v \not= \bm 0$.
  \item Thus, the system of $N-1$ equations is solvable and has a unique
    solution.
  \item The error in the grid points is of second order, i.e.
    $|u(x_i)-u_i| \sim \mathcal{O}(h^2)$.
  \end{itemize}
\end{frame}

%------------------------------------------------------------------------------
\begin{frame}
  \frametitle{Discretization in 2D}
  If the domain $\Omega$ is rectangular, we can use two independent grids
  $\{x_i\}_{i=0}^N$ and $\{y_j\}_{j=0}^M$in the $x$ and $y$-directions.
  \begin{center}
    \scalebox{0.7}{\input{\figs/tma4280/2d-grid}}
  \end{center}
  \begin{align*}
    x_i &= x_0 + ih_x, \\
    y_j &= y_0 + jh_y.
  \end{align*}
\end{frame}

%------------------------------------------------------------------------------
\begin{frame}
  \frametitle{Finite differences in 2D}
  We need to approximate $u_{xx} + u_{yy}$ at a point $(x_i, y_j)$. Therefore,
  we use two one-dimensional finite differences,
  \begin{align*}
    u_{xx}(x_i, y_i) &\approx \frac{1}{h_x^2}
                       \left( u_{i-1,j} - 2u_{i,j} + u_{i+1,j} \right) \\
    u_{yy}(x_i, y_i) &\approx \frac{1}{h_y^2}
                       \left( u_{i,j-1} - 2u_{i,j} + u_{i,j+1} \right) \\
  \end{align*}
  If $h_x = h_y = h$ we get
  \[
    \nabla u(x_i, y_i) \approx \frac{1}{h^2}
    \left( u_{i-1,j} + u_{i+1,j} + u_{i,j-1} + u_{i,j+1} - 4u_{i,j} \right)
  \]

\medskip 2D finite differences result from the discretization of the differential operator along each axis.
\end{frame}

%------------------------------------------------------------------------------
\begin{frame}
  \frametitle{Finite differences in 2D}
  This is known as the \emph{five-point} stencil. (The signs are flipped because
  the Poisson equation involves $-\nabla$.)
  \begin{center}
    \input{\figs/tma4280/five-point-stencil}
  \end{center}
\end{frame}

%------------------------------------------------------------------------------
\begin{frame}
  \frametitle{Node numbering in 2D}
  Numbering nodes is not as trivial in 2D as in 1D. We use a ``natural''
  ordering of the unknowns: we first number all the \emph{internal} nodes along
  ``row'' 1 (in the $x$-direction), followed by the nodes in ``row'' 2, etc.
  \begin{center}
    \input{\figs/tma4280/numbering}
  \end{center}

\medskip The numbering of the nodes is important as it is reflected in the structure of the matrix if the same numbering is used for the algebraic system is used.
\end{frame}

%------------------------------------------------------------------------------
\begin{frame}
  \frametitle{Matrix block structure}
  The final linear system will have a block-like structure,
  \[
    \underbrace{
      \begin{pmatrix}
        \bm A_0 & \bm A_1 & & & \\
        \bm A_1 & \bm A_0 & \bm A_1 & & \\
        & \bm A_1 & \bm A_0 & \ddots & \\
        & & \ddots & \ddots & \bm A_1 \\
        & & & \bm A_1 & \bm A_0
      \end{pmatrix}}_{\bm A}
    \underbrace{
      \begin{pmatrix}
        u_1 \\ u_2 \\ u_3 \\ \vdots \\ u_N
      \end{pmatrix}}_{\bm u}
    =
    \underbrace{
      \begin{pmatrix}
        f_1 \\ f_2 \\ f_3 \\ \vdots \\ f_N
      \end{pmatrix}}_{\bm g}
  \]

\medskip
The resulting matrix is \textit{sparse}, which means that it contains only a few non-zero entries.

\medskip
This structure is the algebraic counterpart of the locality of the stencil: the differential operator only involves neighbours.
\end{frame}

\begin{frame}
  \frametitle{Matrix block structure}
  The blocks $\bm A_0$ and $\bm A_1$ can be written as
  \[
    \bm A_0 =
    \begin{pmatrix}
      4 & -1 & & & \\
      -1 & 4 & -1 & & \\
      & -1 & 4 & \ddots & \\
      & & \ddots & \ddots & -1 \\
      & & & -1 & 4
    \end{pmatrix}
    \qquad
    \bm A_1 =
    \begin{pmatrix}
      -1 & & & & \\
      & -1 & & & \\
      & & -1 & & \\
      & & & \ddots & \\
      & & & & -1
    \end{pmatrix}.
  \]
\end{frame}

%------------------------------------------------------------------------------
\begin{frame}
  \frametitle{Assembling linear systems}

Assembly of the linear system may be non-negligible in terms of computational time:
\begin{itemize}
\item Depending on the time discretization (explicit, implicit, \dots) the matrix may or may not be reassembled.
\item Assembling the contributions may be expensive (finite elements).
\item Performance may be improved by parallelism (OpenMP).
\item Parallelism can be also used for overlapping assembly and solve.
\end{itemize}


\end{frame}

%------------------------------------------------------------------------------
\begin{frame}
  \frametitle{Solving linear systems}

Three sources of performance:
\begin{enumerate}
\item general improvement of algorithms e.g Krylov solvers and preconditioners,
\item exploiting the structure of the discrete problem by choosing a specific solver,
\item using parallelism in implementations (OpenMP).
\end{enumerate}

\bigskip
Improvement can of course come from improving serial implementations: optimized assembly, template meta-programming, optimization of kernels by code generation \dots

\end{frame}

\input{postamble}
