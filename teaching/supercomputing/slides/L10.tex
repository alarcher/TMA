\documentclass[10pt,final,leqno]{beamer}

%\usetheme{CambridgeUS}
\usetheme[style=ntnu,language=en]{ntnu2015}
%\usecolortheme{orchid}

\usepackage[utf8]{inputenc}
\usepackage[T1]{fontenc}

% Paths
\newcommand{\figs}{../figs}
\newcommand{\data}{../data}
\newcommand{\code}{../code}

% URL styles
\usepackage{url}
\urlstyle{sf}

% Units
\usepackage[detect-weight=true, binary-units=true]{siunitx}
\DeclareSIUnit\flop{FLOPS}

% Math
\usepackage{amsmath}
\usepackage{amssymb}
\usepackage{bm}
\usepackage{nicefrac}
\newcommand{\dif}[1]{{\;\text{d}#1}}

% Graphics
\usepackage{graphicx}
\usepackage{caption}
%\usepackage{subcaption}
\graphicspath{{../figs/}}

% Tikz
\usepackage{tikz}
\usetikzlibrary{positioning,shapes,arrows,calc,intersections}
\usepackage{pgfplots}
\usepgfplotslibrary{dateplot}
\pgfplotsset{compat=1.14} 

% Colors
\definecolor{darkblue}{HTML}{00688B}
\definecolor{darkgreen}{HTML}{6E8B3D}
\definecolor{cadet}{HTML}{DAE1FF}
\definecolor{salmon}{HTML}{FFB08A}

\usepackage{algorithm}

% Listings
\usepackage{textcomp}
\usepackage{listings}
\lstset{
  keywordstyle=\bfseries\color{orange},
  stringstyle=\color{darkblue!80},
  commentstyle=\color{darkblue!80},
  showstringspaces=false,
  basicstyle=\ttfamily,
  upquote=true,
}
\lstdefinestyle{fortran}{
  language=Fortran,
  morekeywords={for},
  deletekeywords={status},
}
\lstdefinestyle{c}{
  language=C,
  morekeywords={include},
}
\lstdefinestyle{glsl}{
  language=C,
  morekeywords={attribute, vec2, vec3, vec4, varying, uniform, mat2, mat3, mat4},
}
\lstdefinestyle{cuda}{
  language=C,
  morekeywords={__global__, __device__, __host__, __shared__},
}
\lstdefinestyle{shell}{
  language=bash,
  morekeywords={mkdir, ssh, cmake},
}

% Double hlines
\usepackage{hhline}

% Misc
\usepackage{nth}

\subtitle{TMA4280---Introduction to Supercomputing}

\graphicspath {{../figs/}}

\AtBeginSection[]
{
 \begin{frame}<beamer>
 \frametitle{Outline}
 \tableofcontents[currentsection]
 \end{frame}
}

\begin{document}


%\newcommand{\xC}{\mathbb{C}}
\newcommand{\xR}{\mathbb{R}}
\newcommand{\xRd}{{\xR^d}}
\newcommand{\xRN}{{\xR^N}}
\newcommand{\xMNR}{{M_N(\xR)}}
\newcommand{\bb}{{\boldsymbol b}}
\newcommand{\ee}{{\boldsymbol e}}
\newcommand{\ev}{{\boldsymbol \epsilon}}
\newcommand{\rr}{{\boldsymbol r}}
\newcommand{\xx}{{\boldsymbol x}}
\newcommand{\hx}{\hat{\boldsymbol x}}
\newcommand{\yy}{{\boldsymbol y}}
\newcommand{\vv}{{\boldsymbol v}}
\newcommand{\ww}{{\boldsymbol w}}
\newcommand{\zz}{{\boldsymbol z}}
\renewcommand{\mA}{{\mathrm A}}
\newcommand{\mB}{{\mathrm B}}
\newcommand{\mC}{{\mathrm C}}
\newcommand{\mD}{{\mathrm D}}
\newcommand{\mG}{{\mathrm G}}
\newcommand{\mH}{{\mathrm H}}
\renewcommand{\mL}{{\mathrm L}}
\newcommand{\mLs}{{\mathrm L_0}}
\newcommand{\mM}{{\mathrm M}}
\newcommand{\mRs}{{\mathrm R_0}}
\newcommand{\mR}{{\mathrm R}}
\newcommand{\mP}{{\mathrm P}}
\newcommand{\mQ}{{\mathrm Q}}
\newcommand{\mU}{{\mathrm U}}
\newcommand{\mId}{{\mathbf{Id}}}
\newcommand{\mII}{{\mathbf{\mathbb{I}}}}
\newcommand{\Seq}[1]{\bigl(#1\bigr)}
\newcommand{\Cond}[1]{\mathcal{C}(#1)}
\newcommand{\Order}[1]{\mathcal{O}\left(#1\right)}
\newcommand{\norm}[1]{{\lVert #1 \rVert}}
\newcommand{\norminf}[1]{\norm{#1}_{\infty}}

\newcommand{\InnerK}[2]{{{\mathbf\langle}\;#1\:,\: #2 \;{\rangle}}}
\newcommand{\Inner}[2]{{{\scriptstyle\mathbf{(}}\;#1\:,\: #2 \;{\scriptstyle\mathbf{)}}}}


\title{Sparse systems with PETSc}
\institute{NTNU, IMF}
\date{April 13. 2018}
%\author{Based on 2016v slides by Eivind Fonn}
%\author{Aurélien Larcher}

\maketitle

\begin{frame}
  \frametitle{Linear algebra packages}
  \begin{itemize}
  \item During the course BLAS and LAPACK were discussed for dense linear algebra.
  \item Most engineering applications involve solving sparse linear systems.
  \item For instance Partial Differential Equation discretizations  involve:
  \begin{itemize}
  \item discrete differential operators in the form of stencils (finite differences, finite volums)
  \item discrete spaces with compactly supported basis functions (finite elements)
  \end{itemize}
  \item This locality of the discretization translates into the structure of the matrix.
  \item Sparse linear algebra packages are available:
    \begin{itemize}
    \item MUMPS: MUltifrontal Massively Parallel sparse direct Solver,
    \item UMFPACK: Unsymmetric MultiFrontal method,
    \item SuperLU: Sparse, direct solver (comes in threaded and distributed
      variants but no hybrid)
    \item HYPRE: Sparse, iterative solvers and preconditioners
    \end{itemize}
  \end{itemize}
\end{frame}

\begin{frame}
  \frametitle{Linear algebra packages}
  Some linear algebra packages provide more then linear solvers, but contain other
  tools that are helpful in scientific computing as well.
  \begin{enumerate}
\medskip
  \item \textbf{PETSc}: Portable Extensible Toolkit for Scientific computing.
\begin{center}
\url{https://www.mcs.anl.gov/petsc/}
\end{center}
\medskip
  \item \textbf{Trilinos}: Object-oriented software framework for the solution of large-scale, complex multi-physics engineering and scientific problems (can be used trough PETSc).
\begin{center}
\url{https://trilinos.org/}
\end{center}
  \end{enumerate}
  

\medskip
Such frameworks offer generic interfaces which can be also used for external packages: it is interesting to implement them in your code to be able to use a common interface.

\end{frame}

\begin{frame}
  \frametitle{PETSc}
  \begin{itemize}
  \item PETSc (PET-see) is an open-source scientific computing library written
    in C.
  \item Written in C with an object-oriented design.
  \item Interfaces for Fortran, Python, Java and Matlab available.
  \item Great support for distributed programming: MPI, and GPUs through CUDA or OpenCL, as well as hybrid MPI-GPU parallelism. 
  \end{itemize}

\medskip
  \begin{itemize}
  \item Parallel vectors with synchronization of ghost entries.
  \item Parallel matrices with different storage foramts and distributed implementations.
  \item Linear solvers: direct and preconditioned iterative methods for sparse matrices (\href{list}{https://www.mcs.anl.gov/petsc/documentation/linearsolvertable.html}).
  \item Parallel Newton-based nonlinear solvers.
  \item Parallel timestepping (ODE) solvers.
  \item Distributed arrays for finite difference methods. 
  \end{itemize}

\end{frame}

\begin{frame}
  \frametitle{PETSc conventions}
  \begin{itemize}
  \item PETSc is a large library, so it needs conventions to keep it organized.
  \item Methods and symbols have prefixes which relate to their category.
  \vspace{2ex}
    \begin{tabular}{ll}
    \texttt{Vec} & Vectors \\
    \texttt{Mat} & Matrices \\
    \texttt{KSP} & Krylov solvers (CG, Bi-CGStab, GMRES, \dots)\\
    \texttt{PC}  & Preconditioners (Jacobi, SOR, ILU, \dots)\\
    \end{tabular}
\medskip
\item Extensive documentation:
\begin{center}
http://www.mcs.anl.gov/petsc/petsc-current/docs/manualpages/singleindex.html
\end{center}
  \end{itemize}
\end{frame}

\begin{frame}[fragile]
  \frametitle{PETSc Vectors}
  \begin{itemize}
  \item A vector is stored in an opaque data structure called a \texttt{Vec}:
\begin{lstlisting}[style=c]
// Look at petscvec.h
typedef struct _p_Vec*         Vec;
\end{lstlisting}
  \item Vectors can have different types, such as
    \begin{tabular}{ll}
    seq     & Sequential \\
    mpi     & Distributed \\
    pthread & Threaded \\
    cusp    & CUDA (Nividia GPU) format through \href{https://cusplibrary.github.io/}{CUSP}
    \end{tabular}
  \item The type and its implementation is hidden from the user: it is an
    implementation detail.
  \item There is \emph{no} direct data access: similar to MPI access to data through
    functions.
  \item This helps to abstract away implementation details for different
    types.
  \end{itemize}
\end{frame}

\begin{frame}[fragile]
  \frametitle{PETSc Vectors: first steps}

Declaration of a vector: a pointer to the actual implemented type 
\begin{lstlisting}[style=c]
// PETSc Vec pointer
Vec x;
// Local size
PetscInt n;
\end{lstlisting}

\medskip
Creation of a sequential vector:
\begin{lstlisting}[style=c]
VecCreate(PETSC_COMM_SELF, &x);
VecSetSizes(x, PETSC_DECIDE, n);
VecSetFromOptions(x);
\end{lstlisting}

\medskip
Creation of a MPI vector, the local size only is specified:
\begin{lstlisting}[style=c]
VecCreateMPI(PETSC_COMM_WORLD, n, PETSC_DETERMINE,
             &x);
\end{lstlisting}

\end{frame}

\begin{frame}[fragile]
  \frametitle{PETSc Vectors: first steps}

\medskip
Any operation is performed through functions:
\begin{lstlisting}[style=c]
// Set all entries to zero
PetscScalar a = 0.0;
VecSet(x, a);

// Get global size
PetscInt N;
VecGetSize(x, &N);

\end{lstlisting}

PETSc provides function to display the content:
\begin{lstlisting}[style=c]
VecView(x, PETSC_VIEWER_STDOUT_WORLD);
\end{lstlisting}

PETSc object must be deallocated to reclaim the memory:
\begin{lstlisting}[style=c]
VecDestroy(&x);
\end{lstlisting}

\end{frame}

\begin{frame}[fragile]
  \frametitle{PETSc Vectors: some predefined operations}
\begin{lstlisting}[style=c]
PetscScalar a; Vec x; Vec y; Vec w;

VecScale(x, a);

VecAXPY(x, a, y);

VecDot(x, y, &a);

VecPointwiseMult(w, x, y);

VecMin(x, PETSC_NULL, &a);
VecMax(x, PETSC_NULL, &a);

VecNorm(x, NORM_2, &a);
\end{lstlisting}

\end{frame}

\begin{frame}[fragile]
  \frametitle{PETSc Vectors: setting values}

Two types of operations (example of encapsulation):
\begin{enumerate}
\item Insertion: only last value is retained
\begin{lstlisting}[style=c]
void PETScVector::set(const real* block, int m, 
                      int* rows)
{
  dolfin_assert(x);
  VecSetValues(x, m, rows, block, INSERT_VALUES);
}
\end{lstlisting}
\item Addition: all values are summed
\begin{lstlisting}[style=c]
void PETScVector::add(const real* block, int m,
                      int* rows)
{
  dolfin_assert(x);
  VecSetValues(x, m, rows, block, ADD_VALUES);
}
\end{lstlisting}
\end{enumerate}
\end{frame}

\begin{frame}[fragile]
  \frametitle{PETSc Vectors: ownership}

Data ownership is the crucial aspect of distributed computing:
\begin{itemize}
\item a vector on a process \textit{owns} a contiguous list of entries stored in the local memory.
\item the local size is the number of entries owned
\item the offset is the index of the first entry
\item the range is: $[offset, offset + local size [$
\item PETSc provides a function to get the range:
\begin{lstlisting}[style=c]
int low, high;
VecGetOwnershipRange(x, &low, &high);
\end{lstlisting}
\end{itemize}

\end{frame}

\begin{frame}[fragile]
  \frametitle{PETSc Vectors: ghost entries}

\begin{itemize}
\item If the problem is solved in parallel on a mesh, each process is responsible for a partition.
\item Example of a domain for the simulation of a Turbulent Jet in a cylinder.
\item Degrees of freedom on the inter-process (interior) boundary are shared between processes.
\item Only one owner, ghosted on other processes. 
\end{itemize}

\medskip
Ghosted entries are managed with: \texttt{VecCreateGhost}

\medskip
First step is to get the ownership range:
\begin{lstlisting}[style=c]
int local_size, size, low, high;
VecGetSize(x, &size);
VecGetLocalSize(x, &local_size);
VecGetOwnershipRange(x, &low, &high);
\end{lstlisting}

\end{frame}

\begin{frame}[fragile]
  \frametitle{PETSc Vectors: ghost entries}

Then build a list of indices \texttt{ghost\_indices} present on the process but outside the range.

\medskip
Ghost entries can then be specified:
\begin{lstlisting}[style=c]
VecCreateGhost(MPI::DOLFIN_COMM, local_size, size, 
	       num_ghost_indices, &ghost_indices[0], &x);
\end{lstlisting}

\medskip
Do not forget to synchronize the ghosts!
\begin{lstlisting}[style=c]
VecGhostUpdateBegin(x, INSERT_VALUES, SCATTER_FORWARD);
VecGhostUpdateEnd(x, INSERT_VALUES, SCATTER_FORWARD);
\end{lstlisting}

\end{frame}

\begin{frame}[fragile]
  \frametitle{PETSc Matrices}
  \begin{itemize}
  \item A matrix is stored in an opaque data structure called a \texttt{Mat}:
\begin{lstlisting}[style=c]
// Look at petscmat.h
typedef struct _p_Mat*           Mat;
\end{lstlisting}
  \item Matrices can also have different types depending on structure,
    \begin{tabular}{lll}
MATDENSE  & "dense" & dense matrices \\
MATAIJ    & "aij"   & sparse matrices \\
MATBAIJ   & "baij"  & block sparse matrices \\
MATSBAIJ  & "sbaij" & symmetric block sparse matrices
    \end{tabular}
  \item as well as specialized implementations (sequential, MPI, CUDA):
  \begin{itemize}
    \item seqdense: Sequential (normal) dense
    \item seqaij: Sequential (normal) sparse
    \item mpiaij: Distributed sparse
    \item aijcusp: CUDA (Nvidia GPU) sparse
  \end{itemize}
  \end{itemize}
\end{frame}

\begin{frame}[fragile]
  \frametitle{PETSc Matrices}

Example for a sequential matrix:
\begin{lstlisting}[style=c]
Mat A;
PetscInt M;
PetscInt N;

// Distributed with 50 non-zero entries per-row
MatCreateSeqAIJ(PETSC_COMM_SELF, M, N, 50, PETSC_NULL,
                &A);
\end{lstlisting}

Example for an MPI matrix:
\begin{lstlisting}[style=c]
// Distributed with 120 non-zero entries per-row 
// in DIAGONAL and OFF-DIAGONAL
MatCreateAIJ(PETSC_COMM_WORLD, PETSC_DECIDE, 
             PETSC_DECIDE, M, N, 120, PETSC_NULL,
             120, PETSC_NULL, &A);

\end{lstlisting}
For distributed matrices, DIAGONAL and OFF-DIAGONAL block are the natural extension of vector range to two dimensions.

\end{frame}

\begin{frame}
  \frametitle{Declare the sparsity pattern}
  \begin{itemize}
  \item Why is this important?
  \item PETSc uses a popular sparse matrix data structure called
    \emph{compressed row storage} (CRS).
  \item Each nonzero element in the matrix is stored along with its column index
    in a single, one-dimensional array.
  \item Another array designates the start of each row.
  \end{itemize}
\end{frame}

\begin{frame}[fragile]
  \frametitle{Compressed row storage: example}
  \[
    \bm A = \begin{pmatrix}
      1 & 0 & 0 & 2 \\ 0 & 0 & 4 & 0 \\ 0 & 5 & 0 & 6 \\ 7 & 0 & 0 & 0
    \end{pmatrix}
  \]
  \begin{lstlisting}
    vals = [1, 2, 4, 5, 6, 7]
    cols = [0, 3, 2, 1, 3, 0]
    rowptrs = [0, 2, 3, 5]
  \end{lstlisting}
\end{frame}

\begin{frame}
  \frametitle{Compressed row storage}
  \begin{itemize}
  \item The problem with CRS is that inserting a new nonzero element somewhere
    in the matrix may involve a lot of data shifting.
  \item Other sparse structures exist that are optimized for insertion but not
    for matrix-vector operations.
  \item Various workarounds exist, such as assembling the final CSR structure
    after all elements have been set, or (in this case) pre-declare the sparsity
    pattern before inserting values.
  \item It is necessary to know how many nonzero elements are on each row. For
    MPI applications, it is also good to know how many nonzero elements are off
    or on the diagonal.
  \end{itemize}
\end{frame}


\begin{frame}[fragile]
  \frametitle{PETSc Matrices: specifying the sparsity pattern}

To avoid overallocation of memory the structured should be initialized exactly for non-zero entries.

\medskip
Creation with specification of sparsity pattern:
\begin{lstlisting}[style=c]
// Specify non-zero entries for each row on DIAGONAL and OFF-DIAGONAL
MatCreateAIJ(MPI::DOLFIN_COMM, 
   M, N, PETSC_DETERMINE, PETSC_DETERMINE,
   PETSC_DETERMINE, (PetscInt*) d_nzrow,
   PETSC_DETERMINE, (PetscInt*) o_nzrow, &A);
\end{lstlisting}


\medskip
This is a two-dimensional extension of the vector range + ghosts:
\begin{enumerate}
\item DIAGONAL: the block owned by the current process
\item OFF-DIAGONAL: entries on columns outside the column range
\end{enumerate}

\end{frame}

\begin{frame}
  \frametitle{Poisson solver in PETSc}
  \begin{itemize}
  \item As is common with C libraries like this, much of our code will be
    initialization.
  \item PETSc has tools for finite difference methods, but we will avoid them.
  \item We focus here only on the setup of the vector, the matrix and the
    solution of the linear system.
  \end{itemize}
\end{frame}

\begin{frame}
  \frametitle{Problem}
  \begin{itemize}
  \item Solve
    \[
      \mA \bm x = \bb, \qquad
      \bb, \bm x \in \mathbb{R}^N, \qquad
      \mA \in \xMNR
    \]
    where $\mA$ can be the system resulting from discretizing a Poisson problem
    using finite differences.
  \bigskip
  \item We use standard notation for matrices and vectors, i.e.
    \[
      \begin{pmatrix}
        a_{1,1} & a_{1,2} & \cdots & a_{1,N} \\
        a_{2,1} & a_{2,2} & \cdots & a_{2,N} \\
        \vdots  & \vdots  & \ddots & \vdots  \\
        a_{N,1} & a_{N,2} & \cdots & a_{N,N}
      \end{pmatrix}
      \begin{pmatrix} x_1 \\ x_2 \\ \vdots \\ x_N \end{pmatrix} =
      \begin{pmatrix} b_1 \\ b_2 \\ \vdots \\ b_N \end{pmatrix}
    \]
  \end{itemize}
\end{frame}


\begin{frame}[fragile]
  \frametitle{Vector setup}
  \begin{lstlisting}[style=c]
  Vec b;

  // Use PETSC_COMM_SELF to get a seq vector
  VecCreate(PETSC_COMM_WORLD, &b);

  // Local and global sizes
  VecSetSizes(b, PETSC_DECIDE, m*m);

  double hh = 1.0 / n / n;
  for (size_t j = 0; j < m*m; j++)
    VecSetValue(b, j, hh, INSERT_VALUES);
  \end{lstlisting}
\end{frame}

\begin{frame}[fragile]
  \frametitle{Matrix setup}
  \begin{lstlisting}[style=c, basicstyle=\ttfamily\scriptsize]
  Mat A;
  MatCreate(PETSC_COMM_WORLD, &A);
  MatSetType(A, MATSEQAIJ);
  MatSetSizes(A, PETSC_DECIDE, PETSC_DECIDE, m*m, m*m);

  // Diagonal
  for (size_t i = 0; i < m*m, i++)
    MatSetValue(A, i, i, 4.0, INSERT_VALUES);

  // L-R coupling
  for (size_t i = 0; i < m*m - 1, i++) {
    if (i % m != m-1)
      MatSetValue(A, i, i+1, -1.0, INSERT_VALUES);
    if (i % m)
      MatSetValue(A, i, i-1, -1.0, INSERT_VALUES);
  }

  // U-D coupling
  for (size_t i = m; i < m*m, i++) {
    MatSetValue(A, i, i-m, -1.0, INSERT_VALUES);
    MatSetValue(A, i-m, i, -1.0, INSERT_VALUES);
  }
  \end{lstlisting}
\end{frame}

\begin{frame}[fragile]
  \frametitle{Synchronization}
  The vector and matrix must be \emph{assembled} before we can use them. This
  might involve communication.
  \begin{lstlisting}[style=c]
    VecAssemblyBegin(b);
    VecAssemblyEnd(b);
    MatAssemblyBegin(A, MAT_FINAL_ASSEMBLY);
    MatAssemblyEnd(A, MAT_FINAL_ASSEMBLY);
  \end{lstlisting}
\end{frame}

\begin{frame}[fragile]
  \frametitle{Solver setup}
  \begin{lstlisting}[style=c]
  KSP sol;
  KSPCreate(PETSC_COMM_WORLD, &sol);

  KSPSetType(ksp, "cg");
  KSPSetTolerances(ksp, 1e-10, 1e-10, 1e6, 10000);
  KSPSetOperators(ksp, A, A);

  PC pc;
  KSPGetPC(ksp, &pc);
  PCSetType(pc, "ilu");
  PCSetFromOptions(pc);
  PCSetUp(pc);

  KSPSetFromOptions(ksp);
  KSPSetUp(ksp);
  \end{lstlisting}
\end{frame}

\begin{frame}[fragile]
  \frametitle{Solving}
  \begin{itemize}
  \item To solve,
    \begin{lstlisting}[style=c]
  Vec x;
  VecDuplicate(b, &x);
  KSPSolve(ksp, b, x);
    \end{lstlisting}
  \item You will probably find that the solver performs rather poorly, both in
    serial and parallel.
  \item Reasons for this include
    \begin{enumerate}
    \item We fill the vector element by element,
    \item We fill the matrix element by element,
    \item We haven't declared the sparsity pattern of the matrix, and
    \item All processes fill all elements.
    \end{enumerate}
  \end{itemize}
\end{frame}

\begin{frame}[fragile]
  \frametitle{Fill the whole vector at once}
  \begin{lstlisting}[style=c]
  PetscInt low, high;
  VecGetOwnershipRange(b, &low, &high);

  PetscInt indices[high-low];
  double vals[high-low];
  for (size_t i = 0; i < high - 1; i++) {
    inds[i] = low + i;
    vals[i] = hh;
  }
  VecSetValues(b, high-low, inds, vals, INSERT_VALUES);
  free(inds); free(vals);
  \end{lstlisting}
  Similar but more involved for the matrix.
\end{frame}


\begin{frame}[fragile]
  \frametitle{Declare the sparsity pattern}
  \begin{lstlisting}[style=c, basicstyle=\ttfamily\scriptsize]
    PetscInt first, last;
    MatGetOwnershipRange(A, &first, &last);
    PetscInt d_nz = (PetscInt *)
      malloc((last - first) * sizeof(PetscInt));
    PetscInt o_nz = (PetscInt *)
      malloc((last - first) * sizeof(PetscInt));

    for (size_t i = first; i < last; i++) {
      d_nz[i - first] = 5;
      o_nz[i - first] = 5;
    }
    MatMPIAIJSetPreallocation(
      A, PETSC_DEFAULT, d_nz, PETSC_DEFAULT, o_nz
    );
  \end{lstlisting}
  Here we have slightly overallocated for the sake of simplicity.
\end{frame}

\begin{frame}[fragile]
  \frametitle{Fully declare the sparsity pattern}
  \begin{itemize}
  \item We can also pin the sparsity pattern completely.
    \begin{lstlisting}[style=c, basicstyle=\ttfamily\scriptsize]
PetscInt d_nz = (PetscInt *) malloc(m*m * sizeof(PetscInt));
int total = 0;
// count number of nonzeros per row -> d_nz and total

MatSeqAIJSetPreallocation(A, PETSC_DEFAULT, d_nz);

PetscInt col = (PetscInt *) malloc(total * sizeof(PetscInt));
// compute the actual column indices
MatSeqAIJSetColumnIndices(A, col);
    \end{lstlisting}
  \item This way the matrix format will never change when adding values, which
    allows for multi-threaded assembly.
  \item Unfortunately it doesn't work in hybrid mode. (There is no
    \texttt{MatMPIAIJSetColumnIndices}.)
  \end{itemize}
\end{frame}

\begin{frame}
  \frametitle{Fill only your own elements}
  \begin{itemize}
  \item We can use \texttt{VecGetOwnershipRange} and
    \texttt{MatGetOwnershipRange} to get the global indices that are assigned to
    our own process.
  \item Then, each process can set just those indices.
  \item Note that in PETSc, a process owns whole \emph{rows} of the matrix, not
    columns as we have been (often) using in this course.
  \end{itemize}
\end{frame}

\end{document}

