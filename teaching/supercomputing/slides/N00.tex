\input{preamble}

\title{Supercomputing environment}
\institute{NTNU, IMF}
\date{February 21. 2018}
%\author{Aurélien Larcher}

\maketitle

%------------------------------------------------------------------------------
\begin{frame}
  \frametitle{Supercomputing environment}
  \begin{itemize}
  \item Supercomputers use UNIX-type operating systems.
  \item Predominantly Linux.
  \item Using a shell interpreter is the only way to interact with the system.
  \end{itemize}

\bigskip
Documentation and tutorials are usually offered on the System Administration group's website: \url{https://www.hpc.ntnu.no/display/hpc/User+Guide}
\end{frame}

\begin{frame}
  \frametitle{Login}
  \begin{itemize}
  \item Non-graphical interaction with the supercomputer.
  \item Two kinds of nodes: login/interactive nodes and compute nodes.
  \item Login is handled through \emph{Secure SHell} (SSH).
  \item On Linux/UNIX/MacOS: pre-installed OpenSSH.
  \item On Windows: third-party client PuTTy.
  \end{itemize}
\end{frame}

\begin{frame}[fragile]
  \frametitle{Login}

Three ingredients:
  \begin{itemize}
  \item Username: NTNU login name.
  \item Host: \texttt{training.hpc.ntnu.no}.
  \item Credential: a password or an authentication key.
  \end{itemize}

\bigskip
\begin{lstlisting}[style=shell]
ssh username@training.hpc.ntnu.no
\end{lstlisting}

\end{frame}

\begin{frame}[fragile]
  \frametitle{Login}
\tiny{
\begin{lstlisting}[style=shell]
Last login: Wed Feb 21 16:24:19 2018 from 129.241.15.225

 /$$$$$$$$           /$$                       /$$       /$$$$$$       /$$                    
| $$_____/          |__/                      /$$/      |_  $$_/      | $$                    
| $$        /$$$$$$  /$$  /$$$$$$$           /$$/         | $$    /$$$$$$$ /$$   /$$ /$$$$$$$ 
| $$$$$    /$$__  $$| $$ /$$_____/          /$$/          | $$   /$$__  $$| $$  | $$| $$__  $$
| $$__/   | $$  \ $$| $$| $$               /$$/           | $$  | $$  | $$| $$  | $$| $$  \ $$
| $$      | $$  | $$| $$| $$              /$$/            | $$  | $$  | $$| $$  | $$| $$  | $$
| $$$$$$$$| $$$$$$$/| $$|  $$$$$$$       /$$/            /$$$$$$|  $$$$$$$|  $$$$$$/| $$  | $$
|________/| $$____/ |__/ \_______/      |__/            |______/ \_______/ \______/ |__/  |__/
          | $$                                                                                
          | $$                                                                                
          |__/                      


To run jobs you need to generate and add public keys to authorized file:
NB!
NB! this will overwrite you existing keypair
NB!

# ssh-keygen -b 2048 -f $HOME/.ssh/id_rsa -t rsa -q -N ""
# cat $HOME/.ssh/id_rsa.pub >> $HOME/.ssh/authorized_keys

to list available modules software:
# module spider

Starting 6th of february, 2018:
Courses lectured over several afternoons will give a introduction to parallel programming.
Registration: Send an e-mail to: adm@hpc.ntnu.no
https://www.hpc.ntnu.no/display/hpc/Introduction+to+parallel+programming

To get help and support, please send email to:  help@hpc.ntnu.no
Or check our web page: http://www.hpc.ntnu.no/

Disk quota are now enabled on idun, to see your quota, command: dusage 
(or quota  , see 'man quota')

Be Nice!
\end{lstlisting}
}
\end{frame}

\begin{frame}
  \frametitle{File transfer}
  \begin{itemize}
  \item File transfers is performed using \emph{Secure Copy} (\texttt{scp}).
  \item On Linux/UNIX/MacOS: pre-installed OpenSSH.
  \item On Window: third-party client \emph{WinSCP}.
  \end{itemize}

\bigskip
Advice: for source code and result files in text format use a revision control system like GIT.

\end{frame}

\begin{frame}[fragile]
  \frametitle{Authentication with SSH key}
  \begin{itemize}
  \item Avoid typing your password, use key authentication.
  \item Type only return if you want an empty passphrase.
  \item Generate an SSH key on the \textbf{local}:
\begin{lstlisting}[style=shell]
ssh-keygen
\end{lstlisting}
  \item Copy the content of \textbf{public} key \texttt{id\_rsa.pub} to the \textbf{remote} host file \texttt{scp}
\tiny{
\begin{lstlisting}[style=shell]
 scp ~/.ssh/id_rsa.pub username@training.hpc.ntnu.no:~/.ssh/authorized_keys
\end{lstlisting}
}
  \end{itemize}

\bigskip
Tutorial:

\medskip
\small{
\url{https://debian-administration.org/article/530/SSH_with_authentication_key_instead_of_password}
}
\end{frame}

\begin{frame}
  \frametitle{Editing files}

For such small project, only a good text editor is required:
  \begin{itemize}
  \item Emacs (use locally if you can)
  \item Vim (handy for using remotely, a bit of a learning curve)
  \item Nano (simpler than Vim)
  \item Gedit (nice graphical editor)
  \item Kate (same, not installed on Lille)
  \item Notepad++ (good for Windows users, not installed on Lille)
  \item \ldots
  \end{itemize}

\medskip
In practice, non-graphical editors are preferred since working on a login node requires using the terminal: most people use Vim, Emacs, or Nano.

\end{frame}

\begin{frame}[fragile]
  \frametitle{Graphical display (X11 forwarding)}
  If you want to run graphical programs on Lille you have to tunnel the display
  through ssh. This is called \emph{X forwarding}.
  \begin{itemize}
  \item In Linux, it's quite easy:
\begin{lstlisting}[style=shell]
ssh -X username@training.hpc.ntnu.no
\end{lstlisting}
  Or in your \texttt{\textasciitilde/.ssh/config}:
\begin{lstlisting}[style=shell]
ForwardX11 Yes
\end{lstlisting}
    \item In OSX, you have to start \texttt{X11.app}, then do the same.
    \item In Windows, you can use \emph{X-Win32}, which is available on progdist.
  \end{itemize}

This is usually not required and puts unnecessarily load on the login nodes.
\end{frame}

\begin{frame}[fragile]
  \frametitle{Modules}
  As people sharing a supercomputer have different needs, the tools cannot be all installed in the default system directories.
  Software is offered through a \emph{modules} system. They will not be available to you until you load the module in question.
  \begin{itemize}
  \item List all available modules:
\begin{lstlisting}[style=shell]
module spider
\end{lstlisting}
  \item List available modules:
\begin{lstlisting}[style=shell]
module avail
\end{lstlisting}
  \item Load a module:
\begin{lstlisting}[style=shell]
module load gcc
\end{lstlisting}
  \item Load a module with a specific version:
\begin{lstlisting}[style=shell]
module load gcc/6.3.0
module load openmpi/2.0.1
\end{lstlisting}
  \item List loaded modules:
\begin{lstlisting}[style=shell]
module list
\end{lstlisting}
  \end{itemize}
\end{frame}

\begin{frame}[fragile]
  \frametitle{Modules}
Some relevant modules for this course:
\begin{itemize}
  \item \texttt{gcc/6.3.0}: GCC compilers (\texttt{gcc}, \texttt{g++} and
    \texttt{gfortran}).
  \item \texttt{openmpi/2.0.1}: OpenMPI implementation of MPI (\emph{Message Passing Toolkit}).
  \item \texttt{openblas/0.2.19}: BLAS library.
\end{itemize}

\medskip
Note that if you use CMake to build your programs, you may need to pass the compiler you want to use:
\begin{lstlisting}[style=shell]
mkdir build
cd build
CXX=g++ CC=gcc FC=gfortran cmake ..
\end{lstlisting}
\end{frame}

\begin{frame}[fragile]
  \frametitle{Modules}

\tiny{
\begin{lstlisting}[style=shell]
[aurelila@lille-login2 ~]$ module avail

------------------------------- /share/apps/modules/all/Core --------------------------------
   EasyBuild/3.3.0        Go/1.8.1             foss/2017a                      (D)
   FLUENT/18.0            Java/1.8.0_92        icc/2017.1.132-GCC-6.3.0-2.27
   FLUENT/18.2     (D)    MATLAB/2016b         ifort/2017.1.132-GCC-6.3.0-2.27
   GCC/4.9.3-2.25         MATLAB/2017a  (D)    intel/2017a
   GCC/5.4.0-2.26         foss/2016a
   GCC/6.3.0-2.27  (D)    foss/2016b

------------------------------- /share/apps/modulefiles/Core --------------------------------
   easybuild/2.9.0    gcc/6.2.0        matlab/R2016b
   gcc/4.9.4          gcc/6.3.0 (D)    python/2.7.3

  Where:
   D:  Default Module

Use "module spider" to find all possible modules.
Use "module keyword key1 key2 ..." to search for all possible modules matching any of the
"keys".
\end{lstlisting}
}

\end{frame}

\begin{frame}[fragile]
  \frametitle{Modules}

Modules have dependencies: for example openmpi cannot be loaded unless a compiler has been loaded already:
{\tiny
\begin{lstlisting}[style=shell]
[aurelila@lille-login2 ~]$ module load openmpi
Lmod has detected the following error:  These module(s) exist but cannot be
loaded as requested: "openmpi"

   Try: "module spider openmpi" to see how to load the module(s).

\end{lstlisting}
}

Load gcc first:
{\small
\begin{lstlisting}[style=shell]
[aurelila@lille-login2 ~]$ module load gcc openmpi
\end{lstlisting}
}

\medskip
You can add this line in your shell profile or write a script to do it.


\end{frame}

\begin{frame}[fragile]
\frametitle{Batch scheduler/Queuing system}

To schedule jobs run by users, a queueing system is installed on supercomputers:
\begin{itemize}
\item each job submitted is appended to the queue with a given priority,
\item then launched when reaching the top of the queue (\texttt{workq} on Lille),
\item status (success/failure) is reported accordingly,
\item computational time n core.hour is charged to the project (maximum resource is 20 processes on 2 nodes).
\end{itemize}

\bigskip
A simple job:
{\small
\begin{lstlisting}[style=shell]
echo "sleep 30;echo hello world"|qsub -q training \
  -W group_list=itea_lille-tma4280 \
 -lselect=2:ncpus=20:mpiprocs=20
\end{lstlisting}
}
If the queue you need to use has another name, substitute the \texttt{training} argument.
\end{frame}


\begin{frame}[fragile]
\frametitle{Batch scheduler/Queuing system}

The status of the training queue can be inspected:
{\tiny
\begin{lstlisting}[style=shell]
[aurelila@lille-login2 ~]$ qstat -Q training
Queue              Max   Tot Ena Str   Que   Run   Hld   Wat   Trn   Ext Type
---------------- ----- ----- --- --- ----- ----- ----- ----- ----- ----- ----
training             0     0 yes yes     0     0     0     0     0     0 Exec
\end{lstlisting}
}
with \texttt{training} the name of requested queue for example.

\medskip
The status of jobs for a given username can be displayed:
\begin{lstlisting}[style=shell]
[aurelila@lille-login2 ~]$ qstat -u username
\end{lstlisting}



\end{frame}

\begin{frame}[fragile]
  \frametitle{Running jobs on Lille}
  After compiling your program, you have to write a \emph{job script}. Example
  (the \emph{pi} program):
\begin{lstlisting}[style=shell]
  #!/bin/bash

  #PBS -N pi
  #PBS -A itea_lille-tma4280
  #PBS -W group_list=itea_lille-tma4280
  #PBS -l walltime=00:01:00
  #PBS -l select=2:ncpus=20:mpiprocs=16

  cd $PBS_O_WORKDIR
  module load openmpi
  mpiexec ./pi 1000000
\end{lstlisting}
%$
\end{frame}

\begin{frame}[fragile]
  \frametitle{Running jobs on Lille}
\begin{lstlisting}[style=shell]
  #PBS -N pi
\end{lstlisting}
  My job is called ``pi''.
\begin{lstlisting}[style=shell]
  #PBS -A itea_lille-tma4280
\end{lstlisting}
  The time spent executing this job should be charged to \texttt{itea\_lille-tma4280}.
\begin{lstlisting}[style=shell]
  #PBS -l walltime=00:01:00
\end{lstlisting}
  The walltime limit for this job is one minute.
\begin{lstlisting}[style=shell]
  #PBS -l select=2:ncpus=20:mpiprocs=16
\end{lstlisting}
  I want two units of 20 CPUs each (two nodes, that is) and I want 16 processes
  on each of them (32 in total). On Lille, \texttt{ncpus} should \emph{always}
  be equal to 20.
\end{frame}

\begin{frame}[fragile]
  \frametitle{Running jobs on Lille}
\begin{lstlisting}[style=shell]
  cd $PBS_O_WORKDIR
\end{lstlisting}
%$
  Ensure that we are in the correct directory. This should always be in your job
  script.
\begin{lstlisting}[style=shell]
  module load openmpi
\end{lstlisting}
  Make sure the \texttt{openmpi} module is loaded so that the \texttt{mpiexec}
  command is available to run MPI programs.
\begin{lstlisting}[style=shell]
  mpiexec ./pi 1000000
\end{lstlisting}
  Run the program.
\end{frame}

\begin{frame}[fragile]
  \frametitle{Running jobs on Lille}
  Submit a job using \texttt{qsub}:
\begin{lstlisting}[style=shell]
  qsub job.sh
  5723717.service2
\end{lstlisting}
  \texttt{qsub} will reply with a job ID number. You can ask for the status of
  your job with
\begin{lstlisting}[style=shell]
  qstat -f 5723717.service2
\end{lstlisting}
  or see a list of all jobs running and queued
\begin{lstlisting}[style=shell]
  qstat
\end{lstlisting}
\end{frame}

\begin{frame}[fragile]
  \frametitle{Running jobs on Lille}
  When the program has completed, the accumulated output will be written to
  files in the same folder you launched it from.
\begin{lstlisting}[style=shell]
  ls
  job.sh  pi  pi.c  pi.e5723717  pi.o5723717
\end{lstlisting}
  The \texttt{e}-file contains stderr (empty?) and the \texttt{o}-file contains
  output from stdout (the most interesting one).
\begin{lstlisting}[style=shell,basicstyle=\ttfamily\footnotesize]
  cat pi.o5723717
  Agent pid 21651
  pi=3.141593e+00, error=8.437695e-14, duration=2.177000e-03
  Start Epilogue v3.0.1 Wed Jan 27 14:18:27 CET 2016
  clean up
  End Epilogue v3.0.1 Wed Jan 27 14:18:28 CET 2016
\end{lstlisting}
\end{frame}

\begin{frame}[fragile]
  \frametitle{Other PBS options}
\begin{lstlisting}[style=shell]
  #PBS -o stdout
  #PBS -e stderr
\end{lstlisting}
I want my output files to have more sensible names.
\begin{lstlisting}[style=shell]
  #PBS -m abe
\end{lstlisting}
I want an e-mail notification when the job starts (\texttt{b}), ends
(\texttt{e}) or if it aborts (\texttt{a}).
\begin{lstlisting}[style=shell]
  #PBS -M some@where.com
\end{lstlisting}
\ldots and this is where that e-mail should be sent to.
\begin{lstlisting}[style=shell]
  #PBS -l ...:ompthreads=16
\end{lstlisting}
for 16 OpenMP threads per process.

\medskip
See \url{https://www.hpc.ntnu.no/display/hpc/PBS+Professional}
\end{frame}

\begin{frame}
  \frametitle{More information}
  \begin{center}
    The NTNU HPC Wiki has a very good user guide. \\~\\
    \url{https://www.hpc.ntnu.no/display/hpc/User+Guide}
  \end{center}
\end{frame}

\input{postamble}
